\documentclass{howto}

\title{Macintosh Library Modules}

\author{Guido van Rossum \\
	Fred L. Drake, Jr., editor}
\authoraddress{
	BeOpen PythonLabs \\
	E-mail: \email{python-docs@python.org}
}

\date{September 5, 2000}			% XXX update before release!
\release{1.6}


\makeindex			% tell \index to actually write the
				% .idx file
\makemodindex			% ... and the module index as well.
%\ignorePlatformAnnotation{Mac}


\begin{document}

\maketitle

\ifhtml
\chapter*{Front Matter\label{front}}
\fi

Copyright \copyright{} 1995-2000 Corporation for National Research
Initiatives.  All rights reserved.

Copyright \copyright{} 1991-1995 Stichting Mathematisch Centrum.  All
rights reserved.

\centerline{\strong{CNRI OPEN SOURCE LICENSE AGREEMENT}}


IMPORTANT: PLEASE READ THE FOLLOWING AGREEMENT CAREFULLY.

BY CLICKING ON ``ACCEPT'' WHERE INDICATED BELOW, OR BY COPYING,
INSTALLING OR OTHERWISE USING PYTHON 1.6 SOFTWARE, YOU ARE DEEMED TO
HAVE AGREED TO THE TERMS AND CONDITIONS OF THIS LICENSE AGREEMENT.

\begin{enumerate}
\item
This LICENSE AGREEMENT is between the Corporation for National
Research Initiatives, having an office at 1895 Preston White Drive,
Reston, VA 20191 (``CNRI''), and the Individual or Organization
(``Licensee'') accessing and otherwise using Python 1.6 software in
source or binary form and its associated documentation, as released at
the www.python.org Internet site on September 5, 2000 (``Python 1.6'').

\item
Subject to the terms and conditions of this License Agreement, CNRI
hereby grants Licensee a nonexclusive, royalty-free, world-wide
license to reproduce, analyze, test, perform and/or display publicly,
prepare derivative works, distribute, and otherwise use Python 1.6
alone or in any derivative version, provided, however, that CNRI's
License Agreement and CNRI's notice of copyright, i.e., ``Copyright (c)
1995-2000 Corporation for National Research Initiatives; All Rights
Reserved'' are retained in Python 1.6 alone or in any derivative
version prepared by Licensee.

Alternately, in lieu of CNRI's License Agreement, Licensee may
substitute the following text (omitting the quotes): ``Python 1.6 is
made available subject to the terms and conditions in CNRI's License
Agreement.  This Agreement together with Python 1.6 may be located on
the Internet using the following unique, persistent identifier (known
as a handle): 1895.22/1012.  This Agreement may also be obtained from a
proxy server on the Internet using the following URL:
http://hdl.handle.net/1895.22/1012''.

\item
In the event Licensee prepares a derivative work that is based on
or incorporates Python 1.6 or any part thereof, and wants to make the
derivative work available to others as provided herein, then Licensee
hereby agrees to include in any such work a brief summary of the
changes made to Python 1.6.

\item
CNRI is making Python 1.6 available to Licensee on an ``AS IS''
basis.  CNRI MAKES NO REPRESENTATIONS OR WARRANTIES, EXPRESS OR
IMPLIED.  BY WAY OF EXAMPLE, BUT NOT LIMITATION, CNRI MAKES NO AND
DISCLAIMS ANY REPRESENTATION OR WARRANTY OF MERCHANTABILITY OR FITNESS
FOR ANY PARTICULAR PURPOSE OR THAT THE USE OF PYTHON 1.6 WILL NOT
INFRINGE ANY THIRD PARTY RIGHTS.

\item
CNRI SHALL NOT BE LIABLE TO LICENSEE OR ANY OTHER USERS OF PYTHON
1.6 FOR ANY INCIDENTAL, SPECIAL, OR CONSEQUENTIAL DAMAGES OR LOSS AS A
RESULT OF MODIFYING, DISTRIBUTING, OR OTHERWISE USING PYTHON 1.6, OR
ANY DERIVATIVE THEREOF, EVEN IF ADVISED OF THE POSSIBILITY THEREOF.

\item
This License Agreement will automatically terminate upon a material
breach of its terms and conditions.

\item
This License Agreement shall be governed by and interpreted in all
respects by the law of the State of Virginia, excluding conflict of
law provisions.  Nothing in this License Agreement shall be deemed to
create any relationship of agency, partnership, or joint venture
between CNRI and Licensee.  This License Agreement does not grant
permission to use CNRI trademarks or trade name in a trademark sense
to endorse or promote products or services of Licensee, or any third
party.

\item
By clicking on the ``ACCEPT'' button where indicated, or by copying,
installing or otherwise using Python 1.6, Licensee agrees to be bound
by the terms and conditions of this License Agreement.

\centerline{ACCEPT}
\end{enumerate}


\begin{abstract}

\noindent
This library reference manual documents Python's extensions for the
Macintosh.  It should be used in conjunction with the
\citetitle[../lib/lib.html]{Python Library Reference}, which documents
the standard library and built-in types.

This manual assumes basic knowledge about the Python language.  For an
informal introduction to Python, see the
\citetitle[../tut/tut.html]{Python Tutorial}; the
\citetitle[../ref/ref.html]{Python Reference Manual} remains the
highest authority on syntactic and semantic questions.  Finally, the
manual entitled \citetitle[../ext/ext.html]{Extending and Embedding
the Python Interpreter} describes how to add new extensions to Python
and how to embed it in other applications.

\end{abstract}

\tableofcontents

\section{Introduction}
\label{intro}

The modules in this manual are available on the Apple Macintosh only.

Aside from the modules described here there are also interfaces to
various MacOS toolboxes, which are currently not extensively
described. The toolboxes for which modules exist are:
\module{AE} (Apple Events),
\module{Cm} (Component Manager),
\module{Ctl} (Control Manager),
\module{Dlg} (Dialog Manager),
\module{Evt} (Event Manager),
\module{Fm} (Font Manager),
\module{List} (List Manager),
\module{Menu} (Moenu Manager),
\module{Qd} (QuickDraw),
\module{Qt} (QuickTime),
\module{Res} (Resource Manager and Handles),
\module{Scrap} (Scrap Manager),
\module{Snd} (Sound Manager),
\module{TE} (TextEdit),
\module{Waste} (non-Apple \program{TextEdit} replacement) and
\module{Win} (Window Manager).

If applicable the module will define a number of Python objects for
the various structures declared by the toolbox, and operations will be
implemented as methods of the object. Other operations will be
implemented as functions in the module. Not all operations possible in
\C{} will also be possible in Python (callbacks are often a problem), and
parameters will occasionally be different in Python (input and output
buffers, especially). All methods and functions have a \code{__doc__}
string describing their arguments and return values, and for
additional description you are referred to \citetitle{Inside
Macintosh} or similar works.

The following modules are documented here:

\localmoduletable


\section{\module{mac} ---
         Implementations for the \module{os} module}

\declaremodule{builtin}{mac}
  \platform{Mac}
\modulesynopsis{Implementations for the \module{os} module.}


This module implements the operating system dependent functionality
provided by the standard module \module{os}\refstmodindex{os}.  It is
best accessed through the \module{os} module.

The following functions are available in this module:
\function{chdir()},
\function{close()},
\function{dup()},
\function{fdopen()},
\function{getcwd()},
\function{lseek()},
\function{listdir()},
\function{mkdir()},
\function{open()},
\function{read()},
\function{rename()},
\function{rmdir()},
\function{stat()},
\function{sync()},
\function{unlink()},
\function{write()},
as well as the exception \exception{error}. Note that the times
returned by \function{stat()} are floating-point values, like all time
values in MacPython.

One additional function is available:

\begin{funcdesc}{xstat}{path}
  This function returns the same information as \function{stat()}, but
  with three additional values appended: the size of the resource fork
  of the file and its 4-character creator and type.
\end{funcdesc}


\section{\module{macpath} ---
         MacOS path manipulation functions}

\declaremodule{standard}{macpath}
% Could be labeled \platform{Mac}, but the module should work anywhere and
% is distributed with the standard library.
\modulesynopsis{MacOS path manipulation functions.}


This module is the Macintosh implementation of the \module{os.path}
module.  It is most portably accessed as
\module{os.path}\refstmodindex{os.path}.  Refer to the
\citetitle[../lib/lib.html]{Python Library Reference} for
documentation of \module{os.path}.

The following functions are available in this module:
\function{normcase()},
\function{normpath()},
\function{isabs()},
\function{join()},
\function{split()},
\function{isdir()},
\function{isfile()},
\function{walk()},
\function{exists()}.
For other functions available in \module{os.path} dummy counterparts
are available.
			% MACINTOSH ONLY
\section{\module{ctb} ---
         Interface to the Communications Tool Box}

\declaremodule{builtin}{ctb}
  \platform{Mac}
\modulesynopsis{Interfaces to the Communications Tool Box.  Only the Connection
Manager is supported.}


This module provides a partial interface to the Macintosh
Communications Toolbox. Currently, only Connection Manager tools are
supported.  It may not be available in all Mac Python versions.
\index{Communications Toolbox, Macintosh}
\index{Macintosh Communications Toolbox}
\index{Connection Manager}

\begin{datadesc}{error}
The exception raised on errors.
\end{datadesc}

\begin{datadesc}{cmData}
\dataline{cmCntl}
\dataline{cmAttn}
Flags for the \var{channel} argument of the \method{Read()} and
\method{Write()} methods.
\end{datadesc}

\begin{datadesc}{cmFlagsEOM}
End-of-message flag for \method{Read()} and \method{Write()}.
\end{datadesc}

\begin{datadesc}{choose*}
Values returned by \method{Choose()}.
\end{datadesc}

\begin{datadesc}{cmStatus*}
Bits in the status as returned by \method{Status()}.
\end{datadesc}

\begin{funcdesc}{available}{}
Return \code{1} if the Communication Toolbox is available, zero otherwise.
\end{funcdesc}

\begin{funcdesc}{CMNew}{name, sizes}
Create a connection object using the connection tool named
\var{name}. \var{sizes} is a 6-tuple given buffer sizes for data in,
data out, control in, control out, attention in and attention out.
Alternatively, passing \code{None} for \var{sizes} will result in
default buffer sizes.
\end{funcdesc}


\subsection{Connection Objects \label{connection-object}}

For all connection methods that take a \var{timeout} argument, a value
of \code{-1} is indefinite, meaning that the command runs to completion.

\begin{memberdesc}[connection]{callback}
If this member is set to a value other than \code{None} it should point
to a function accepting a single argument (the connection
object). This will make all connection object methods work
asynchronously, with the callback routine being called upon
completion.

\emph{Note:} for reasons beyond my understanding the callback routine
is currently never called. You are advised against using asynchronous
calls for the time being.
\end{memberdesc}


\begin{methoddesc}[connection]{Open}{timeout}
Open an outgoing connection, waiting at most \var{timeout} seconds for
the connection to be established.
\end{methoddesc}

\begin{methoddesc}[connection]{Listen}{timeout}
Wait for an incoming connection. Stop waiting after \var{timeout}
seconds. This call is only meaningful to some tools.
\end{methoddesc}

\begin{methoddesc}[connection]{accept}{yesno}
Accept (when \var{yesno} is non-zero) or reject an incoming call after
\method{Listen()} returned.
\end{methoddesc}

\begin{methoddesc}[connection]{Close}{timeout, now}
Close a connection. When \var{now} is zero, the close is orderly
(i.e.\ outstanding output is flushed, etc.)\ with a timeout of
\var{timeout} seconds. When \var{now} is non-zero the close is
immediate, discarding output.
\end{methoddesc}

\begin{methoddesc}[connection]{Read}{len, chan, timeout}
Read \var{len} bytes, or until \var{timeout} seconds have passed, from 
the channel \var{chan} (which is one of \constant{cmData},
\constant{cmCntl} or \constant{cmAttn}). Return a 2-tuple:\ the data
read and the end-of-message flag, \constant{cmFlagsEOM}.
\end{methoddesc}

\begin{methoddesc}[connection]{Write}{buf, chan, timeout, eom}
Write \var{buf} to channel \var{chan}, aborting after \var{timeout}
seconds. When \var{eom} has the value \constant{cmFlagsEOM}, an
end-of-message indicator will be written after the data (if this
concept has a meaning for this communication tool). The method returns
the number of bytes written.
\end{methoddesc}

\begin{methoddesc}[connection]{Status}{}
Return connection status as the 2-tuple \code{(\var{sizes},
\var{flags})}. \var{sizes} is a 6-tuple giving the actual buffer sizes used
(see \function{CMNew()}), \var{flags} is a set of bits describing the state
of the connection.
\end{methoddesc}

\begin{methoddesc}[connection]{GetConfig}{}
Return the configuration string of the communication tool. These
configuration strings are tool-dependent, but usually easily parsed
and modified.
\end{methoddesc}

\begin{methoddesc}[connection]{SetConfig}{str}
Set the configuration string for the tool. The strings are parsed
left-to-right, with later values taking precedence. This means
individual configuration parameters can be modified by simply appending
something like \code{'baud 4800'} to the end of the string returned by
\method{GetConfig()} and passing that to this method. The method returns
the number of characters actually parsed by the tool before it
encountered an error (or completed successfully).
\end{methoddesc}

\begin{methoddesc}[connection]{Choose}{}
Present the user with a dialog to choose a communication tool and
configure it. If there is an outstanding connection some choices (like
selecting a different tool) may cause the connection to be
aborted. The return value (one of the \constant{choose*} constants) will
indicate this.
\end{methoddesc}

\begin{methoddesc}[connection]{Idle}{}
Give the tool a chance to use the processor. You should call this
method regularly.
\end{methoddesc}

\begin{methoddesc}[connection]{Abort}{}
Abort an outstanding asynchronous \method{Open()} or \method{Listen()}.
\end{methoddesc}

\begin{methoddesc}[connection]{Reset}{}
Reset a connection. Exact meaning depends on the tool.
\end{methoddesc}

\begin{methoddesc}[connection]{Break}{length}
Send a break. Whether this means anything, what it means and
interpretation of the \var{length} parameter depends on the tool in
use.
\end{methoddesc}

\section{\module{macconsole} ---
         Think C's console package}

\declaremodule{builtin}{macconsole}
  \platform{Mac}
\modulesynopsis{Think C's console package.}


This module is available on the Macintosh, provided Python has been
built using the Think C compiler. It provides an interface to the
Think console package, with which basic text windows can be created.

\begin{datadesc}{options}
An object allowing you to set various options when creating windows,
see below.
\end{datadesc}

\begin{datadesc}{C_ECHO}
\dataline{C_NOECHO}
\dataline{C_CBREAK}
\dataline{C_RAW}
Options for the \code{setmode} method. \constant{C_ECHO} and
\constant{C_CBREAK} enable character echo, the other two disable it,
\constant{C_ECHO} and \constant{C_NOECHO} enable line-oriented input
(erase/kill processing, etc).
\end{datadesc}

\begin{funcdesc}{copen}{}
Open a new console window. Return a console window object.
\end{funcdesc}

\begin{funcdesc}{fopen}{fp}
Return the console window object corresponding with the given file
object. \var{fp} should be one of \code{sys.stdin}, \code{sys.stdout} or
\code{sys.stderr}.
\end{funcdesc}

\subsection{macconsole options object}
These options are examined when a window is created:

\setindexsubitem{(macconsole option)}
\begin{datadesc}{top}
\dataline{left}
The origin of the window.
\end{datadesc}

\begin{datadesc}{nrows}
\dataline{ncols}
The size of the window.
\end{datadesc}

\begin{datadesc}{txFont}
\dataline{txSize}
\dataline{txStyle}
The font, fontsize and fontstyle to be used in the window.
\end{datadesc}

\begin{datadesc}{title}
The title of the window.
\end{datadesc}

\begin{datadesc}{pause_atexit}
If set non-zero, the window will wait for user action before closing.
\end{datadesc}

\subsection{console window object}

\setindexsubitem{(console window attribute)}

\begin{datadesc}{file}
The file object corresponding to this console window. If the file is
buffered, you should call \code{\var{file}.flush()} between
\code{write()} and \code{read()} calls.
\end{datadesc}

\setindexsubitem{(console window method)}

\begin{funcdesc}{setmode}{mode}
Set the input mode of the console to \constant{C_ECHO}, etc.
\end{funcdesc}

\begin{funcdesc}{settabs}{n}
Set the tabsize to \var{n} spaces.
\end{funcdesc}

\begin{funcdesc}{cleos}{}
Clear to end-of-screen.
\end{funcdesc}

\begin{funcdesc}{cleol}{}
Clear to end-of-line.
\end{funcdesc}

\begin{funcdesc}{inverse}{onoff}
Enable inverse-video mode:\ characters with the high bit set are
displayed in inverse video (this disables the upper half of a
non-\ASCII{} character set).
\end{funcdesc}

\begin{funcdesc}{gotoxy}{x, y}
Set the cursor to position \code{(\var{x}, \var{y})}.
\end{funcdesc}

\begin{funcdesc}{hide}{}
Hide the window, remembering the contents.
\end{funcdesc}

\begin{funcdesc}{show}{}
Show the window again.
\end{funcdesc}

\begin{funcdesc}{echo2printer}{}
Copy everything written to the window to the printer as well.
\end{funcdesc}

\section{\module{macdnr} ---
         Interface to the Macintosh Domain Name Resolver}

\declaremodule{builtin}{macdnr}
  \platform{Mac}
\modulesynopsis{Interfaces to the Macintosh Domain Name Resolver.}


This module provides an interface to the Macintosh Domain Name
Resolver.  It is usually used in conjunction with the \refmodule{mactcp}
module, to map hostnames to IP addresses.  It may not be available in
all Mac Python versions.
\index{Macintosh Domain Name Resolver}
\index{Domain Name Resolver, Macintosh}

The \module{macdnr} module defines the following functions:


\begin{funcdesc}{Open}{\optional{filename}}
Open the domain name resolver extension.  If \var{filename} is given it
should be the pathname of the extension, otherwise a default is
used.  Normally, this call is not needed since the other calls will
open the extension automatically.
\end{funcdesc}

\begin{funcdesc}{Close}{}
Close the resolver extension.  Again, not needed for normal use.
\end{funcdesc}

\begin{funcdesc}{StrToAddr}{hostname}
Look up the IP address for \var{hostname}.  This call returns a dnr
result object of the ``address'' variation.
\end{funcdesc}

\begin{funcdesc}{AddrToName}{addr}
Do a reverse lookup on the 32-bit integer IP-address
\var{addr}.  Returns a dnr result object of the ``address'' variation.
\end{funcdesc}

\begin{funcdesc}{AddrToStr}{addr}
Convert the 32-bit integer IP-address \var{addr} to a dotted-decimal
string.  Returns the string.
\end{funcdesc}

\begin{funcdesc}{HInfo}{hostname}
Query the nameservers for a \code{HInfo} record for host
\var{hostname}.  These records contain hardware and software
information about the machine in question (if they are available in
the first place).  Returns a dnr result object of the ``hinfo''
variety.
\end{funcdesc}

\begin{funcdesc}{MXInfo}{domain}
Query the nameservers for a mail exchanger for \var{domain}.  This is
the hostname of a host willing to accept SMTP\index{SMTP} mail for the
given domain.  Returns a dnr result object of the ``mx'' variety.
\end{funcdesc}


\subsection{DNR Result Objects \label{dnr-result-object}}

Since the DNR calls all execute asynchronously you do not get the
results back immediately.  Instead, you get a dnr result object.  You
can check this object to see whether the query is complete, and access
its attributes to obtain the information when it is.

Alternatively, you can also reference the result attributes directly,
this will result in an implicit wait for the query to complete.

The \member{rtnCode} and \member{cname} attributes are always
available, the others depend on the type of query (address, hinfo or
mx).


% Add args, as in {arg1, arg2 \optional{, arg3}}
\begin{methoddesc}[dnr result]{wait}{}
Wait for the query to complete.
\end{methoddesc}

% Add args, as in {arg1, arg2 \optional{, arg3}}
\begin{methoddesc}[dnr result]{isdone}{}
Return \code{1} if the query is complete.
\end{methoddesc}


\begin{memberdesc}[dnr result]{rtnCode}
The error code returned by the query.
\end{memberdesc}

\begin{memberdesc}[dnr result]{cname}
The canonical name of the host that was queried.
\end{memberdesc}

\begin{memberdesc}[dnr result]{ip0}
\memberline{ip1}
\memberline{ip2}
\memberline{ip3}
At most four integer IP addresses for this host.  Unused entries are
zero.  Valid only for address queries.
\end{memberdesc}

\begin{memberdesc}[dnr result]{cpuType}
\memberline{osType}
Textual strings giving the machine type an OS name.  Valid for ``hinfo''
queries.
\end{memberdesc}

\begin{memberdesc}[dnr result]{exchange}
The name of a mail-exchanger host.  Valid for ``mx'' queries.
\end{memberdesc}

\begin{memberdesc}[dnr result]{preference}
The preference of this mx record.  Not too useful, since the Macintosh
will only return a single mx record.  Valid for ``mx'' queries only.
\end{memberdesc}

The simplest way to use the module to convert names to dotted-decimal
strings, without worrying about idle time, etc:

\begin{verbatim}
>>> def gethostname(name):
...     import macdnr
...     dnrr = macdnr.StrToAddr(name)
...     return macdnr.AddrToStr(dnrr.ip0)
\end{verbatim}

\section{\module{macfs} ---
         Various file system services}

\declaremodule{builtin}{macfs}
  \platform{Mac}
\modulesynopsis{Support for FSSpec, the Alias Manager,
                \program{finder} aliases, and the Standard File package.}


This module provides access to Macintosh FSSpec handling, the Alias
Manager, \program{finder} aliases and the Standard File package.
\index{Macintosh Alias Manager}
\index{Alias Manager, Macintosh}
\index{Standard File}

Whenever a function or method expects a \var{file} argument, this
argument can be one of three things:\ (1) a full or partial Macintosh
pathname, (2) an \pytype{FSSpec} object or (3) a 3-tuple \code{(\var{wdRefNum},
\var{parID}, \var{name})} as described in \citetitle{Inside
Macintosh:\ Files}. A description of aliases and the Standard File
package can also be found there.

\begin{funcdesc}{FSSpec}{file}
Create an \pytype{FSSpec} object for the specified file.
\end{funcdesc}

\begin{funcdesc}{RawFSSpec}{data}
Create an \pytype{FSSpec} object given the raw data for the \C{}
structure for the \pytype{FSSpec} as a string.  This is mainly useful
if you have obtained an \pytype{FSSpec} structure over a network.
\end{funcdesc}

\begin{funcdesc}{RawAlias}{data}
Create an \pytype{Alias} object given the raw data for the \C{}
structure for the alias as a string.  This is mainly useful if you
have obtained an \pytype{FSSpec} structure over a network.
\end{funcdesc}

\begin{funcdesc}{FInfo}{}
Create a zero-filled \pytype{FInfo} object.
\end{funcdesc}

\begin{funcdesc}{ResolveAliasFile}{file}
Resolve an alias file. Returns a 3-tuple \code{(\var{fsspec},
\var{isfolder}, \var{aliased})} where \var{fsspec} is the resulting
\pytype{FSSpec} object, \var{isfolder} is true if \var{fsspec} points
to a folder and \var{aliased} is true if the file was an alias in the
first place (otherwise the \pytype{FSSpec} object for the file itself
is returned).
\end{funcdesc}

\begin{funcdesc}{StandardGetFile}{\optional{type, ...}}
Present the user with a standard ``open input file''
dialog. Optionally, you can pass up to four 4-character file types to limit
the files the user can choose from. The function returns an \pytype{FSSpec}
object and a flag indicating that the user completed the dialog
without cancelling.
\end{funcdesc}

\begin{funcdesc}{PromptGetFile}{prompt\optional{, type, ...}}
Similar to \function{StandardGetFile()} but allows you to specify a
prompt.
\end{funcdesc}

\begin{funcdesc}{StandardPutFile}{prompt, \optional{default}}
Present the user with a standard ``open output file''
dialog. \var{prompt} is the prompt string, and the optional
\var{default} argument initializes the output file name. The function
returns an \pytype{FSSpec} object and a flag indicating that the user
completed the dialog without cancelling.
\end{funcdesc}

\begin{funcdesc}{GetDirectory}{\optional{prompt}}
Present the user with a non-standard ``select a directory''
dialog. \var{prompt} is the prompt string, and the optional.
Return an \pytype{FSSpec} object and a success-indicator.
\end{funcdesc}

\begin{funcdesc}{SetFolder}{\optional{fsspec}}
Set the folder that is initially presented to the user when one of
the file selection dialogs is presented. \var{fsspec} should point to
a file in the folder, not the folder itself (the file need not exist,
though). If no argument is passed the folder will be set to the
current directory, i.e. what \function{os.getcwd()} returns.

Note that starting with system 7.5 the user can change Standard File
behaviour with the ``general controls'' controlpanel, thereby making
this call inoperative.
\end{funcdesc}

\begin{funcdesc}{FindFolder}{where, which, create}
Locates one of the ``special'' folders that MacOS knows about, such as
the trash or the Preferences folder. \var{where} is the disk to
search, \var{which} is the 4-character string specifying which folder to
locate. Setting \var{create} causes the folder to be created if it
does not exist. Returns a \code{(\var{vrefnum}, \var{dirid})} tuple.
\end{funcdesc}

\begin{funcdesc}{NewAliasMinimalFromFullPath}{pathname}
Return a minimal \pytype{alias} object that points to the given file, which
must be specified as a full pathname. This is the only way to create an
\pytype{Alias} pointing to a non-existing file.

The constants for \var{where} and \var{which} can be obtained from the
standard module \var{MACFS}.
\end{funcdesc}

\begin{funcdesc}{FindApplication}{creator}
Locate the application with 4-char creator code \var{creator}. The
function returns an \pytype{FSSpec} object pointing to the application.
\end{funcdesc}


\subsection{FSSpec objects \label{fsspec-objects}}

\begin{memberdesc}[FSSpec]{data}
The raw data from the FSSpec object, suitable for passing
to other applications, for instance.
\end{memberdesc}

\begin{methoddesc}[FSSpec]{as_pathname}{}
Return the full pathname of the file described by the \pytype{FSSpec}
object.
\end{methoddesc}

\begin{methoddesc}[FSSpec]{as_tuple}{}
Return the \code{(\var{wdRefNum}, \var{parID}, \var{name})} tuple of
the file described by the \pytype{FSSpec} object.
\end{methoddesc}

\begin{methoddesc}[FSSpec]{NewAlias}{\optional{file}}
Create an Alias object pointing to the file described by this
FSSpec. If the optional \var{file} parameter is present the alias
will be relative to that file, otherwise it will be absolute.
\end{methoddesc}

\begin{methoddesc}[FSSpec]{NewAliasMinimal}{}
Create a minimal alias pointing to this file.
\end{methoddesc}

\begin{methoddesc}[FSSpec]{GetCreatorType}{}
Return the 4-character creator and type of the file.
\end{methoddesc}

\begin{methoddesc}[FSSpec]{SetCreatorType}{creator, type}
Set the 4-character creator and type of the file.
\end{methoddesc}

\begin{methoddesc}[FSSpec]{GetFInfo}{}
Return a \pytype{FInfo} object describing the finder info for the file.
\end{methoddesc}

\begin{methoddesc}[FSSpec]{SetFInfo}{finfo}
Set the finder info for the file to the values given as \var{finfo}
(an \pytype{FInfo} object).
\end{methoddesc}

\begin{methoddesc}[FSSpec]{GetDates}{}
Return a tuple with three floating point values representing the
creation date, modification date and backup date of the file.
\end{methoddesc}

\begin{methoddesc}[FSSpec]{SetDates}{crdate, moddate, backupdate}
Set the creation, modification and backup date of the file. The values
are in the standard floating point format used for times throughout
Python.
\end{methoddesc}


\subsection{Alias Objects \label{alias-objects}}

\begin{memberdesc}[Alias]{data}
The raw data for the Alias record, suitable for storing in a resource
or transmitting to other programs.
\end{memberdesc}

\begin{methoddesc}[Alias]{Resolve}{\optional{file}}
Resolve the alias. If the alias was created as a relative alias you
should pass the file relative to which it is. Return the FSSpec for
the file pointed to and a flag indicating whether the \pytype{Alias} object
itself was modified during the search process. If the file does
not exist but the path leading up to it does exist a valid fsspec
is returned.
\end{methoddesc}

\begin{methoddesc}[Alias]{GetInfo}{num}
An interface to the \C{} routine \cfunction{GetAliasInfo()}.
\end{methoddesc}

\begin{methoddesc}[Alias]{Update}{file, \optional{file2}}
Update the alias to point to the \var{file} given. If \var{file2} is
present a relative alias will be created.
\end{methoddesc}

Note that it is currently not possible to directly manipulate a
resource as an \pytype{Alias} object. Hence, after calling
\method{Update()} or after \method{Resolve()} indicates that the alias
has changed the Python program is responsible for getting the
\member{data} value from the \pytype{Alias} object and modifying the
resource.


\subsection{FInfo Objects \label{finfo-objects}}

See \citetitle{Inside Macintosh: Files} for a complete description of what
the various fields mean.

\begin{memberdesc}[FInfo]{Creator}
The 4-character creator code of the file.
\end{memberdesc}

\begin{memberdesc}[FInfo]{Type}
The 4-character type code of the file.
\end{memberdesc}

\begin{memberdesc}[FInfo]{Flags}
The finder flags for the file as 16-bit integer. The bit values in
\var{Flags} are defined in standard module \module{MACFS}.
\end{memberdesc}

\begin{memberdesc}[FInfo]{Location}
A Point giving the position of the file's icon in its folder.
\end{memberdesc}

\begin{memberdesc}[FInfo]{Fldr}
The folder the file is in (as an integer).
\end{memberdesc}

\section{\module{ic} ---
         Access to Internet Config}

\declaremodule{builtin}{ic}
  \platform{Mac}
\modulesynopsis{Access to Internet Config.}


This module provides access to Macintosh Internet
Config\index{Internet Config} package,
which stores preferences for Internet programs such as mail address,
default homepage, etc. Also, Internet Config contains an elaborate set
of mappings from Macintosh creator/type codes to foreign filename
extensions plus information on how to transfer files (binary, ascii,
etc).

There is a low-level companion module
\module{icglue}\refbimodindex{icglue} which provides the basic
Internet Config access functionality.  This low-level module is not
documented, but the docstrings of the routines document the parameters
and the routine names are the same as for the Pascal or \C{} API to
Internet Config, so the standard IC programmers' documentation can be
used if this module is needed.

The \module{ic} module defines the \exception{error} exception and
symbolic names for all error codes Internet Config can produce; see
the source for details.

\begin{excdesc}{error}
Exception raised on errors in the \module{ic} module.
\end{excdesc}


The \module{ic} module defines the following class and function:

\begin{classdesc}{IC}{\optional{signature\optional{, ic}}}
Create an internet config object. The signature is a 4-character creator
code of the current application (default \code{'Pyth'}) which may
influence some of ICs settings. The optional \var{ic} argument is a
low-level \code{icglue.icinstance} created beforehand, this may be
useful if you want to get preferences from a different config file,
etc.
\end{classdesc}

\begin{funcdesc}{launchurl}{url\optional{, hint}}
\funcline{parseurl}{data\optional{, start\optional{, end\optional{, hint}}}}
\funcline{mapfile}{file}
\funcline{maptypecreator}{type, creator\optional{, filename}}
\funcline{settypecreator}{file}
These functions are ``shortcuts'' to the methods of the same name,
described below.
\end{funcdesc}


\subsection{IC Objects}

\class{IC} objects have a mapping interface, hence to obtain the mail
address you simply get \code{\var{ic}['MailAddress']}. Assignment also
works, and changes the option in the configuration file.

The module knows about various datatypes, and converts the internal IC
representation to a ``logical'' Python data structure. Running the
\module{ic} module standalone will run a test program that lists all
keys and values in your IC database, this will have to server as
documentation.

If the module does not know how to represent the data it returns an
instance of the \code{ICOpaqueData} type, with the raw data in its
\member{data} attribute. Objects of this type are also acceptable values
for assignment.

Besides the dictionary interface, \class{IC} objects have the
following methods:


\begin{methoddesc}{launchurl}{url\optional{, hint}}
Parse the given URL, lauch the correct application and pass it the
URL. The optional \var{hint} can be a scheme name such as
\code{'mailto:'}, in which case incomplete URLs are completed with this
scheme.  If \var{hint} is not provided, incomplete URLs are invalid.
\end{methoddesc}

\begin{methoddesc}{parseurl}{data\optional{, start\optional{, end\optional{, hint}}}}
Find an URL somewhere in \var{data} and return start position, end
position and the URL. The optional \var{start} and \var{end} can be
used to limit the search, so for instance if a user clicks in a long
textfield you can pass the whole textfield and the click-position in
\var{start} and this routine will return the whole URL in which the
user clicked.  As above, \var{hint} is an optional scheme used to
complete incomplete URLs.
\end{methoddesc}

\begin{methoddesc}{mapfile}{file}
Return the mapping entry for the given \var{file}, which can be passed
as either a filename or an \function{macfs.FSSpec()} result, and which
need not exist.

The mapping entry is returned as a tuple \code{(\var{version},
\var{type}, \var{creator}, \var{postcreator}, \var{flags},
\var{extension}, \var{appname}, \var{postappname}, \var{mimetype},
\var{entryname})}, where \var{version} is the entry version
number, \var{type} is the 4-character filetype, \var{creator} is the
4-character creator type, \var{postcreator} is the 4-character creator
code of an
optional application to post-process the file after downloading,
\var{flags} are various bits specifying whether to transfer in binary
or ascii and such, \var{extension} is the filename extension for this
file type, \var{appname} is the printable name of the application to
which this file belongs, \var{postappname} is the name of the
postprocessing application, \var{mimetype} is the MIME type of this
file and \var{entryname} is the name of this entry.
\end{methoddesc}

\begin{methoddesc}{maptypecreator}{type, creator\optional{, filename}}
Return the mapping entry for files with given 4-character \var{type} and
\var{creator} codes. The optional \var{filename} may be specified to
further help finding the correct entry (if the creator code is
\code{'????'}, for instance).

The mapping entry is returned in the same format as for \var{mapfile}.
\end{methoddesc}

\begin{methoddesc}{settypecreator}{file}
Given an existing \var{file}, specified either as a filename or as an
\function{macfs.FSSpec()} result, set its creator and type correctly based
on its extension.  The finder is told about the change, so the finder
icon will be updated quickly.
\end{methoddesc}

\section{\module{MacOS} ---
         Access to MacOS interpreter features}

\declaremodule{builtin}{MacOS}
  \platform{Mac}
\modulesynopsis{Access to MacOS specific interpreter features.}


This module provides access to MacOS specific functionality in the
Python interpreter, such as how the interpreter eventloop functions
and the like. Use with care.

Note the capitalisation of the module name, this is a historical
artifact.

\begin{excdesc}{Error}
This exception is raised on MacOS generated errors, either from
functions in this module or from other mac-specific modules like the
toolbox interfaces. The arguments are the integer error code (the
\cdata{OSErr} value) and a textual description of the error code.
Symbolic names for all known error codes are defined in the standard
module \module{macerrors}\refstmodindex{macerrors}.
\end{excdesc}

\begin{funcdesc}{SetEventHandler}{handler}
In the inner interpreter loop Python will occasionally check for events,
unless disabled with \function{ScheduleParams()}. With this function you
can pass a Python event-handler function that will be called if an event
is available. The event is passed as parameter and the function should return
non-zero if the event has been fully processed, otherwise event processing
continues (by passing the event to the console window package, for instance).

Call \function{SetEventHandler()} without a parameter to clear the
event handler. Setting an event handler while one is already set is an
error.
\end{funcdesc}

\begin{funcdesc}{SchedParams}{\optional{doint\optional{, evtmask\optional{,
                              besocial\optional{, interval\optional{,
                              bgyield}}}}}}
Influence the interpreter inner loop event handling. \var{Interval}
specifies how often (in seconds, floating point) the interpreter
should enter the event processing code. When true, \var{doint} causes
interrupt (command-dot) checking to be done. \var{evtmask} tells the
interpreter to do event processing for events in the mask (redraws,
mouseclicks to switch to other applications, etc). The \var{besocial}
flag gives other processes a chance to run. They are granted minimal
runtime when Python is in the foreground and \var{bgyield} seconds per
\var{interval} when Python runs in the background.

All parameters are optional, and default to the current value. The return
value of this function is a tuple with the old values of these options.
Initial defaults are that all processing is enabled, checking is done every
quarter second and the CPU is given up for a quarter second when in the
background.
\end{funcdesc}

\begin{funcdesc}{HandleEvent}{ev}
Pass the event record \var{ev} back to the Python event loop, or
possibly to the handler for the \code{sys.stdout} window (based on the
compiler used to build Python). This allows Python programs that do
their own event handling to still have some command-period and
window-switching capability.

If you attempt to call this function from an event handler set through
\function{SetEventHandler()} you will get an exception.
\end{funcdesc}

\begin{funcdesc}{GetErrorString}{errno}
Return the textual description of MacOS error code \var{errno}.
\end{funcdesc}

\begin{funcdesc}{splash}{resid}
This function will put a splash window
on-screen, with the contents of the DLOG resource specified by
\var{resid}. Calling with a zero argument will remove the splash
screen. This function is useful if you want an applet to post a splash screen
early in initialization without first having to load numerous
extension modules.
\end{funcdesc}

\begin{funcdesc}{DebugStr}{message \optional{, object}}
Drop to the low-level debugger with message \var{message}. The
optional \var{object} argument is not used, but can easily be
inspected from the debugger.

Note that you should use this function with extreme care: if no
low-level debugger like MacsBug is installed this call will crash your
system. It is intended mainly for developers of Python extension
modules.
\end{funcdesc}

\begin{funcdesc}{openrf}{name \optional{, mode}}
Open the resource fork of a file. Arguments are the same as for the
built-in function \function{open()}. The object returned has file-like
semantics, but it is not a Python file object, so there may be subtle
differences.
\end{funcdesc}

\section{\module{macostools} ---
         Convenience routines for file manipulation}

\declaremodule{standard}{macostools}
  \platform{Mac}
\modulesynopsis{Convenience routines for file manipulation.}


This module contains some convenience routines for file-manipulation
on the Macintosh.

The \module{macostools} module defines the following functions:


\begin{funcdesc}{copy}{src, dst\optional{, createpath\optional{, copytimes}}}
Copy file \var{src} to \var{dst}. The files can be specified as
pathnames or \pytype{FSSpec} objects. If \var{createpath} is non-zero
\var{dst} must be a pathname and the folders leading to the
destination are created if necessary.  The method copies data and
resource fork and some finder information (creator, type, flags) and
optionally the creation, modification and backup times (default is to
copy them). Custom icons, comments and icon position are not copied.

If the source is an alias the original to which the alias points is
copied, not the aliasfile.
\end{funcdesc}

\begin{funcdesc}{copytree}{src, dst}
Recursively copy a file tree from \var{src} to \var{dst}, creating
folders as needed. \var{src} and \var{dst} should be specified as
pathnames.
\end{funcdesc}

\begin{funcdesc}{mkalias}{src, dst}
Create a finder alias \var{dst} pointing to \var{src}. Both may be
specified as pathnames or \pytype{FSSpec} objects.
\end{funcdesc}

\begin{funcdesc}{touched}{dst}
Tell the finder that some bits of finder-information such as creator
or type for file \var{dst} has changed. The file can be specified by
pathname or fsspec. This call should prod the finder into redrawing the
files icon.
\end{funcdesc}

\begin{datadesc}{BUFSIZ}
The buffer size for \code{copy}, default 1 megabyte.
\end{datadesc}

Note that the process of creating finder aliases is not specified in
the Apple documentation. Hence, aliases created with \function{mkalias()}
could conceivably have incompatible behaviour in some cases.


\section{\module{findertools} ---
         The \program{finder}'s Apple Events interface}

\declaremodule{standard}{findertools}
  \platform{Mac}
\modulesynopsis{Wrappers around the \program{finder}'s Apple Events interface.}


This module contains routines that give Python programs access to some
functionality provided by the finder. They are implemented as wrappers
around the AppleEvent\index{AppleEvents} interface to the finder.

All file and folder parameters can be specified either as full
pathnames or as \pytype{FSSpec} objects.

The \module{findertools} module defines the following functions:


\begin{funcdesc}{launch}{file}
Tell the finder to launch \var{file}. What launching means depends on the file:
applications are started, folders are opened and documents are opened
in the correct application.
\end{funcdesc}

\begin{funcdesc}{Print}{file}
Tell the finder to print a file (again specified by full pathname or
\pytype{FSSpec}). The behaviour is identical to selecting the file and using
the print command in the finder.
\end{funcdesc}

\begin{funcdesc}{copy}{file, destdir}
Tell the finder to copy a file or folder \var{file} to folder
\var{destdir}. The function returns an \pytype{Alias} object pointing to
the new file.
\end{funcdesc}

\begin{funcdesc}{move}{file, destdir}
Tell the finder to move a file or folder \var{file} to folder
\var{destdir}. The function returns an \pytype{Alias} object pointing to
the new file.
\end{funcdesc}

\begin{funcdesc}{sleep}{}
Tell the finder to put the Macintosh to sleep, if your machine
supports it.
\end{funcdesc}

\begin{funcdesc}{restart}{}
Tell the finder to perform an orderly restart of the machine.
\end{funcdesc}

\begin{funcdesc}{shutdown}{}
Tell the finder to perform an orderly shutdown of the machine.
\end{funcdesc}

\section{\module{mactcp} ---
         The MacTCP interfaces}

\declaremodule{builtin}{mactcp}
  \platform{Mac}
\modulesynopsis{The MacTCP interfaces.}


This module provides an interface to the Macintosh TCP/IP driver%
\index{MacTCP} MacTCP. There is an accompanying module,
\refmodule{macdnr}\refbimodindex{macdnr}, which provides an interface
to the name-server (allowing you to translate hostnames to IP
addresses), a module \module{MACTCPconst}\refstmodindex{MACTCPconst}
which has symbolic names for constants constants used by MacTCP. Since
the built-in module \module{socket}\refbimodindex{socket} is also
available on the Macintosh it is usually easier to use sockets instead
of the Macintosh-specific MacTCP API.

A complete description of the MacTCP interface can be found in the
Apple MacTCP API documentation.

\begin{funcdesc}{MTU}{}
Return the Maximum Transmit Unit (the packet size) of the network
interface.\index{Maximum Transmit Unit}
\end{funcdesc}

\begin{funcdesc}{IPAddr}{}
Return the 32-bit integer IP address of the network interface.
\end{funcdesc}

\begin{funcdesc}{NetMask}{}
Return the 32-bit integer network mask of the interface.
\end{funcdesc}

\begin{funcdesc}{TCPCreate}{size}
Create a TCP Stream object. \var{size} is the size of the receive
buffer, \code{4096} is suggested by various sources.
\end{funcdesc}

\begin{funcdesc}{UDPCreate}{size, port}
Create a UDP Stream object. \var{size} is the size of the receive
buffer (and, hence, the size of the biggest datagram you can receive
on this port). \var{port} is the UDP port number you want to receive
datagrams on, a value of zero will make MacTCP select a free port.
\end{funcdesc}


\subsection{TCP Stream Objects}

\begin{memberdesc}[TCP Stream]{asr}
\index{asynchronous service routine}
\index{service routine, asynchronous}
When set to a value different than \code{None} this should refer to a
function with two integer parameters:\ an event code and a detail. This
function will be called upon network-generated events such as urgent
data arrival.  Macintosh documentation calls this the
\dfn{asynchronous service routine}.  In addition, it is called with
eventcode \code{MACTCP.PassiveOpenDone} when a \method{PassiveOpen()}
completes. This is a Python addition to the MacTCP semantics.
It is safe to do further calls from \var{asr}.
\end{memberdesc}


\begin{methoddesc}[TCP Stream]{PassiveOpen}{port}
Wait for an incoming connection on TCP port \var{port} (zero makes the
system pick a free port). The call returns immediately, and you should
use \method{wait()} to wait for completion. You should not issue any method
calls other than \method{wait()}, \method{isdone()} or
\method{GetSockName()} before the call completes.
\end{methoddesc}

\begin{methoddesc}[TCP Stream]{wait}{}
Wait for \method{PassiveOpen()} to complete.
\end{methoddesc}

\begin{methoddesc}[TCP Stream]{isdone}{}
Return \code{1} if a \method{PassiveOpen()} has completed.
\end{methoddesc}

\begin{methoddesc}[TCP Stream]{GetSockName}{}
Return the TCP address of this side of a connection as a 2-tuple
\code{(\var{host}, \var{port})}, both integers.
\end{methoddesc}

\begin{methoddesc}[TCP Stream]{ActiveOpen}{lport, host, rport}
Open an outgoing connection to TCP address \code{(\var{host},
\var{rport})}. Use
local port \var{lport} (zero makes the system pick a free port). This
call blocks until the connection has been established.
\end{methoddesc}

\begin{methoddesc}[TCP Stream]{Send}{buf, push, urgent}
Send data \var{buf} over the connection. \var{push} and \var{urgent}
are flags as specified by the TCP standard.
\end{methoddesc}

\begin{methoddesc}[TCP Stream]{Rcv}{timeout}
Receive data. The call returns when \var{timeout} seconds have passed
or when (according to the MacTCP documentation) ``a reasonable amount
of data has been received''. The return value is a 3-tuple
\code{(\var{data}, \var{urgent}, \var{mark})}. If urgent data is
outstanding \code{Rcv} will always return that before looking at any
normal data. The first call returning urgent data will have the
\var{urgent} flag set, the last will have the \var{mark} flag set.
\end{methoddesc}

\begin{methoddesc}[TCP Stream]{Close}{}
Tell MacTCP that no more data will be transmitted on this
connection. The call returns when all data has been acknowledged by
the receiving side.
\end{methoddesc}

\begin{methoddesc}[TCP Stream]{Abort}{}
Forcibly close both sides of a connection, ignoring outstanding data.
\end{methoddesc}

\begin{methoddesc}[TCP Stream]{Status}{}
Return a TCP status object for this stream giving the current status
(see below).
\end{methoddesc}


\subsection{TCP Status Objects}

This object has no methods, only some members holding information on
the connection. A complete description of all fields in this objects
can be found in the Apple documentation. The most interesting ones are:

\begin{memberdesc}[TCP Status]{localHost}
\memberline{localPort}
\memberline{remoteHost}
\memberline{remotePort}
The integer IP-addresses and port numbers of both endpoints of the
connection. 
\end{memberdesc}

\begin{memberdesc}[TCP Status]{sendWindow}
The current window size.
\end{memberdesc}

\begin{memberdesc}[TCP Status]{amtUnackedData}
The number of bytes sent but not yet acknowledged. \code{sendWindow -
amtUnackedData} is what you can pass to \method{Send()} without
blocking.
\end{memberdesc}

\begin{memberdesc}[TCP Status]{amtUnreadData}
The number of bytes received but not yet read (what you can
\method{Recv()} without blocking).
\end{memberdesc}



\subsection{UDP Stream Objects}

Note that, unlike the name suggests, there is nothing stream-like
about UDP.


\begin{memberdesc}[UDP Stream]{asr}
\index{asynchronous service routine}
\index{service routine, asynchronous}
The asynchronous service routine to be called on events such as
datagram arrival without outstanding \code{Read} call. The \var{asr}
has a single argument, the event code.
\end{memberdesc}

\begin{memberdesc}[UDP Stream]{port}
A read-only member giving the port number of this UDP Stream.
\end{memberdesc}


\begin{methoddesc}[UDP Stream]{Read}{timeout}
Read a datagram, waiting at most \var{timeout} seconds (-1 is
infinite).  Return the data.
\end{methoddesc}

\begin{methoddesc}[UDP Stream]{Write}{host, port, buf}
Send \var{buf} as a datagram to IP-address \var{host}, port
\var{port}.
\end{methoddesc}

\section{\module{macspeech} ---
         Interface to the Macintosh Speech Manager}

\declaremodule{builtin}{macspeech}
  \platform{Mac}
\modulesynopsis{Interface to the Macintosh Speech Manager.}


This module provides an interface to the Macintosh Speech Manager,
\index{Macintosh Speech Manager}
\index{Speech Manager, Macintosh}
allowing you to let the Macintosh utter phrases. You need a version of
the Speech Manager extension (version 1 and 2 have been tested) in
your \file{Extensions} folder for this to work. The module does not
provide full access to all features of the Speech Manager yet.  It may
not be available in all Mac Python versions.

\begin{funcdesc}{Available}{}
Test availability of the Speech Manager extension (and, on the
PowerPC, the Speech Manager shared library). Return \code{0} or
\code{1}.
\end{funcdesc}

\begin{funcdesc}{Version}{}
Return the (integer) version number of the Speech Manager.
\end{funcdesc}

\begin{funcdesc}{SpeakString}{str}
Utter the string \var{str} using the default voice,
asynchronously. This aborts any speech that may still be active from
prior \function{SpeakString()} invocations.
\end{funcdesc}

\begin{funcdesc}{Busy}{}
Return the number of speech channels busy, system-wide.
\end{funcdesc}

\begin{funcdesc}{CountVoices}{}
Return the number of different voices available.
\end{funcdesc}

\begin{funcdesc}{GetIndVoice}{num}
Return a \pytype{Voice} object for voice number \var{num}.
\end{funcdesc}

\subsection{Voice Objects}
\label{voice-objects}

Voice objects contain the description of a voice. It is currently not
yet possible to access the parameters of a voice.

\setindexsubitem{(voice object method)}

\begin{methoddesc}[Voice]{GetGender}{}
Return the gender of the voice: \code{0} for male, \code{1} for female
and \code{-1} for neuter.
\end{methoddesc}

\begin{methoddesc}[Voice]{NewChannel}{}
Return a new Speech Channel object using this voice.
\end{methoddesc}

\subsection{Speech Channel Objects}
\label{speech-channel-objects}

A Speech Channel object allows you to speak strings with slightly more
control than \function{SpeakString()}, and allows you to use multiple
speakers at the same time. Please note that channel pitch and rate are
interrelated in some way, so that to make your Macintosh sing you will
have to adjust both.

\begin{methoddesc}[Speech Channel]{SpeakText}{str}
Start uttering the given string.
\end{methoddesc}

\begin{methoddesc}[Speech Channel]{Stop}{}
Stop babbling.
\end{methoddesc}

\begin{methoddesc}[Speech Channel]{GetPitch}{}
Return the current pitch of the channel, as a floating-point number.
\end{methoddesc}

\begin{methoddesc}[Speech Channel]{SetPitch}{pitch}
Set the pitch of the channel.
\end{methoddesc}

\begin{methoddesc}[Speech Channel]{GetRate}{}
Get the speech rate (utterances per minute) of the channel as a
floating point number.
\end{methoddesc}

\begin{methoddesc}[Speech Channel]{SetRate}{rate}
Set the speech rate of the channel.
\end{methoddesc}


\section{\module{EasyDialogs} ---
         Basic Macintosh dialogs}

\declaremodule{standard}{EasyDialogs}
  \platform{Mac}
\modulesynopsis{Basic Macintosh dialogs.}


The \module{EasyDialogs} module contains some simple dialogs for
the Macintosh, modelled after the
\module{stdwin}\refbimodindex{stdwin} dialogs with similar names. All
routines have an optional parameter \var{id} with which you can
override the DLOG resource used for the dialog, as long as the item
numbers correspond. See the source for details.
 
The \module{EasyDialogs} module defines the following functions:


\begin{funcdesc}{Message}{str}
A modal dialog with the message text \var{str}, which should be at
most 255 characters long, is displayed. Control is returned when the
user clicks ``OK''.
\end{funcdesc}

\begin{funcdesc}{AskString}{prompt\optional{, default}}
Ask the user to input a string value, in a modal dialog. \var{prompt}
is the promt message, the optional \var{default} arg is the initial
value for the string. All strings can be at most 255 bytes
long. \function{AskString()} returns the string entered or \code{None}
in case the user cancelled.
\end{funcdesc}

\begin{funcdesc}{AskYesNoCancel}{question\optional{, default}}
Present a dialog with text \var{question} and three buttons labelled
``yes'', ``no'' and ``cancel''. Return \code{1} for yes, \code{0} for
no and \code{-1} for cancel. The default return value chosen by
hitting return is \code{0}. This can be changed with the optional
\var{default} argument.
\end{funcdesc}

\begin{funcdesc}{ProgressBar}{\optional{label\optional{, maxval}}}
Display a modeless progress dialog with a thermometer bar. \var{label}
is the text string displayed (default ``Working...''), \var{maxval} is
the value at which progress is complete (default \code{100}). The
returned object has one method, \code{set(\var{value})}, which sets
the value of the progress bar. The bar remains visible until the
object returned is discarded.

The progress bar has a ``cancel'' button, but it is currently
non-functional.
\end{funcdesc}

Note that \module{EasyDialogs} does not currently use the notification
manager. This means that displaying dialogs while the program is in
the background will lead to unexpected results and possibly
crashes. Also, all dialogs are modeless and hence expect to be at the
top of the stacking order. This is true when the dialogs are created,
but windows that pop-up later (like a console window) may also result
in crashes.

\section{\module{FrameWork} ---
         Interactive application framework}

\declaremodule{standard}{FrameWork}
  \platform{Mac}
\modulesynopsis{Interactive application framework.}


The \module{FrameWork} module contains classes that together provide a
framework for an interactive Macintosh application. The programmer
builds an application by creating subclasses that override various
methods of the bases classes, thereby implementing the functionality
wanted. Overriding functionality can often be done on various
different levels, i.e. to handle clicks in a single dialog window in a
non-standard way it is not necessary to override the complete event
handling.

The \module{FrameWork} is still very much work-in-progress, and the
documentation describes only the most important functionality, and not
in the most logical manner at that. Examine the source or the examples
for more details.

The \module{FrameWork} module defines the following functions:


\begin{funcdesc}{Application}{}
An object representing the complete application. See below for a
description of the methods. The default \method{__init__()} routine
creates an empty window dictionary and a menu bar with an apple menu.
\end{funcdesc}

\begin{funcdesc}{MenuBar}{}
An object representing the menubar. This object is usually not created
by the user.
\end{funcdesc}

\begin{funcdesc}{Menu}{bar, title\optional{, after}}
An object representing a menu. Upon creation you pass the
\code{MenuBar} the menu appears in, the \var{title} string and a
position (1-based) \var{after} where the menu should appear (default:
at the end).
\end{funcdesc}

\begin{funcdesc}{MenuItem}{menu, title\optional{, shortcut, callback}}
Create a menu item object. The arguments are the menu to crate the
item it, the item title string and optionally the keyboard shortcut
and a callback routine. The callback is called with the arguments
menu-id, item number within menu (1-based), current front window and
the event record.

In stead of a callable object the callback can also be a string. In
this case menu selection causes the lookup of a method in the topmost
window and the application. The method name is the callback string
with \code{'domenu_'} prepended.

Calling the \code{MenuBar} \method{fixmenudimstate()} method sets the
correct dimming for all menu items based on the current front window.
\end{funcdesc}

\begin{funcdesc}{Separator}{menu}
Add a separator to the end of a menu.
\end{funcdesc}

\begin{funcdesc}{SubMenu}{menu, label}
Create a submenu named \var{label} under menu \var{menu}. The menu
object is returned.
\end{funcdesc}

\begin{funcdesc}{Window}{parent}
Creates a (modeless) window. \var{Parent} is the application object to
which the window belongs. The window is not displayed until later.
\end{funcdesc}

\begin{funcdesc}{DialogWindow}{parent}
Creates a modeless dialog window.
\end{funcdesc}

\begin{funcdesc}{windowbounds}{width, height}
Return a \code{(\var{left}, \var{top}, \var{right}, \var{bottom})}
tuple suitable for creation of a window of given width and height. The
window will be staggered with respect to previous windows, and an
attempt is made to keep the whole window on-screen. The window will
however always be exact the size given, so parts may be offscreen.
\end{funcdesc}

\begin{funcdesc}{setwatchcursor}{}
Set the mouse cursor to a watch.
\end{funcdesc}

\begin{funcdesc}{setarrowcursor}{}
Set the mouse cursor to an arrow.
\end{funcdesc}


\subsection{Application Objects \label{application-objects}}

Application objects have the following methods, among others:


\begin{methoddesc}[Application]{makeusermenus}{}
Override this method if you need menus in your application. Append the
menus to the attribute \member{menubar}.
\end{methoddesc}

\begin{methoddesc}[Application]{getabouttext}{}
Override this method to return a text string describing your
application.  Alternatively, override the \method{do_about()} method
for more elaborate ``about'' messages.
\end{methoddesc}

\begin{methoddesc}[Application]{mainloop}{\optional{mask\optional{, wait}}}
This routine is the main event loop, call it to set your application
rolling. \var{Mask} is the mask of events you want to handle,
\var{wait} is the number of ticks you want to leave to other
concurrent application (default 0, which is probably not a good
idea). While raising \var{self} to exit the mainloop is still
supported it is not recommended: call \code{self._quit()} instead.

The event loop is split into many small parts, each of which can be
overridden. The default methods take care of dispatching events to
windows and dialogs, handling drags and resizes, Apple Events, events
for non-FrameWork windows, etc.

In general, all event handlers should return \code{1} if the event is fully
handled and \code{0} otherwise (because the front window was not a FrameWork
window, for instance). This is needed so that update events and such
can be passed on to other windows like the Sioux console window.
Calling \function{MacOS.HandleEvent()} is not allowed within
\var{our_dispatch} or its callees, since this may result in an
infinite loop if the code is called through the Python inner-loop
event handler.
\end{methoddesc}

\begin{methoddesc}[Application]{asyncevents}{onoff}
Call this method with a nonzero parameter to enable
asynchronous event handling. This will tell the inner interpreter loop
to call the application event handler \var{async_dispatch} whenever events
are available. This will cause FrameWork window updates and the user
interface to remain working during long computations, but will slow the
interpreter down and may cause surprising results in non-reentrant code
(such as FrameWork itself). By default \var{async_dispatch} will immedeately
call \var{our_dispatch} but you may override this to handle only certain
events asynchronously. Events you do not handle will be passed to Sioux
and such.

The old on/off value is returned.
\end{methoddesc}

\begin{methoddesc}[Application]{_quit}{}
Terminate the running \method{mainloop()} call at the next convenient
moment.
\end{methoddesc}

\begin{methoddesc}[Application]{do_char}{c, event}
The user typed character \var{c}. The complete details of the event
can be found in the \var{event} structure. This method can also be
provided in a \code{Window} object, which overrides the
application-wide handler if the window is frontmost.
\end{methoddesc}

\begin{methoddesc}[Application]{do_dialogevent}{event}
Called early in the event loop to handle modeless dialog events. The
default method simply dispatches the event to the relevant dialog (not
through the the \code{DialogWindow} object involved). Override if you
need special handling of dialog events (keyboard shortcuts, etc).
\end{methoddesc}

\begin{methoddesc}[Application]{idle}{event}
Called by the main event loop when no events are available. The
null-event is passed (so you can look at mouse position, etc).
\end{methoddesc}


\subsection{Window Objects \label{window-objects}}

Window objects have the following methods, among others:

\setindexsubitem{(Window method)}

\begin{methoddesc}[Window]{open}{}
Override this method to open a window. Store the MacOS window-id in
\member{self.wid} and call the \method{do_postopen()} method to
register the window with the parent application.
\end{methoddesc}

\begin{methoddesc}[Window]{close}{}
Override this method to do any special processing on window
close. Call the \method{do_postclose()} method to cleanup the parent
state.
\end{methoddesc}

\begin{methoddesc}[Window]{do_postresize}{width, height, macoswindowid}
Called after the window is resized. Override if more needs to be done
than calling \code{InvalRect}.
\end{methoddesc}

\begin{methoddesc}[Window]{do_contentclick}{local, modifiers, event}
The user clicked in the content part of a window. The arguments are
the coordinates (window-relative), the key modifiers and the raw
event.
\end{methoddesc}

\begin{methoddesc}[Window]{do_update}{macoswindowid, event}
An update event for the window was received. Redraw the window.
\end{methoddesc}

\begin{methoddesc}{do_activate}{activate, event}
The window was activated (\code{\var{activate} == 1}) or deactivated
(\code{\var{activate} == 0}). Handle things like focus highlighting,
etc.
\end{methoddesc}


\subsection{ControlsWindow Object \label{controlswindow-object}}

ControlsWindow objects have the following methods besides those of
\code{Window} objects:


\begin{methoddesc}[ControlsWindow]{do_controlhit}{window, control,
                                                  pcode, event}
Part \var{pcode} of control \var{control} was hit by the
user. Tracking and such has already been taken care of.
\end{methoddesc}


\subsection{ScrolledWindow Object \label{scrolledwindow-object}}

ScrolledWindow objects are ControlsWindow objects with the following
extra methods:


\begin{methoddesc}[ScrolledWindow]{scrollbars}{\optional{wantx\optional{,
                                               wanty}}}
Create (or destroy) horizontal and vertical scrollbars. The arguments
specify which you want (default: both). The scrollbars always have
minimum \code{0} and maximum \code{32767}.
\end{methoddesc}

\begin{methoddesc}[ScrolledWindow]{getscrollbarvalues}{}
You must supply this method. It should return a tuple \code{(\var{x},
\var{y})} giving the current position of the scrollbars (between
\code{0} and \code{32767}). You can return \code{None} for either to
indicate the whole document is visible in that direction.
\end{methoddesc}

\begin{methoddesc}[ScrolledWindow]{updatescrollbars}{}
Call this method when the document has changed. It will call
\method{getscrollbarvalues()} and update the scrollbars.
\end{methoddesc}

\begin{methoddesc}[ScrolledWindow]{scrollbar_callback}{which, what, value}
Supplied by you and called after user interaction. \var{which} will
be \code{'x'} or \code{'y'}, \var{what} will be \code{'-'},
\code{'--'}, \code{'set'}, \code{'++'} or \code{'+'}. For
\code{'set'}, \var{value} will contain the new scrollbar position.
\end{methoddesc}

\begin{methoddesc}[ScrolledWindow]{scalebarvalues}{absmin, absmax,
                                                   curmin, curmax}
Auxiliary method to help you calculate values to return from
\method{getscrollbarvalues()}. You pass document minimum and maximum value
and topmost (leftmost) and bottommost (rightmost) visible values and
it returns the correct number or \code{None}.
\end{methoddesc}

\begin{methoddesc}[ScrolledWindow]{do_activate}{onoff, event}
Takes care of dimming/highlighting scrollbars when a window becomes
frontmost vv. If you override this method call this one at the end of
your method.
\end{methoddesc}

\begin{methoddesc}[ScrolledWindow]{do_postresize}{width, height, window}
Moves scrollbars to the correct position. Call this method initially
if you override it.
\end{methoddesc}

\begin{methoddesc}[ScrolledWindow]{do_controlhit}{window, control,
                                                  pcode, event}
Handles scrollbar interaction. If you override it call this method
first, a nonzero return value indicates the hit was in the scrollbars
and has been handled.
\end{methoddesc}


\subsection{DialogWindow Objects \label{dialogwindow-objects}}

DialogWindow objects have the following methods besides those of
\code{Window} objects:


\begin{methoddesc}[DialogWindow]{open}{resid}
Create the dialog window, from the DLOG resource with id
\var{resid}. The dialog object is stored in \member{self.wid}.
\end{methoddesc}

\begin{methoddesc}[DialogWindow]{do_itemhit}{item, event}
Item number \var{item} was hit. You are responsible for redrawing
toggle buttons, etc.
\end{methoddesc}

\section{\module{MiniAEFrame} ---
         Open Scripting Architecture server support}

\declaremodule{standard}{MiniAEFrame}
  \platform{Mac}
\modulesynopsis{Support to act as an Open Scripting Architecture (OSA) server
(``Apple Events'').}


The module \module{MiniAEFrame} provides a framework for an application
that can function as an Open Scripting Architecture
\index{Open Scripting Architecture}
(OSA) server, i.e. receive and process
AppleEvents\index{AppleEvents}. It can be used in conjunction with
\refmodule{FrameWork}\refstmodindex{FrameWork} or standalone.

This module is temporary, it will eventually be replaced by a module
that handles argument names better and possibly automates making your
application scriptable.

The \module{MiniAEFrame} module defines the following classes:


\begin{classdesc}{AEServer}{}
A class that handles AppleEvent dispatch. Your application should
subclass this class together with either
\class{MiniApplication} or
\class{FrameWork.Application}. Your \method{__init__()} method should
call the \method{__init__()} method for both classes.
\end{classdesc}

\begin{classdesc}{MiniApplication}{}
A class that is more or less compatible with
\class{FrameWork.Application} but with less functionality. Its
event loop supports the apple menu, command-dot and AppleEvents; other
events are passed on to the Python interpreter and/or Sioux.
Useful if your application wants to use \class{AEServer} but does not
provide its own windows, etc.
\end{classdesc}


\subsection{AEServer Objects \label{aeserver-objects}}

\begin{methoddesc}[AEServer]{installaehandler}{classe, type, callback}
Installs an AppleEvent handler. \var{classe} and \var{type} are the
four-character OSA Class and Type designators, \code{'****'} wildcards
are allowed. When a matching AppleEvent is received the parameters are
decoded and your callback is invoked.
\end{methoddesc}

\begin{methoddesc}[AEServer]{callback}{_object, **kwargs}
Your callback is called with the OSA Direct Object as first positional
parameter. The other parameters are passed as keyword arguments, with
the 4-character designator as name. Three extra keyword parameters are
passed: \code{_class} and \code{_type} are the Class and Type
designators and \code{_attributes} is a dictionary with the AppleEvent
attributes.

The return value of your method is packed with
\function{aetools.packevent()} and sent as reply.
\end{methoddesc}

Note that there are some serious problems with the current
design. AppleEvents which have non-identifier 4-character designators
for arguments are not implementable, and it is not possible to return
an error to the originator. This will be addressed in a future
release.


%
%  The ugly "%begin{latexonly}" pseudo-environments are really just to
%  keep LaTeX2HTML quiet during the \renewcommand{} macros; they're
%  not really valuable.
%

%begin{latexonly}
\renewcommand{\indexname}{Module Index}
%end{latexonly}
\input{modmac.ind}		% Module Index

%begin{latexonly}
\renewcommand{\indexname}{Index}
%end{latexonly}
\documentclass{howto}

\title{Macintosh Library Modules}

\author{Guido van Rossum \\
	Fred L. Drake, Jr., editor}
\authoraddress{
	BeOpen PythonLabs \\
	E-mail: \email{python-docs@python.org}
}

\date{September 5, 2000}			% XXX update before release!
\release{1.6}


\makeindex			% tell \index to actually write the
				% .idx file
\makemodindex			% ... and the module index as well.
%\ignorePlatformAnnotation{Mac}


\begin{document}

\maketitle

\ifhtml
\chapter*{Front Matter\label{front}}
\fi

Copyright \copyright{} 1995-2000 Corporation for National Research
Initiatives.  All rights reserved.

Copyright \copyright{} 1991-1995 Stichting Mathematisch Centrum.  All
rights reserved.

\centerline{\strong{CNRI OPEN SOURCE LICENSE AGREEMENT}}


IMPORTANT: PLEASE READ THE FOLLOWING AGREEMENT CAREFULLY.

BY CLICKING ON ``ACCEPT'' WHERE INDICATED BELOW, OR BY COPYING,
INSTALLING OR OTHERWISE USING PYTHON 1.6 SOFTWARE, YOU ARE DEEMED TO
HAVE AGREED TO THE TERMS AND CONDITIONS OF THIS LICENSE AGREEMENT.

\begin{enumerate}
\item
This LICENSE AGREEMENT is between the Corporation for National
Research Initiatives, having an office at 1895 Preston White Drive,
Reston, VA 20191 (``CNRI''), and the Individual or Organization
(``Licensee'') accessing and otherwise using Python 1.6 software in
source or binary form and its associated documentation, as released at
the www.python.org Internet site on September 5, 2000 (``Python 1.6'').

\item
Subject to the terms and conditions of this License Agreement, CNRI
hereby grants Licensee a nonexclusive, royalty-free, world-wide
license to reproduce, analyze, test, perform and/or display publicly,
prepare derivative works, distribute, and otherwise use Python 1.6
alone or in any derivative version, provided, however, that CNRI's
License Agreement and CNRI's notice of copyright, i.e., ``Copyright (c)
1995-2000 Corporation for National Research Initiatives; All Rights
Reserved'' are retained in Python 1.6 alone or in any derivative
version prepared by Licensee.

Alternately, in lieu of CNRI's License Agreement, Licensee may
substitute the following text (omitting the quotes): ``Python 1.6 is
made available subject to the terms and conditions in CNRI's License
Agreement.  This Agreement together with Python 1.6 may be located on
the Internet using the following unique, persistent identifier (known
as a handle): 1895.22/1012.  This Agreement may also be obtained from a
proxy server on the Internet using the following URL:
http://hdl.handle.net/1895.22/1012''.

\item
In the event Licensee prepares a derivative work that is based on
or incorporates Python 1.6 or any part thereof, and wants to make the
derivative work available to others as provided herein, then Licensee
hereby agrees to include in any such work a brief summary of the
changes made to Python 1.6.

\item
CNRI is making Python 1.6 available to Licensee on an ``AS IS''
basis.  CNRI MAKES NO REPRESENTATIONS OR WARRANTIES, EXPRESS OR
IMPLIED.  BY WAY OF EXAMPLE, BUT NOT LIMITATION, CNRI MAKES NO AND
DISCLAIMS ANY REPRESENTATION OR WARRANTY OF MERCHANTABILITY OR FITNESS
FOR ANY PARTICULAR PURPOSE OR THAT THE USE OF PYTHON 1.6 WILL NOT
INFRINGE ANY THIRD PARTY RIGHTS.

\item
CNRI SHALL NOT BE LIABLE TO LICENSEE OR ANY OTHER USERS OF PYTHON
1.6 FOR ANY INCIDENTAL, SPECIAL, OR CONSEQUENTIAL DAMAGES OR LOSS AS A
RESULT OF MODIFYING, DISTRIBUTING, OR OTHERWISE USING PYTHON 1.6, OR
ANY DERIVATIVE THEREOF, EVEN IF ADVISED OF THE POSSIBILITY THEREOF.

\item
This License Agreement will automatically terminate upon a material
breach of its terms and conditions.

\item
This License Agreement shall be governed by and interpreted in all
respects by the law of the State of Virginia, excluding conflict of
law provisions.  Nothing in this License Agreement shall be deemed to
create any relationship of agency, partnership, or joint venture
between CNRI and Licensee.  This License Agreement does not grant
permission to use CNRI trademarks or trade name in a trademark sense
to endorse or promote products or services of Licensee, or any third
party.

\item
By clicking on the ``ACCEPT'' button where indicated, or by copying,
installing or otherwise using Python 1.6, Licensee agrees to be bound
by the terms and conditions of this License Agreement.

\centerline{ACCEPT}
\end{enumerate}


\begin{abstract}

\noindent
This library reference manual documents Python's extensions for the
Macintosh.  It should be used in conjunction with the
\citetitle[../lib/lib.html]{Python Library Reference}, which documents
the standard library and built-in types.

This manual assumes basic knowledge about the Python language.  For an
informal introduction to Python, see the
\citetitle[../tut/tut.html]{Python Tutorial}; the
\citetitle[../ref/ref.html]{Python Reference Manual} remains the
highest authority on syntactic and semantic questions.  Finally, the
manual entitled \citetitle[../ext/ext.html]{Extending and Embedding
the Python Interpreter} describes how to add new extensions to Python
and how to embed it in other applications.

\end{abstract}

\tableofcontents

\section{Introduction}
\label{intro}

The modules in this manual are available on the Apple Macintosh only.

Aside from the modules described here there are also interfaces to
various MacOS toolboxes, which are currently not extensively
described. The toolboxes for which modules exist are:
\module{AE} (Apple Events),
\module{Cm} (Component Manager),
\module{Ctl} (Control Manager),
\module{Dlg} (Dialog Manager),
\module{Evt} (Event Manager),
\module{Fm} (Font Manager),
\module{List} (List Manager),
\module{Menu} (Moenu Manager),
\module{Qd} (QuickDraw),
\module{Qt} (QuickTime),
\module{Res} (Resource Manager and Handles),
\module{Scrap} (Scrap Manager),
\module{Snd} (Sound Manager),
\module{TE} (TextEdit),
\module{Waste} (non-Apple \program{TextEdit} replacement) and
\module{Win} (Window Manager).

If applicable the module will define a number of Python objects for
the various structures declared by the toolbox, and operations will be
implemented as methods of the object. Other operations will be
implemented as functions in the module. Not all operations possible in
\C{} will also be possible in Python (callbacks are often a problem), and
parameters will occasionally be different in Python (input and output
buffers, especially). All methods and functions have a \code{__doc__}
string describing their arguments and return values, and for
additional description you are referred to \citetitle{Inside
Macintosh} or similar works.

The following modules are documented here:

\localmoduletable


\section{\module{mac} ---
         Implementations for the \module{os} module}

\declaremodule{builtin}{mac}
  \platform{Mac}
\modulesynopsis{Implementations for the \module{os} module.}


This module implements the operating system dependent functionality
provided by the standard module \module{os}\refstmodindex{os}.  It is
best accessed through the \module{os} module.

The following functions are available in this module:
\function{chdir()},
\function{close()},
\function{dup()},
\function{fdopen()},
\function{getcwd()},
\function{lseek()},
\function{listdir()},
\function{mkdir()},
\function{open()},
\function{read()},
\function{rename()},
\function{rmdir()},
\function{stat()},
\function{sync()},
\function{unlink()},
\function{write()},
as well as the exception \exception{error}. Note that the times
returned by \function{stat()} are floating-point values, like all time
values in MacPython.

One additional function is available:

\begin{funcdesc}{xstat}{path}
  This function returns the same information as \function{stat()}, but
  with three additional values appended: the size of the resource fork
  of the file and its 4-character creator and type.
\end{funcdesc}


\section{\module{macpath} ---
         MacOS path manipulation functions}

\declaremodule{standard}{macpath}
% Could be labeled \platform{Mac}, but the module should work anywhere and
% is distributed with the standard library.
\modulesynopsis{MacOS path manipulation functions.}


This module is the Macintosh implementation of the \module{os.path}
module.  It is most portably accessed as
\module{os.path}\refstmodindex{os.path}.  Refer to the
\citetitle[../lib/lib.html]{Python Library Reference} for
documentation of \module{os.path}.

The following functions are available in this module:
\function{normcase()},
\function{normpath()},
\function{isabs()},
\function{join()},
\function{split()},
\function{isdir()},
\function{isfile()},
\function{walk()},
\function{exists()}.
For other functions available in \module{os.path} dummy counterparts
are available.
			% MACINTOSH ONLY
\section{\module{ctb} ---
         Interface to the Communications Tool Box}

\declaremodule{builtin}{ctb}
  \platform{Mac}
\modulesynopsis{Interfaces to the Communications Tool Box.  Only the Connection
Manager is supported.}


This module provides a partial interface to the Macintosh
Communications Toolbox. Currently, only Connection Manager tools are
supported.  It may not be available in all Mac Python versions.
\index{Communications Toolbox, Macintosh}
\index{Macintosh Communications Toolbox}
\index{Connection Manager}

\begin{datadesc}{error}
The exception raised on errors.
\end{datadesc}

\begin{datadesc}{cmData}
\dataline{cmCntl}
\dataline{cmAttn}
Flags for the \var{channel} argument of the \method{Read()} and
\method{Write()} methods.
\end{datadesc}

\begin{datadesc}{cmFlagsEOM}
End-of-message flag for \method{Read()} and \method{Write()}.
\end{datadesc}

\begin{datadesc}{choose*}
Values returned by \method{Choose()}.
\end{datadesc}

\begin{datadesc}{cmStatus*}
Bits in the status as returned by \method{Status()}.
\end{datadesc}

\begin{funcdesc}{available}{}
Return \code{1} if the Communication Toolbox is available, zero otherwise.
\end{funcdesc}

\begin{funcdesc}{CMNew}{name, sizes}
Create a connection object using the connection tool named
\var{name}. \var{sizes} is a 6-tuple given buffer sizes for data in,
data out, control in, control out, attention in and attention out.
Alternatively, passing \code{None} for \var{sizes} will result in
default buffer sizes.
\end{funcdesc}


\subsection{Connection Objects \label{connection-object}}

For all connection methods that take a \var{timeout} argument, a value
of \code{-1} is indefinite, meaning that the command runs to completion.

\begin{memberdesc}[connection]{callback}
If this member is set to a value other than \code{None} it should point
to a function accepting a single argument (the connection
object). This will make all connection object methods work
asynchronously, with the callback routine being called upon
completion.

\emph{Note:} for reasons beyond my understanding the callback routine
is currently never called. You are advised against using asynchronous
calls for the time being.
\end{memberdesc}


\begin{methoddesc}[connection]{Open}{timeout}
Open an outgoing connection, waiting at most \var{timeout} seconds for
the connection to be established.
\end{methoddesc}

\begin{methoddesc}[connection]{Listen}{timeout}
Wait for an incoming connection. Stop waiting after \var{timeout}
seconds. This call is only meaningful to some tools.
\end{methoddesc}

\begin{methoddesc}[connection]{accept}{yesno}
Accept (when \var{yesno} is non-zero) or reject an incoming call after
\method{Listen()} returned.
\end{methoddesc}

\begin{methoddesc}[connection]{Close}{timeout, now}
Close a connection. When \var{now} is zero, the close is orderly
(i.e.\ outstanding output is flushed, etc.)\ with a timeout of
\var{timeout} seconds. When \var{now} is non-zero the close is
immediate, discarding output.
\end{methoddesc}

\begin{methoddesc}[connection]{Read}{len, chan, timeout}
Read \var{len} bytes, or until \var{timeout} seconds have passed, from 
the channel \var{chan} (which is one of \constant{cmData},
\constant{cmCntl} or \constant{cmAttn}). Return a 2-tuple:\ the data
read and the end-of-message flag, \constant{cmFlagsEOM}.
\end{methoddesc}

\begin{methoddesc}[connection]{Write}{buf, chan, timeout, eom}
Write \var{buf} to channel \var{chan}, aborting after \var{timeout}
seconds. When \var{eom} has the value \constant{cmFlagsEOM}, an
end-of-message indicator will be written after the data (if this
concept has a meaning for this communication tool). The method returns
the number of bytes written.
\end{methoddesc}

\begin{methoddesc}[connection]{Status}{}
Return connection status as the 2-tuple \code{(\var{sizes},
\var{flags})}. \var{sizes} is a 6-tuple giving the actual buffer sizes used
(see \function{CMNew()}), \var{flags} is a set of bits describing the state
of the connection.
\end{methoddesc}

\begin{methoddesc}[connection]{GetConfig}{}
Return the configuration string of the communication tool. These
configuration strings are tool-dependent, but usually easily parsed
and modified.
\end{methoddesc}

\begin{methoddesc}[connection]{SetConfig}{str}
Set the configuration string for the tool. The strings are parsed
left-to-right, with later values taking precedence. This means
individual configuration parameters can be modified by simply appending
something like \code{'baud 4800'} to the end of the string returned by
\method{GetConfig()} and passing that to this method. The method returns
the number of characters actually parsed by the tool before it
encountered an error (or completed successfully).
\end{methoddesc}

\begin{methoddesc}[connection]{Choose}{}
Present the user with a dialog to choose a communication tool and
configure it. If there is an outstanding connection some choices (like
selecting a different tool) may cause the connection to be
aborted. The return value (one of the \constant{choose*} constants) will
indicate this.
\end{methoddesc}

\begin{methoddesc}[connection]{Idle}{}
Give the tool a chance to use the processor. You should call this
method regularly.
\end{methoddesc}

\begin{methoddesc}[connection]{Abort}{}
Abort an outstanding asynchronous \method{Open()} or \method{Listen()}.
\end{methoddesc}

\begin{methoddesc}[connection]{Reset}{}
Reset a connection. Exact meaning depends on the tool.
\end{methoddesc}

\begin{methoddesc}[connection]{Break}{length}
Send a break. Whether this means anything, what it means and
interpretation of the \var{length} parameter depends on the tool in
use.
\end{methoddesc}

\section{\module{macconsole} ---
         Think C's console package}

\declaremodule{builtin}{macconsole}
  \platform{Mac}
\modulesynopsis{Think C's console package.}


This module is available on the Macintosh, provided Python has been
built using the Think C compiler. It provides an interface to the
Think console package, with which basic text windows can be created.

\begin{datadesc}{options}
An object allowing you to set various options when creating windows,
see below.
\end{datadesc}

\begin{datadesc}{C_ECHO}
\dataline{C_NOECHO}
\dataline{C_CBREAK}
\dataline{C_RAW}
Options for the \code{setmode} method. \constant{C_ECHO} and
\constant{C_CBREAK} enable character echo, the other two disable it,
\constant{C_ECHO} and \constant{C_NOECHO} enable line-oriented input
(erase/kill processing, etc).
\end{datadesc}

\begin{funcdesc}{copen}{}
Open a new console window. Return a console window object.
\end{funcdesc}

\begin{funcdesc}{fopen}{fp}
Return the console window object corresponding with the given file
object. \var{fp} should be one of \code{sys.stdin}, \code{sys.stdout} or
\code{sys.stderr}.
\end{funcdesc}

\subsection{macconsole options object}
These options are examined when a window is created:

\setindexsubitem{(macconsole option)}
\begin{datadesc}{top}
\dataline{left}
The origin of the window.
\end{datadesc}

\begin{datadesc}{nrows}
\dataline{ncols}
The size of the window.
\end{datadesc}

\begin{datadesc}{txFont}
\dataline{txSize}
\dataline{txStyle}
The font, fontsize and fontstyle to be used in the window.
\end{datadesc}

\begin{datadesc}{title}
The title of the window.
\end{datadesc}

\begin{datadesc}{pause_atexit}
If set non-zero, the window will wait for user action before closing.
\end{datadesc}

\subsection{console window object}

\setindexsubitem{(console window attribute)}

\begin{datadesc}{file}
The file object corresponding to this console window. If the file is
buffered, you should call \code{\var{file}.flush()} between
\code{write()} and \code{read()} calls.
\end{datadesc}

\setindexsubitem{(console window method)}

\begin{funcdesc}{setmode}{mode}
Set the input mode of the console to \constant{C_ECHO}, etc.
\end{funcdesc}

\begin{funcdesc}{settabs}{n}
Set the tabsize to \var{n} spaces.
\end{funcdesc}

\begin{funcdesc}{cleos}{}
Clear to end-of-screen.
\end{funcdesc}

\begin{funcdesc}{cleol}{}
Clear to end-of-line.
\end{funcdesc}

\begin{funcdesc}{inverse}{onoff}
Enable inverse-video mode:\ characters with the high bit set are
displayed in inverse video (this disables the upper half of a
non-\ASCII{} character set).
\end{funcdesc}

\begin{funcdesc}{gotoxy}{x, y}
Set the cursor to position \code{(\var{x}, \var{y})}.
\end{funcdesc}

\begin{funcdesc}{hide}{}
Hide the window, remembering the contents.
\end{funcdesc}

\begin{funcdesc}{show}{}
Show the window again.
\end{funcdesc}

\begin{funcdesc}{echo2printer}{}
Copy everything written to the window to the printer as well.
\end{funcdesc}

\section{\module{macdnr} ---
         Interface to the Macintosh Domain Name Resolver}

\declaremodule{builtin}{macdnr}
  \platform{Mac}
\modulesynopsis{Interfaces to the Macintosh Domain Name Resolver.}


This module provides an interface to the Macintosh Domain Name
Resolver.  It is usually used in conjunction with the \refmodule{mactcp}
module, to map hostnames to IP addresses.  It may not be available in
all Mac Python versions.
\index{Macintosh Domain Name Resolver}
\index{Domain Name Resolver, Macintosh}

The \module{macdnr} module defines the following functions:


\begin{funcdesc}{Open}{\optional{filename}}
Open the domain name resolver extension.  If \var{filename} is given it
should be the pathname of the extension, otherwise a default is
used.  Normally, this call is not needed since the other calls will
open the extension automatically.
\end{funcdesc}

\begin{funcdesc}{Close}{}
Close the resolver extension.  Again, not needed for normal use.
\end{funcdesc}

\begin{funcdesc}{StrToAddr}{hostname}
Look up the IP address for \var{hostname}.  This call returns a dnr
result object of the ``address'' variation.
\end{funcdesc}

\begin{funcdesc}{AddrToName}{addr}
Do a reverse lookup on the 32-bit integer IP-address
\var{addr}.  Returns a dnr result object of the ``address'' variation.
\end{funcdesc}

\begin{funcdesc}{AddrToStr}{addr}
Convert the 32-bit integer IP-address \var{addr} to a dotted-decimal
string.  Returns the string.
\end{funcdesc}

\begin{funcdesc}{HInfo}{hostname}
Query the nameservers for a \code{HInfo} record for host
\var{hostname}.  These records contain hardware and software
information about the machine in question (if they are available in
the first place).  Returns a dnr result object of the ``hinfo''
variety.
\end{funcdesc}

\begin{funcdesc}{MXInfo}{domain}
Query the nameservers for a mail exchanger for \var{domain}.  This is
the hostname of a host willing to accept SMTP\index{SMTP} mail for the
given domain.  Returns a dnr result object of the ``mx'' variety.
\end{funcdesc}


\subsection{DNR Result Objects \label{dnr-result-object}}

Since the DNR calls all execute asynchronously you do not get the
results back immediately.  Instead, you get a dnr result object.  You
can check this object to see whether the query is complete, and access
its attributes to obtain the information when it is.

Alternatively, you can also reference the result attributes directly,
this will result in an implicit wait for the query to complete.

The \member{rtnCode} and \member{cname} attributes are always
available, the others depend on the type of query (address, hinfo or
mx).


% Add args, as in {arg1, arg2 \optional{, arg3}}
\begin{methoddesc}[dnr result]{wait}{}
Wait for the query to complete.
\end{methoddesc}

% Add args, as in {arg1, arg2 \optional{, arg3}}
\begin{methoddesc}[dnr result]{isdone}{}
Return \code{1} if the query is complete.
\end{methoddesc}


\begin{memberdesc}[dnr result]{rtnCode}
The error code returned by the query.
\end{memberdesc}

\begin{memberdesc}[dnr result]{cname}
The canonical name of the host that was queried.
\end{memberdesc}

\begin{memberdesc}[dnr result]{ip0}
\memberline{ip1}
\memberline{ip2}
\memberline{ip3}
At most four integer IP addresses for this host.  Unused entries are
zero.  Valid only for address queries.
\end{memberdesc}

\begin{memberdesc}[dnr result]{cpuType}
\memberline{osType}
Textual strings giving the machine type an OS name.  Valid for ``hinfo''
queries.
\end{memberdesc}

\begin{memberdesc}[dnr result]{exchange}
The name of a mail-exchanger host.  Valid for ``mx'' queries.
\end{memberdesc}

\begin{memberdesc}[dnr result]{preference}
The preference of this mx record.  Not too useful, since the Macintosh
will only return a single mx record.  Valid for ``mx'' queries only.
\end{memberdesc}

The simplest way to use the module to convert names to dotted-decimal
strings, without worrying about idle time, etc:

\begin{verbatim}
>>> def gethostname(name):
...     import macdnr
...     dnrr = macdnr.StrToAddr(name)
...     return macdnr.AddrToStr(dnrr.ip0)
\end{verbatim}

\section{\module{macfs} ---
         Various file system services}

\declaremodule{builtin}{macfs}
  \platform{Mac}
\modulesynopsis{Support for FSSpec, the Alias Manager,
                \program{finder} aliases, and the Standard File package.}


This module provides access to Macintosh FSSpec handling, the Alias
Manager, \program{finder} aliases and the Standard File package.
\index{Macintosh Alias Manager}
\index{Alias Manager, Macintosh}
\index{Standard File}

Whenever a function or method expects a \var{file} argument, this
argument can be one of three things:\ (1) a full or partial Macintosh
pathname, (2) an \pytype{FSSpec} object or (3) a 3-tuple \code{(\var{wdRefNum},
\var{parID}, \var{name})} as described in \citetitle{Inside
Macintosh:\ Files}. A description of aliases and the Standard File
package can also be found there.

\begin{funcdesc}{FSSpec}{file}
Create an \pytype{FSSpec} object for the specified file.
\end{funcdesc}

\begin{funcdesc}{RawFSSpec}{data}
Create an \pytype{FSSpec} object given the raw data for the \C{}
structure for the \pytype{FSSpec} as a string.  This is mainly useful
if you have obtained an \pytype{FSSpec} structure over a network.
\end{funcdesc}

\begin{funcdesc}{RawAlias}{data}
Create an \pytype{Alias} object given the raw data for the \C{}
structure for the alias as a string.  This is mainly useful if you
have obtained an \pytype{FSSpec} structure over a network.
\end{funcdesc}

\begin{funcdesc}{FInfo}{}
Create a zero-filled \pytype{FInfo} object.
\end{funcdesc}

\begin{funcdesc}{ResolveAliasFile}{file}
Resolve an alias file. Returns a 3-tuple \code{(\var{fsspec},
\var{isfolder}, \var{aliased})} where \var{fsspec} is the resulting
\pytype{FSSpec} object, \var{isfolder} is true if \var{fsspec} points
to a folder and \var{aliased} is true if the file was an alias in the
first place (otherwise the \pytype{FSSpec} object for the file itself
is returned).
\end{funcdesc}

\begin{funcdesc}{StandardGetFile}{\optional{type, ...}}
Present the user with a standard ``open input file''
dialog. Optionally, you can pass up to four 4-character file types to limit
the files the user can choose from. The function returns an \pytype{FSSpec}
object and a flag indicating that the user completed the dialog
without cancelling.
\end{funcdesc}

\begin{funcdesc}{PromptGetFile}{prompt\optional{, type, ...}}
Similar to \function{StandardGetFile()} but allows you to specify a
prompt.
\end{funcdesc}

\begin{funcdesc}{StandardPutFile}{prompt, \optional{default}}
Present the user with a standard ``open output file''
dialog. \var{prompt} is the prompt string, and the optional
\var{default} argument initializes the output file name. The function
returns an \pytype{FSSpec} object and a flag indicating that the user
completed the dialog without cancelling.
\end{funcdesc}

\begin{funcdesc}{GetDirectory}{\optional{prompt}}
Present the user with a non-standard ``select a directory''
dialog. \var{prompt} is the prompt string, and the optional.
Return an \pytype{FSSpec} object and a success-indicator.
\end{funcdesc}

\begin{funcdesc}{SetFolder}{\optional{fsspec}}
Set the folder that is initially presented to the user when one of
the file selection dialogs is presented. \var{fsspec} should point to
a file in the folder, not the folder itself (the file need not exist,
though). If no argument is passed the folder will be set to the
current directory, i.e. what \function{os.getcwd()} returns.

Note that starting with system 7.5 the user can change Standard File
behaviour with the ``general controls'' controlpanel, thereby making
this call inoperative.
\end{funcdesc}

\begin{funcdesc}{FindFolder}{where, which, create}
Locates one of the ``special'' folders that MacOS knows about, such as
the trash or the Preferences folder. \var{where} is the disk to
search, \var{which} is the 4-character string specifying which folder to
locate. Setting \var{create} causes the folder to be created if it
does not exist. Returns a \code{(\var{vrefnum}, \var{dirid})} tuple.
\end{funcdesc}

\begin{funcdesc}{NewAliasMinimalFromFullPath}{pathname}
Return a minimal \pytype{alias} object that points to the given file, which
must be specified as a full pathname. This is the only way to create an
\pytype{Alias} pointing to a non-existing file.

The constants for \var{where} and \var{which} can be obtained from the
standard module \var{MACFS}.
\end{funcdesc}

\begin{funcdesc}{FindApplication}{creator}
Locate the application with 4-char creator code \var{creator}. The
function returns an \pytype{FSSpec} object pointing to the application.
\end{funcdesc}


\subsection{FSSpec objects \label{fsspec-objects}}

\begin{memberdesc}[FSSpec]{data}
The raw data from the FSSpec object, suitable for passing
to other applications, for instance.
\end{memberdesc}

\begin{methoddesc}[FSSpec]{as_pathname}{}
Return the full pathname of the file described by the \pytype{FSSpec}
object.
\end{methoddesc}

\begin{methoddesc}[FSSpec]{as_tuple}{}
Return the \code{(\var{wdRefNum}, \var{parID}, \var{name})} tuple of
the file described by the \pytype{FSSpec} object.
\end{methoddesc}

\begin{methoddesc}[FSSpec]{NewAlias}{\optional{file}}
Create an Alias object pointing to the file described by this
FSSpec. If the optional \var{file} parameter is present the alias
will be relative to that file, otherwise it will be absolute.
\end{methoddesc}

\begin{methoddesc}[FSSpec]{NewAliasMinimal}{}
Create a minimal alias pointing to this file.
\end{methoddesc}

\begin{methoddesc}[FSSpec]{GetCreatorType}{}
Return the 4-character creator and type of the file.
\end{methoddesc}

\begin{methoddesc}[FSSpec]{SetCreatorType}{creator, type}
Set the 4-character creator and type of the file.
\end{methoddesc}

\begin{methoddesc}[FSSpec]{GetFInfo}{}
Return a \pytype{FInfo} object describing the finder info for the file.
\end{methoddesc}

\begin{methoddesc}[FSSpec]{SetFInfo}{finfo}
Set the finder info for the file to the values given as \var{finfo}
(an \pytype{FInfo} object).
\end{methoddesc}

\begin{methoddesc}[FSSpec]{GetDates}{}
Return a tuple with three floating point values representing the
creation date, modification date and backup date of the file.
\end{methoddesc}

\begin{methoddesc}[FSSpec]{SetDates}{crdate, moddate, backupdate}
Set the creation, modification and backup date of the file. The values
are in the standard floating point format used for times throughout
Python.
\end{methoddesc}


\subsection{Alias Objects \label{alias-objects}}

\begin{memberdesc}[Alias]{data}
The raw data for the Alias record, suitable for storing in a resource
or transmitting to other programs.
\end{memberdesc}

\begin{methoddesc}[Alias]{Resolve}{\optional{file}}
Resolve the alias. If the alias was created as a relative alias you
should pass the file relative to which it is. Return the FSSpec for
the file pointed to and a flag indicating whether the \pytype{Alias} object
itself was modified during the search process. If the file does
not exist but the path leading up to it does exist a valid fsspec
is returned.
\end{methoddesc}

\begin{methoddesc}[Alias]{GetInfo}{num}
An interface to the \C{} routine \cfunction{GetAliasInfo()}.
\end{methoddesc}

\begin{methoddesc}[Alias]{Update}{file, \optional{file2}}
Update the alias to point to the \var{file} given. If \var{file2} is
present a relative alias will be created.
\end{methoddesc}

Note that it is currently not possible to directly manipulate a
resource as an \pytype{Alias} object. Hence, after calling
\method{Update()} or after \method{Resolve()} indicates that the alias
has changed the Python program is responsible for getting the
\member{data} value from the \pytype{Alias} object and modifying the
resource.


\subsection{FInfo Objects \label{finfo-objects}}

See \citetitle{Inside Macintosh: Files} for a complete description of what
the various fields mean.

\begin{memberdesc}[FInfo]{Creator}
The 4-character creator code of the file.
\end{memberdesc}

\begin{memberdesc}[FInfo]{Type}
The 4-character type code of the file.
\end{memberdesc}

\begin{memberdesc}[FInfo]{Flags}
The finder flags for the file as 16-bit integer. The bit values in
\var{Flags} are defined in standard module \module{MACFS}.
\end{memberdesc}

\begin{memberdesc}[FInfo]{Location}
A Point giving the position of the file's icon in its folder.
\end{memberdesc}

\begin{memberdesc}[FInfo]{Fldr}
The folder the file is in (as an integer).
\end{memberdesc}

\section{\module{ic} ---
         Access to Internet Config}

\declaremodule{builtin}{ic}
  \platform{Mac}
\modulesynopsis{Access to Internet Config.}


This module provides access to Macintosh Internet
Config\index{Internet Config} package,
which stores preferences for Internet programs such as mail address,
default homepage, etc. Also, Internet Config contains an elaborate set
of mappings from Macintosh creator/type codes to foreign filename
extensions plus information on how to transfer files (binary, ascii,
etc).

There is a low-level companion module
\module{icglue}\refbimodindex{icglue} which provides the basic
Internet Config access functionality.  This low-level module is not
documented, but the docstrings of the routines document the parameters
and the routine names are the same as for the Pascal or \C{} API to
Internet Config, so the standard IC programmers' documentation can be
used if this module is needed.

The \module{ic} module defines the \exception{error} exception and
symbolic names for all error codes Internet Config can produce; see
the source for details.

\begin{excdesc}{error}
Exception raised on errors in the \module{ic} module.
\end{excdesc}


The \module{ic} module defines the following class and function:

\begin{classdesc}{IC}{\optional{signature\optional{, ic}}}
Create an internet config object. The signature is a 4-character creator
code of the current application (default \code{'Pyth'}) which may
influence some of ICs settings. The optional \var{ic} argument is a
low-level \code{icglue.icinstance} created beforehand, this may be
useful if you want to get preferences from a different config file,
etc.
\end{classdesc}

\begin{funcdesc}{launchurl}{url\optional{, hint}}
\funcline{parseurl}{data\optional{, start\optional{, end\optional{, hint}}}}
\funcline{mapfile}{file}
\funcline{maptypecreator}{type, creator\optional{, filename}}
\funcline{settypecreator}{file}
These functions are ``shortcuts'' to the methods of the same name,
described below.
\end{funcdesc}


\subsection{IC Objects}

\class{IC} objects have a mapping interface, hence to obtain the mail
address you simply get \code{\var{ic}['MailAddress']}. Assignment also
works, and changes the option in the configuration file.

The module knows about various datatypes, and converts the internal IC
representation to a ``logical'' Python data structure. Running the
\module{ic} module standalone will run a test program that lists all
keys and values in your IC database, this will have to server as
documentation.

If the module does not know how to represent the data it returns an
instance of the \code{ICOpaqueData} type, with the raw data in its
\member{data} attribute. Objects of this type are also acceptable values
for assignment.

Besides the dictionary interface, \class{IC} objects have the
following methods:


\begin{methoddesc}{launchurl}{url\optional{, hint}}
Parse the given URL, lauch the correct application and pass it the
URL. The optional \var{hint} can be a scheme name such as
\code{'mailto:'}, in which case incomplete URLs are completed with this
scheme.  If \var{hint} is not provided, incomplete URLs are invalid.
\end{methoddesc}

\begin{methoddesc}{parseurl}{data\optional{, start\optional{, end\optional{, hint}}}}
Find an URL somewhere in \var{data} and return start position, end
position and the URL. The optional \var{start} and \var{end} can be
used to limit the search, so for instance if a user clicks in a long
textfield you can pass the whole textfield and the click-position in
\var{start} and this routine will return the whole URL in which the
user clicked.  As above, \var{hint} is an optional scheme used to
complete incomplete URLs.
\end{methoddesc}

\begin{methoddesc}{mapfile}{file}
Return the mapping entry for the given \var{file}, which can be passed
as either a filename or an \function{macfs.FSSpec()} result, and which
need not exist.

The mapping entry is returned as a tuple \code{(\var{version},
\var{type}, \var{creator}, \var{postcreator}, \var{flags},
\var{extension}, \var{appname}, \var{postappname}, \var{mimetype},
\var{entryname})}, where \var{version} is the entry version
number, \var{type} is the 4-character filetype, \var{creator} is the
4-character creator type, \var{postcreator} is the 4-character creator
code of an
optional application to post-process the file after downloading,
\var{flags} are various bits specifying whether to transfer in binary
or ascii and such, \var{extension} is the filename extension for this
file type, \var{appname} is the printable name of the application to
which this file belongs, \var{postappname} is the name of the
postprocessing application, \var{mimetype} is the MIME type of this
file and \var{entryname} is the name of this entry.
\end{methoddesc}

\begin{methoddesc}{maptypecreator}{type, creator\optional{, filename}}
Return the mapping entry for files with given 4-character \var{type} and
\var{creator} codes. The optional \var{filename} may be specified to
further help finding the correct entry (if the creator code is
\code{'????'}, for instance).

The mapping entry is returned in the same format as for \var{mapfile}.
\end{methoddesc}

\begin{methoddesc}{settypecreator}{file}
Given an existing \var{file}, specified either as a filename or as an
\function{macfs.FSSpec()} result, set its creator and type correctly based
on its extension.  The finder is told about the change, so the finder
icon will be updated quickly.
\end{methoddesc}

\section{\module{MacOS} ---
         Access to MacOS interpreter features}

\declaremodule{builtin}{MacOS}
  \platform{Mac}
\modulesynopsis{Access to MacOS specific interpreter features.}


This module provides access to MacOS specific functionality in the
Python interpreter, such as how the interpreter eventloop functions
and the like. Use with care.

Note the capitalisation of the module name, this is a historical
artifact.

\begin{excdesc}{Error}
This exception is raised on MacOS generated errors, either from
functions in this module or from other mac-specific modules like the
toolbox interfaces. The arguments are the integer error code (the
\cdata{OSErr} value) and a textual description of the error code.
Symbolic names for all known error codes are defined in the standard
module \module{macerrors}\refstmodindex{macerrors}.
\end{excdesc}

\begin{funcdesc}{SetEventHandler}{handler}
In the inner interpreter loop Python will occasionally check for events,
unless disabled with \function{ScheduleParams()}. With this function you
can pass a Python event-handler function that will be called if an event
is available. The event is passed as parameter and the function should return
non-zero if the event has been fully processed, otherwise event processing
continues (by passing the event to the console window package, for instance).

Call \function{SetEventHandler()} without a parameter to clear the
event handler. Setting an event handler while one is already set is an
error.
\end{funcdesc}

\begin{funcdesc}{SchedParams}{\optional{doint\optional{, evtmask\optional{,
                              besocial\optional{, interval\optional{,
                              bgyield}}}}}}
Influence the interpreter inner loop event handling. \var{Interval}
specifies how often (in seconds, floating point) the interpreter
should enter the event processing code. When true, \var{doint} causes
interrupt (command-dot) checking to be done. \var{evtmask} tells the
interpreter to do event processing for events in the mask (redraws,
mouseclicks to switch to other applications, etc). The \var{besocial}
flag gives other processes a chance to run. They are granted minimal
runtime when Python is in the foreground and \var{bgyield} seconds per
\var{interval} when Python runs in the background.

All parameters are optional, and default to the current value. The return
value of this function is a tuple with the old values of these options.
Initial defaults are that all processing is enabled, checking is done every
quarter second and the CPU is given up for a quarter second when in the
background.
\end{funcdesc}

\begin{funcdesc}{HandleEvent}{ev}
Pass the event record \var{ev} back to the Python event loop, or
possibly to the handler for the \code{sys.stdout} window (based on the
compiler used to build Python). This allows Python programs that do
their own event handling to still have some command-period and
window-switching capability.

If you attempt to call this function from an event handler set through
\function{SetEventHandler()} you will get an exception.
\end{funcdesc}

\begin{funcdesc}{GetErrorString}{errno}
Return the textual description of MacOS error code \var{errno}.
\end{funcdesc}

\begin{funcdesc}{splash}{resid}
This function will put a splash window
on-screen, with the contents of the DLOG resource specified by
\var{resid}. Calling with a zero argument will remove the splash
screen. This function is useful if you want an applet to post a splash screen
early in initialization without first having to load numerous
extension modules.
\end{funcdesc}

\begin{funcdesc}{DebugStr}{message \optional{, object}}
Drop to the low-level debugger with message \var{message}. The
optional \var{object} argument is not used, but can easily be
inspected from the debugger.

Note that you should use this function with extreme care: if no
low-level debugger like MacsBug is installed this call will crash your
system. It is intended mainly for developers of Python extension
modules.
\end{funcdesc}

\begin{funcdesc}{openrf}{name \optional{, mode}}
Open the resource fork of a file. Arguments are the same as for the
built-in function \function{open()}. The object returned has file-like
semantics, but it is not a Python file object, so there may be subtle
differences.
\end{funcdesc}

\section{\module{macostools} ---
         Convenience routines for file manipulation}

\declaremodule{standard}{macostools}
  \platform{Mac}
\modulesynopsis{Convenience routines for file manipulation.}


This module contains some convenience routines for file-manipulation
on the Macintosh.

The \module{macostools} module defines the following functions:


\begin{funcdesc}{copy}{src, dst\optional{, createpath\optional{, copytimes}}}
Copy file \var{src} to \var{dst}. The files can be specified as
pathnames or \pytype{FSSpec} objects. If \var{createpath} is non-zero
\var{dst} must be a pathname and the folders leading to the
destination are created if necessary.  The method copies data and
resource fork and some finder information (creator, type, flags) and
optionally the creation, modification and backup times (default is to
copy them). Custom icons, comments and icon position are not copied.

If the source is an alias the original to which the alias points is
copied, not the aliasfile.
\end{funcdesc}

\begin{funcdesc}{copytree}{src, dst}
Recursively copy a file tree from \var{src} to \var{dst}, creating
folders as needed. \var{src} and \var{dst} should be specified as
pathnames.
\end{funcdesc}

\begin{funcdesc}{mkalias}{src, dst}
Create a finder alias \var{dst} pointing to \var{src}. Both may be
specified as pathnames or \pytype{FSSpec} objects.
\end{funcdesc}

\begin{funcdesc}{touched}{dst}
Tell the finder that some bits of finder-information such as creator
or type for file \var{dst} has changed. The file can be specified by
pathname or fsspec. This call should prod the finder into redrawing the
files icon.
\end{funcdesc}

\begin{datadesc}{BUFSIZ}
The buffer size for \code{copy}, default 1 megabyte.
\end{datadesc}

Note that the process of creating finder aliases is not specified in
the Apple documentation. Hence, aliases created with \function{mkalias()}
could conceivably have incompatible behaviour in some cases.


\section{\module{findertools} ---
         The \program{finder}'s Apple Events interface}

\declaremodule{standard}{findertools}
  \platform{Mac}
\modulesynopsis{Wrappers around the \program{finder}'s Apple Events interface.}


This module contains routines that give Python programs access to some
functionality provided by the finder. They are implemented as wrappers
around the AppleEvent\index{AppleEvents} interface to the finder.

All file and folder parameters can be specified either as full
pathnames or as \pytype{FSSpec} objects.

The \module{findertools} module defines the following functions:


\begin{funcdesc}{launch}{file}
Tell the finder to launch \var{file}. What launching means depends on the file:
applications are started, folders are opened and documents are opened
in the correct application.
\end{funcdesc}

\begin{funcdesc}{Print}{file}
Tell the finder to print a file (again specified by full pathname or
\pytype{FSSpec}). The behaviour is identical to selecting the file and using
the print command in the finder.
\end{funcdesc}

\begin{funcdesc}{copy}{file, destdir}
Tell the finder to copy a file or folder \var{file} to folder
\var{destdir}. The function returns an \pytype{Alias} object pointing to
the new file.
\end{funcdesc}

\begin{funcdesc}{move}{file, destdir}
Tell the finder to move a file or folder \var{file} to folder
\var{destdir}. The function returns an \pytype{Alias} object pointing to
the new file.
\end{funcdesc}

\begin{funcdesc}{sleep}{}
Tell the finder to put the Macintosh to sleep, if your machine
supports it.
\end{funcdesc}

\begin{funcdesc}{restart}{}
Tell the finder to perform an orderly restart of the machine.
\end{funcdesc}

\begin{funcdesc}{shutdown}{}
Tell the finder to perform an orderly shutdown of the machine.
\end{funcdesc}

\section{\module{mactcp} ---
         The MacTCP interfaces}

\declaremodule{builtin}{mactcp}
  \platform{Mac}
\modulesynopsis{The MacTCP interfaces.}


This module provides an interface to the Macintosh TCP/IP driver%
\index{MacTCP} MacTCP. There is an accompanying module,
\refmodule{macdnr}\refbimodindex{macdnr}, which provides an interface
to the name-server (allowing you to translate hostnames to IP
addresses), a module \module{MACTCPconst}\refstmodindex{MACTCPconst}
which has symbolic names for constants constants used by MacTCP. Since
the built-in module \module{socket}\refbimodindex{socket} is also
available on the Macintosh it is usually easier to use sockets instead
of the Macintosh-specific MacTCP API.

A complete description of the MacTCP interface can be found in the
Apple MacTCP API documentation.

\begin{funcdesc}{MTU}{}
Return the Maximum Transmit Unit (the packet size) of the network
interface.\index{Maximum Transmit Unit}
\end{funcdesc}

\begin{funcdesc}{IPAddr}{}
Return the 32-bit integer IP address of the network interface.
\end{funcdesc}

\begin{funcdesc}{NetMask}{}
Return the 32-bit integer network mask of the interface.
\end{funcdesc}

\begin{funcdesc}{TCPCreate}{size}
Create a TCP Stream object. \var{size} is the size of the receive
buffer, \code{4096} is suggested by various sources.
\end{funcdesc}

\begin{funcdesc}{UDPCreate}{size, port}
Create a UDP Stream object. \var{size} is the size of the receive
buffer (and, hence, the size of the biggest datagram you can receive
on this port). \var{port} is the UDP port number you want to receive
datagrams on, a value of zero will make MacTCP select a free port.
\end{funcdesc}


\subsection{TCP Stream Objects}

\begin{memberdesc}[TCP Stream]{asr}
\index{asynchronous service routine}
\index{service routine, asynchronous}
When set to a value different than \code{None} this should refer to a
function with two integer parameters:\ an event code and a detail. This
function will be called upon network-generated events such as urgent
data arrival.  Macintosh documentation calls this the
\dfn{asynchronous service routine}.  In addition, it is called with
eventcode \code{MACTCP.PassiveOpenDone} when a \method{PassiveOpen()}
completes. This is a Python addition to the MacTCP semantics.
It is safe to do further calls from \var{asr}.
\end{memberdesc}


\begin{methoddesc}[TCP Stream]{PassiveOpen}{port}
Wait for an incoming connection on TCP port \var{port} (zero makes the
system pick a free port). The call returns immediately, and you should
use \method{wait()} to wait for completion. You should not issue any method
calls other than \method{wait()}, \method{isdone()} or
\method{GetSockName()} before the call completes.
\end{methoddesc}

\begin{methoddesc}[TCP Stream]{wait}{}
Wait for \method{PassiveOpen()} to complete.
\end{methoddesc}

\begin{methoddesc}[TCP Stream]{isdone}{}
Return \code{1} if a \method{PassiveOpen()} has completed.
\end{methoddesc}

\begin{methoddesc}[TCP Stream]{GetSockName}{}
Return the TCP address of this side of a connection as a 2-tuple
\code{(\var{host}, \var{port})}, both integers.
\end{methoddesc}

\begin{methoddesc}[TCP Stream]{ActiveOpen}{lport, host, rport}
Open an outgoing connection to TCP address \code{(\var{host},
\var{rport})}. Use
local port \var{lport} (zero makes the system pick a free port). This
call blocks until the connection has been established.
\end{methoddesc}

\begin{methoddesc}[TCP Stream]{Send}{buf, push, urgent}
Send data \var{buf} over the connection. \var{push} and \var{urgent}
are flags as specified by the TCP standard.
\end{methoddesc}

\begin{methoddesc}[TCP Stream]{Rcv}{timeout}
Receive data. The call returns when \var{timeout} seconds have passed
or when (according to the MacTCP documentation) ``a reasonable amount
of data has been received''. The return value is a 3-tuple
\code{(\var{data}, \var{urgent}, \var{mark})}. If urgent data is
outstanding \code{Rcv} will always return that before looking at any
normal data. The first call returning urgent data will have the
\var{urgent} flag set, the last will have the \var{mark} flag set.
\end{methoddesc}

\begin{methoddesc}[TCP Stream]{Close}{}
Tell MacTCP that no more data will be transmitted on this
connection. The call returns when all data has been acknowledged by
the receiving side.
\end{methoddesc}

\begin{methoddesc}[TCP Stream]{Abort}{}
Forcibly close both sides of a connection, ignoring outstanding data.
\end{methoddesc}

\begin{methoddesc}[TCP Stream]{Status}{}
Return a TCP status object for this stream giving the current status
(see below).
\end{methoddesc}


\subsection{TCP Status Objects}

This object has no methods, only some members holding information on
the connection. A complete description of all fields in this objects
can be found in the Apple documentation. The most interesting ones are:

\begin{memberdesc}[TCP Status]{localHost}
\memberline{localPort}
\memberline{remoteHost}
\memberline{remotePort}
The integer IP-addresses and port numbers of both endpoints of the
connection. 
\end{memberdesc}

\begin{memberdesc}[TCP Status]{sendWindow}
The current window size.
\end{memberdesc}

\begin{memberdesc}[TCP Status]{amtUnackedData}
The number of bytes sent but not yet acknowledged. \code{sendWindow -
amtUnackedData} is what you can pass to \method{Send()} without
blocking.
\end{memberdesc}

\begin{memberdesc}[TCP Status]{amtUnreadData}
The number of bytes received but not yet read (what you can
\method{Recv()} without blocking).
\end{memberdesc}



\subsection{UDP Stream Objects}

Note that, unlike the name suggests, there is nothing stream-like
about UDP.


\begin{memberdesc}[UDP Stream]{asr}
\index{asynchronous service routine}
\index{service routine, asynchronous}
The asynchronous service routine to be called on events such as
datagram arrival without outstanding \code{Read} call. The \var{asr}
has a single argument, the event code.
\end{memberdesc}

\begin{memberdesc}[UDP Stream]{port}
A read-only member giving the port number of this UDP Stream.
\end{memberdesc}


\begin{methoddesc}[UDP Stream]{Read}{timeout}
Read a datagram, waiting at most \var{timeout} seconds (-1 is
infinite).  Return the data.
\end{methoddesc}

\begin{methoddesc}[UDP Stream]{Write}{host, port, buf}
Send \var{buf} as a datagram to IP-address \var{host}, port
\var{port}.
\end{methoddesc}

\section{\module{macspeech} ---
         Interface to the Macintosh Speech Manager}

\declaremodule{builtin}{macspeech}
  \platform{Mac}
\modulesynopsis{Interface to the Macintosh Speech Manager.}


This module provides an interface to the Macintosh Speech Manager,
\index{Macintosh Speech Manager}
\index{Speech Manager, Macintosh}
allowing you to let the Macintosh utter phrases. You need a version of
the Speech Manager extension (version 1 and 2 have been tested) in
your \file{Extensions} folder for this to work. The module does not
provide full access to all features of the Speech Manager yet.  It may
not be available in all Mac Python versions.

\begin{funcdesc}{Available}{}
Test availability of the Speech Manager extension (and, on the
PowerPC, the Speech Manager shared library). Return \code{0} or
\code{1}.
\end{funcdesc}

\begin{funcdesc}{Version}{}
Return the (integer) version number of the Speech Manager.
\end{funcdesc}

\begin{funcdesc}{SpeakString}{str}
Utter the string \var{str} using the default voice,
asynchronously. This aborts any speech that may still be active from
prior \function{SpeakString()} invocations.
\end{funcdesc}

\begin{funcdesc}{Busy}{}
Return the number of speech channels busy, system-wide.
\end{funcdesc}

\begin{funcdesc}{CountVoices}{}
Return the number of different voices available.
\end{funcdesc}

\begin{funcdesc}{GetIndVoice}{num}
Return a \pytype{Voice} object for voice number \var{num}.
\end{funcdesc}

\subsection{Voice Objects}
\label{voice-objects}

Voice objects contain the description of a voice. It is currently not
yet possible to access the parameters of a voice.

\setindexsubitem{(voice object method)}

\begin{methoddesc}[Voice]{GetGender}{}
Return the gender of the voice: \code{0} for male, \code{1} for female
and \code{-1} for neuter.
\end{methoddesc}

\begin{methoddesc}[Voice]{NewChannel}{}
Return a new Speech Channel object using this voice.
\end{methoddesc}

\subsection{Speech Channel Objects}
\label{speech-channel-objects}

A Speech Channel object allows you to speak strings with slightly more
control than \function{SpeakString()}, and allows you to use multiple
speakers at the same time. Please note that channel pitch and rate are
interrelated in some way, so that to make your Macintosh sing you will
have to adjust both.

\begin{methoddesc}[Speech Channel]{SpeakText}{str}
Start uttering the given string.
\end{methoddesc}

\begin{methoddesc}[Speech Channel]{Stop}{}
Stop babbling.
\end{methoddesc}

\begin{methoddesc}[Speech Channel]{GetPitch}{}
Return the current pitch of the channel, as a floating-point number.
\end{methoddesc}

\begin{methoddesc}[Speech Channel]{SetPitch}{pitch}
Set the pitch of the channel.
\end{methoddesc}

\begin{methoddesc}[Speech Channel]{GetRate}{}
Get the speech rate (utterances per minute) of the channel as a
floating point number.
\end{methoddesc}

\begin{methoddesc}[Speech Channel]{SetRate}{rate}
Set the speech rate of the channel.
\end{methoddesc}


\section{\module{EasyDialogs} ---
         Basic Macintosh dialogs}

\declaremodule{standard}{EasyDialogs}
  \platform{Mac}
\modulesynopsis{Basic Macintosh dialogs.}


The \module{EasyDialogs} module contains some simple dialogs for
the Macintosh, modelled after the
\module{stdwin}\refbimodindex{stdwin} dialogs with similar names. All
routines have an optional parameter \var{id} with which you can
override the DLOG resource used for the dialog, as long as the item
numbers correspond. See the source for details.
 
The \module{EasyDialogs} module defines the following functions:


\begin{funcdesc}{Message}{str}
A modal dialog with the message text \var{str}, which should be at
most 255 characters long, is displayed. Control is returned when the
user clicks ``OK''.
\end{funcdesc}

\begin{funcdesc}{AskString}{prompt\optional{, default}}
Ask the user to input a string value, in a modal dialog. \var{prompt}
is the promt message, the optional \var{default} arg is the initial
value for the string. All strings can be at most 255 bytes
long. \function{AskString()} returns the string entered or \code{None}
in case the user cancelled.
\end{funcdesc}

\begin{funcdesc}{AskYesNoCancel}{question\optional{, default}}
Present a dialog with text \var{question} and three buttons labelled
``yes'', ``no'' and ``cancel''. Return \code{1} for yes, \code{0} for
no and \code{-1} for cancel. The default return value chosen by
hitting return is \code{0}. This can be changed with the optional
\var{default} argument.
\end{funcdesc}

\begin{funcdesc}{ProgressBar}{\optional{label\optional{, maxval}}}
Display a modeless progress dialog with a thermometer bar. \var{label}
is the text string displayed (default ``Working...''), \var{maxval} is
the value at which progress is complete (default \code{100}). The
returned object has one method, \code{set(\var{value})}, which sets
the value of the progress bar. The bar remains visible until the
object returned is discarded.

The progress bar has a ``cancel'' button, but it is currently
non-functional.
\end{funcdesc}

Note that \module{EasyDialogs} does not currently use the notification
manager. This means that displaying dialogs while the program is in
the background will lead to unexpected results and possibly
crashes. Also, all dialogs are modeless and hence expect to be at the
top of the stacking order. This is true when the dialogs are created,
but windows that pop-up later (like a console window) may also result
in crashes.

\section{\module{FrameWork} ---
         Interactive application framework}

\declaremodule{standard}{FrameWork}
  \platform{Mac}
\modulesynopsis{Interactive application framework.}


The \module{FrameWork} module contains classes that together provide a
framework for an interactive Macintosh application. The programmer
builds an application by creating subclasses that override various
methods of the bases classes, thereby implementing the functionality
wanted. Overriding functionality can often be done on various
different levels, i.e. to handle clicks in a single dialog window in a
non-standard way it is not necessary to override the complete event
handling.

The \module{FrameWork} is still very much work-in-progress, and the
documentation describes only the most important functionality, and not
in the most logical manner at that. Examine the source or the examples
for more details.

The \module{FrameWork} module defines the following functions:


\begin{funcdesc}{Application}{}
An object representing the complete application. See below for a
description of the methods. The default \method{__init__()} routine
creates an empty window dictionary and a menu bar with an apple menu.
\end{funcdesc}

\begin{funcdesc}{MenuBar}{}
An object representing the menubar. This object is usually not created
by the user.
\end{funcdesc}

\begin{funcdesc}{Menu}{bar, title\optional{, after}}
An object representing a menu. Upon creation you pass the
\code{MenuBar} the menu appears in, the \var{title} string and a
position (1-based) \var{after} where the menu should appear (default:
at the end).
\end{funcdesc}

\begin{funcdesc}{MenuItem}{menu, title\optional{, shortcut, callback}}
Create a menu item object. The arguments are the menu to crate the
item it, the item title string and optionally the keyboard shortcut
and a callback routine. The callback is called with the arguments
menu-id, item number within menu (1-based), current front window and
the event record.

In stead of a callable object the callback can also be a string. In
this case menu selection causes the lookup of a method in the topmost
window and the application. The method name is the callback string
with \code{'domenu_'} prepended.

Calling the \code{MenuBar} \method{fixmenudimstate()} method sets the
correct dimming for all menu items based on the current front window.
\end{funcdesc}

\begin{funcdesc}{Separator}{menu}
Add a separator to the end of a menu.
\end{funcdesc}

\begin{funcdesc}{SubMenu}{menu, label}
Create a submenu named \var{label} under menu \var{menu}. The menu
object is returned.
\end{funcdesc}

\begin{funcdesc}{Window}{parent}
Creates a (modeless) window. \var{Parent} is the application object to
which the window belongs. The window is not displayed until later.
\end{funcdesc}

\begin{funcdesc}{DialogWindow}{parent}
Creates a modeless dialog window.
\end{funcdesc}

\begin{funcdesc}{windowbounds}{width, height}
Return a \code{(\var{left}, \var{top}, \var{right}, \var{bottom})}
tuple suitable for creation of a window of given width and height. The
window will be staggered with respect to previous windows, and an
attempt is made to keep the whole window on-screen. The window will
however always be exact the size given, so parts may be offscreen.
\end{funcdesc}

\begin{funcdesc}{setwatchcursor}{}
Set the mouse cursor to a watch.
\end{funcdesc}

\begin{funcdesc}{setarrowcursor}{}
Set the mouse cursor to an arrow.
\end{funcdesc}


\subsection{Application Objects \label{application-objects}}

Application objects have the following methods, among others:


\begin{methoddesc}[Application]{makeusermenus}{}
Override this method if you need menus in your application. Append the
menus to the attribute \member{menubar}.
\end{methoddesc}

\begin{methoddesc}[Application]{getabouttext}{}
Override this method to return a text string describing your
application.  Alternatively, override the \method{do_about()} method
for more elaborate ``about'' messages.
\end{methoddesc}

\begin{methoddesc}[Application]{mainloop}{\optional{mask\optional{, wait}}}
This routine is the main event loop, call it to set your application
rolling. \var{Mask} is the mask of events you want to handle,
\var{wait} is the number of ticks you want to leave to other
concurrent application (default 0, which is probably not a good
idea). While raising \var{self} to exit the mainloop is still
supported it is not recommended: call \code{self._quit()} instead.

The event loop is split into many small parts, each of which can be
overridden. The default methods take care of dispatching events to
windows and dialogs, handling drags and resizes, Apple Events, events
for non-FrameWork windows, etc.

In general, all event handlers should return \code{1} if the event is fully
handled and \code{0} otherwise (because the front window was not a FrameWork
window, for instance). This is needed so that update events and such
can be passed on to other windows like the Sioux console window.
Calling \function{MacOS.HandleEvent()} is not allowed within
\var{our_dispatch} or its callees, since this may result in an
infinite loop if the code is called through the Python inner-loop
event handler.
\end{methoddesc}

\begin{methoddesc}[Application]{asyncevents}{onoff}
Call this method with a nonzero parameter to enable
asynchronous event handling. This will tell the inner interpreter loop
to call the application event handler \var{async_dispatch} whenever events
are available. This will cause FrameWork window updates and the user
interface to remain working during long computations, but will slow the
interpreter down and may cause surprising results in non-reentrant code
(such as FrameWork itself). By default \var{async_dispatch} will immedeately
call \var{our_dispatch} but you may override this to handle only certain
events asynchronously. Events you do not handle will be passed to Sioux
and such.

The old on/off value is returned.
\end{methoddesc}

\begin{methoddesc}[Application]{_quit}{}
Terminate the running \method{mainloop()} call at the next convenient
moment.
\end{methoddesc}

\begin{methoddesc}[Application]{do_char}{c, event}
The user typed character \var{c}. The complete details of the event
can be found in the \var{event} structure. This method can also be
provided in a \code{Window} object, which overrides the
application-wide handler if the window is frontmost.
\end{methoddesc}

\begin{methoddesc}[Application]{do_dialogevent}{event}
Called early in the event loop to handle modeless dialog events. The
default method simply dispatches the event to the relevant dialog (not
through the the \code{DialogWindow} object involved). Override if you
need special handling of dialog events (keyboard shortcuts, etc).
\end{methoddesc}

\begin{methoddesc}[Application]{idle}{event}
Called by the main event loop when no events are available. The
null-event is passed (so you can look at mouse position, etc).
\end{methoddesc}


\subsection{Window Objects \label{window-objects}}

Window objects have the following methods, among others:

\setindexsubitem{(Window method)}

\begin{methoddesc}[Window]{open}{}
Override this method to open a window. Store the MacOS window-id in
\member{self.wid} and call the \method{do_postopen()} method to
register the window with the parent application.
\end{methoddesc}

\begin{methoddesc}[Window]{close}{}
Override this method to do any special processing on window
close. Call the \method{do_postclose()} method to cleanup the parent
state.
\end{methoddesc}

\begin{methoddesc}[Window]{do_postresize}{width, height, macoswindowid}
Called after the window is resized. Override if more needs to be done
than calling \code{InvalRect}.
\end{methoddesc}

\begin{methoddesc}[Window]{do_contentclick}{local, modifiers, event}
The user clicked in the content part of a window. The arguments are
the coordinates (window-relative), the key modifiers and the raw
event.
\end{methoddesc}

\begin{methoddesc}[Window]{do_update}{macoswindowid, event}
An update event for the window was received. Redraw the window.
\end{methoddesc}

\begin{methoddesc}{do_activate}{activate, event}
The window was activated (\code{\var{activate} == 1}) or deactivated
(\code{\var{activate} == 0}). Handle things like focus highlighting,
etc.
\end{methoddesc}


\subsection{ControlsWindow Object \label{controlswindow-object}}

ControlsWindow objects have the following methods besides those of
\code{Window} objects:


\begin{methoddesc}[ControlsWindow]{do_controlhit}{window, control,
                                                  pcode, event}
Part \var{pcode} of control \var{control} was hit by the
user. Tracking and such has already been taken care of.
\end{methoddesc}


\subsection{ScrolledWindow Object \label{scrolledwindow-object}}

ScrolledWindow objects are ControlsWindow objects with the following
extra methods:


\begin{methoddesc}[ScrolledWindow]{scrollbars}{\optional{wantx\optional{,
                                               wanty}}}
Create (or destroy) horizontal and vertical scrollbars. The arguments
specify which you want (default: both). The scrollbars always have
minimum \code{0} and maximum \code{32767}.
\end{methoddesc}

\begin{methoddesc}[ScrolledWindow]{getscrollbarvalues}{}
You must supply this method. It should return a tuple \code{(\var{x},
\var{y})} giving the current position of the scrollbars (between
\code{0} and \code{32767}). You can return \code{None} for either to
indicate the whole document is visible in that direction.
\end{methoddesc}

\begin{methoddesc}[ScrolledWindow]{updatescrollbars}{}
Call this method when the document has changed. It will call
\method{getscrollbarvalues()} and update the scrollbars.
\end{methoddesc}

\begin{methoddesc}[ScrolledWindow]{scrollbar_callback}{which, what, value}
Supplied by you and called after user interaction. \var{which} will
be \code{'x'} or \code{'y'}, \var{what} will be \code{'-'},
\code{'--'}, \code{'set'}, \code{'++'} or \code{'+'}. For
\code{'set'}, \var{value} will contain the new scrollbar position.
\end{methoddesc}

\begin{methoddesc}[ScrolledWindow]{scalebarvalues}{absmin, absmax,
                                                   curmin, curmax}
Auxiliary method to help you calculate values to return from
\method{getscrollbarvalues()}. You pass document minimum and maximum value
and topmost (leftmost) and bottommost (rightmost) visible values and
it returns the correct number or \code{None}.
\end{methoddesc}

\begin{methoddesc}[ScrolledWindow]{do_activate}{onoff, event}
Takes care of dimming/highlighting scrollbars when a window becomes
frontmost vv. If you override this method call this one at the end of
your method.
\end{methoddesc}

\begin{methoddesc}[ScrolledWindow]{do_postresize}{width, height, window}
Moves scrollbars to the correct position. Call this method initially
if you override it.
\end{methoddesc}

\begin{methoddesc}[ScrolledWindow]{do_controlhit}{window, control,
                                                  pcode, event}
Handles scrollbar interaction. If you override it call this method
first, a nonzero return value indicates the hit was in the scrollbars
and has been handled.
\end{methoddesc}


\subsection{DialogWindow Objects \label{dialogwindow-objects}}

DialogWindow objects have the following methods besides those of
\code{Window} objects:


\begin{methoddesc}[DialogWindow]{open}{resid}
Create the dialog window, from the DLOG resource with id
\var{resid}. The dialog object is stored in \member{self.wid}.
\end{methoddesc}

\begin{methoddesc}[DialogWindow]{do_itemhit}{item, event}
Item number \var{item} was hit. You are responsible for redrawing
toggle buttons, etc.
\end{methoddesc}

\section{\module{MiniAEFrame} ---
         Open Scripting Architecture server support}

\declaremodule{standard}{MiniAEFrame}
  \platform{Mac}
\modulesynopsis{Support to act as an Open Scripting Architecture (OSA) server
(``Apple Events'').}


The module \module{MiniAEFrame} provides a framework for an application
that can function as an Open Scripting Architecture
\index{Open Scripting Architecture}
(OSA) server, i.e. receive and process
AppleEvents\index{AppleEvents}. It can be used in conjunction with
\refmodule{FrameWork}\refstmodindex{FrameWork} or standalone.

This module is temporary, it will eventually be replaced by a module
that handles argument names better and possibly automates making your
application scriptable.

The \module{MiniAEFrame} module defines the following classes:


\begin{classdesc}{AEServer}{}
A class that handles AppleEvent dispatch. Your application should
subclass this class together with either
\class{MiniApplication} or
\class{FrameWork.Application}. Your \method{__init__()} method should
call the \method{__init__()} method for both classes.
\end{classdesc}

\begin{classdesc}{MiniApplication}{}
A class that is more or less compatible with
\class{FrameWork.Application} but with less functionality. Its
event loop supports the apple menu, command-dot and AppleEvents; other
events are passed on to the Python interpreter and/or Sioux.
Useful if your application wants to use \class{AEServer} but does not
provide its own windows, etc.
\end{classdesc}


\subsection{AEServer Objects \label{aeserver-objects}}

\begin{methoddesc}[AEServer]{installaehandler}{classe, type, callback}
Installs an AppleEvent handler. \var{classe} and \var{type} are the
four-character OSA Class and Type designators, \code{'****'} wildcards
are allowed. When a matching AppleEvent is received the parameters are
decoded and your callback is invoked.
\end{methoddesc}

\begin{methoddesc}[AEServer]{callback}{_object, **kwargs}
Your callback is called with the OSA Direct Object as first positional
parameter. The other parameters are passed as keyword arguments, with
the 4-character designator as name. Three extra keyword parameters are
passed: \code{_class} and \code{_type} are the Class and Type
designators and \code{_attributes} is a dictionary with the AppleEvent
attributes.

The return value of your method is packed with
\function{aetools.packevent()} and sent as reply.
\end{methoddesc}

Note that there are some serious problems with the current
design. AppleEvents which have non-identifier 4-character designators
for arguments are not implementable, and it is not possible to return
an error to the originator. This will be addressed in a future
release.


%
%  The ugly "%begin{latexonly}" pseudo-environments are really just to
%  keep LaTeX2HTML quiet during the \renewcommand{} macros; they're
%  not really valuable.
%

%begin{latexonly}
\renewcommand{\indexname}{Module Index}
%end{latexonly}
\input{modmac.ind}		% Module Index

%begin{latexonly}
\renewcommand{\indexname}{Index}
%end{latexonly}
\documentclass{howto}

\title{Macintosh Library Modules}

\author{Guido van Rossum \\
	Fred L. Drake, Jr., editor}
\authoraddress{
	BeOpen PythonLabs \\
	E-mail: \email{python-docs@python.org}
}

\date{September 5, 2000}			% XXX update before release!
\release{1.6}


\makeindex			% tell \index to actually write the
				% .idx file
\makemodindex			% ... and the module index as well.
%\ignorePlatformAnnotation{Mac}


\begin{document}

\maketitle

\ifhtml
\chapter*{Front Matter\label{front}}
\fi

Copyright \copyright{} 1995-2000 Corporation for National Research
Initiatives.  All rights reserved.

Copyright \copyright{} 1991-1995 Stichting Mathematisch Centrum.  All
rights reserved.

\centerline{\strong{CNRI OPEN SOURCE LICENSE AGREEMENT}}


IMPORTANT: PLEASE READ THE FOLLOWING AGREEMENT CAREFULLY.

BY CLICKING ON ``ACCEPT'' WHERE INDICATED BELOW, OR BY COPYING,
INSTALLING OR OTHERWISE USING PYTHON 1.6 SOFTWARE, YOU ARE DEEMED TO
HAVE AGREED TO THE TERMS AND CONDITIONS OF THIS LICENSE AGREEMENT.

\begin{enumerate}
\item
This LICENSE AGREEMENT is between the Corporation for National
Research Initiatives, having an office at 1895 Preston White Drive,
Reston, VA 20191 (``CNRI''), and the Individual or Organization
(``Licensee'') accessing and otherwise using Python 1.6 software in
source or binary form and its associated documentation, as released at
the www.python.org Internet site on September 5, 2000 (``Python 1.6'').

\item
Subject to the terms and conditions of this License Agreement, CNRI
hereby grants Licensee a nonexclusive, royalty-free, world-wide
license to reproduce, analyze, test, perform and/or display publicly,
prepare derivative works, distribute, and otherwise use Python 1.6
alone or in any derivative version, provided, however, that CNRI's
License Agreement and CNRI's notice of copyright, i.e., ``Copyright (c)
1995-2000 Corporation for National Research Initiatives; All Rights
Reserved'' are retained in Python 1.6 alone or in any derivative
version prepared by Licensee.

Alternately, in lieu of CNRI's License Agreement, Licensee may
substitute the following text (omitting the quotes): ``Python 1.6 is
made available subject to the terms and conditions in CNRI's License
Agreement.  This Agreement together with Python 1.6 may be located on
the Internet using the following unique, persistent identifier (known
as a handle): 1895.22/1012.  This Agreement may also be obtained from a
proxy server on the Internet using the following URL:
http://hdl.handle.net/1895.22/1012''.

\item
In the event Licensee prepares a derivative work that is based on
or incorporates Python 1.6 or any part thereof, and wants to make the
derivative work available to others as provided herein, then Licensee
hereby agrees to include in any such work a brief summary of the
changes made to Python 1.6.

\item
CNRI is making Python 1.6 available to Licensee on an ``AS IS''
basis.  CNRI MAKES NO REPRESENTATIONS OR WARRANTIES, EXPRESS OR
IMPLIED.  BY WAY OF EXAMPLE, BUT NOT LIMITATION, CNRI MAKES NO AND
DISCLAIMS ANY REPRESENTATION OR WARRANTY OF MERCHANTABILITY OR FITNESS
FOR ANY PARTICULAR PURPOSE OR THAT THE USE OF PYTHON 1.6 WILL NOT
INFRINGE ANY THIRD PARTY RIGHTS.

\item
CNRI SHALL NOT BE LIABLE TO LICENSEE OR ANY OTHER USERS OF PYTHON
1.6 FOR ANY INCIDENTAL, SPECIAL, OR CONSEQUENTIAL DAMAGES OR LOSS AS A
RESULT OF MODIFYING, DISTRIBUTING, OR OTHERWISE USING PYTHON 1.6, OR
ANY DERIVATIVE THEREOF, EVEN IF ADVISED OF THE POSSIBILITY THEREOF.

\item
This License Agreement will automatically terminate upon a material
breach of its terms and conditions.

\item
This License Agreement shall be governed by and interpreted in all
respects by the law of the State of Virginia, excluding conflict of
law provisions.  Nothing in this License Agreement shall be deemed to
create any relationship of agency, partnership, or joint venture
between CNRI and Licensee.  This License Agreement does not grant
permission to use CNRI trademarks or trade name in a trademark sense
to endorse or promote products or services of Licensee, or any third
party.

\item
By clicking on the ``ACCEPT'' button where indicated, or by copying,
installing or otherwise using Python 1.6, Licensee agrees to be bound
by the terms and conditions of this License Agreement.

\centerline{ACCEPT}
\end{enumerate}


\begin{abstract}

\noindent
This library reference manual documents Python's extensions for the
Macintosh.  It should be used in conjunction with the
\citetitle[../lib/lib.html]{Python Library Reference}, which documents
the standard library and built-in types.

This manual assumes basic knowledge about the Python language.  For an
informal introduction to Python, see the
\citetitle[../tut/tut.html]{Python Tutorial}; the
\citetitle[../ref/ref.html]{Python Reference Manual} remains the
highest authority on syntactic and semantic questions.  Finally, the
manual entitled \citetitle[../ext/ext.html]{Extending and Embedding
the Python Interpreter} describes how to add new extensions to Python
and how to embed it in other applications.

\end{abstract}

\tableofcontents

\section{Introduction}
\label{intro}

The modules in this manual are available on the Apple Macintosh only.

Aside from the modules described here there are also interfaces to
various MacOS toolboxes, which are currently not extensively
described. The toolboxes for which modules exist are:
\module{AE} (Apple Events),
\module{Cm} (Component Manager),
\module{Ctl} (Control Manager),
\module{Dlg} (Dialog Manager),
\module{Evt} (Event Manager),
\module{Fm} (Font Manager),
\module{List} (List Manager),
\module{Menu} (Moenu Manager),
\module{Qd} (QuickDraw),
\module{Qt} (QuickTime),
\module{Res} (Resource Manager and Handles),
\module{Scrap} (Scrap Manager),
\module{Snd} (Sound Manager),
\module{TE} (TextEdit),
\module{Waste} (non-Apple \program{TextEdit} replacement) and
\module{Win} (Window Manager).

If applicable the module will define a number of Python objects for
the various structures declared by the toolbox, and operations will be
implemented as methods of the object. Other operations will be
implemented as functions in the module. Not all operations possible in
\C{} will also be possible in Python (callbacks are often a problem), and
parameters will occasionally be different in Python (input and output
buffers, especially). All methods and functions have a \code{__doc__}
string describing their arguments and return values, and for
additional description you are referred to \citetitle{Inside
Macintosh} or similar works.

The following modules are documented here:

\localmoduletable


\section{\module{mac} ---
         Implementations for the \module{os} module}

\declaremodule{builtin}{mac}
  \platform{Mac}
\modulesynopsis{Implementations for the \module{os} module.}


This module implements the operating system dependent functionality
provided by the standard module \module{os}\refstmodindex{os}.  It is
best accessed through the \module{os} module.

The following functions are available in this module:
\function{chdir()},
\function{close()},
\function{dup()},
\function{fdopen()},
\function{getcwd()},
\function{lseek()},
\function{listdir()},
\function{mkdir()},
\function{open()},
\function{read()},
\function{rename()},
\function{rmdir()},
\function{stat()},
\function{sync()},
\function{unlink()},
\function{write()},
as well as the exception \exception{error}. Note that the times
returned by \function{stat()} are floating-point values, like all time
values in MacPython.

One additional function is available:

\begin{funcdesc}{xstat}{path}
  This function returns the same information as \function{stat()}, but
  with three additional values appended: the size of the resource fork
  of the file and its 4-character creator and type.
\end{funcdesc}


\section{\module{macpath} ---
         MacOS path manipulation functions}

\declaremodule{standard}{macpath}
% Could be labeled \platform{Mac}, but the module should work anywhere and
% is distributed with the standard library.
\modulesynopsis{MacOS path manipulation functions.}


This module is the Macintosh implementation of the \module{os.path}
module.  It is most portably accessed as
\module{os.path}\refstmodindex{os.path}.  Refer to the
\citetitle[../lib/lib.html]{Python Library Reference} for
documentation of \module{os.path}.

The following functions are available in this module:
\function{normcase()},
\function{normpath()},
\function{isabs()},
\function{join()},
\function{split()},
\function{isdir()},
\function{isfile()},
\function{walk()},
\function{exists()}.
For other functions available in \module{os.path} dummy counterparts
are available.
			% MACINTOSH ONLY
\section{\module{ctb} ---
         Interface to the Communications Tool Box}

\declaremodule{builtin}{ctb}
  \platform{Mac}
\modulesynopsis{Interfaces to the Communications Tool Box.  Only the Connection
Manager is supported.}


This module provides a partial interface to the Macintosh
Communications Toolbox. Currently, only Connection Manager tools are
supported.  It may not be available in all Mac Python versions.
\index{Communications Toolbox, Macintosh}
\index{Macintosh Communications Toolbox}
\index{Connection Manager}

\begin{datadesc}{error}
The exception raised on errors.
\end{datadesc}

\begin{datadesc}{cmData}
\dataline{cmCntl}
\dataline{cmAttn}
Flags for the \var{channel} argument of the \method{Read()} and
\method{Write()} methods.
\end{datadesc}

\begin{datadesc}{cmFlagsEOM}
End-of-message flag for \method{Read()} and \method{Write()}.
\end{datadesc}

\begin{datadesc}{choose*}
Values returned by \method{Choose()}.
\end{datadesc}

\begin{datadesc}{cmStatus*}
Bits in the status as returned by \method{Status()}.
\end{datadesc}

\begin{funcdesc}{available}{}
Return \code{1} if the Communication Toolbox is available, zero otherwise.
\end{funcdesc}

\begin{funcdesc}{CMNew}{name, sizes}
Create a connection object using the connection tool named
\var{name}. \var{sizes} is a 6-tuple given buffer sizes for data in,
data out, control in, control out, attention in and attention out.
Alternatively, passing \code{None} for \var{sizes} will result in
default buffer sizes.
\end{funcdesc}


\subsection{Connection Objects \label{connection-object}}

For all connection methods that take a \var{timeout} argument, a value
of \code{-1} is indefinite, meaning that the command runs to completion.

\begin{memberdesc}[connection]{callback}
If this member is set to a value other than \code{None} it should point
to a function accepting a single argument (the connection
object). This will make all connection object methods work
asynchronously, with the callback routine being called upon
completion.

\emph{Note:} for reasons beyond my understanding the callback routine
is currently never called. You are advised against using asynchronous
calls for the time being.
\end{memberdesc}


\begin{methoddesc}[connection]{Open}{timeout}
Open an outgoing connection, waiting at most \var{timeout} seconds for
the connection to be established.
\end{methoddesc}

\begin{methoddesc}[connection]{Listen}{timeout}
Wait for an incoming connection. Stop waiting after \var{timeout}
seconds. This call is only meaningful to some tools.
\end{methoddesc}

\begin{methoddesc}[connection]{accept}{yesno}
Accept (when \var{yesno} is non-zero) or reject an incoming call after
\method{Listen()} returned.
\end{methoddesc}

\begin{methoddesc}[connection]{Close}{timeout, now}
Close a connection. When \var{now} is zero, the close is orderly
(i.e.\ outstanding output is flushed, etc.)\ with a timeout of
\var{timeout} seconds. When \var{now} is non-zero the close is
immediate, discarding output.
\end{methoddesc}

\begin{methoddesc}[connection]{Read}{len, chan, timeout}
Read \var{len} bytes, or until \var{timeout} seconds have passed, from 
the channel \var{chan} (which is one of \constant{cmData},
\constant{cmCntl} or \constant{cmAttn}). Return a 2-tuple:\ the data
read and the end-of-message flag, \constant{cmFlagsEOM}.
\end{methoddesc}

\begin{methoddesc}[connection]{Write}{buf, chan, timeout, eom}
Write \var{buf} to channel \var{chan}, aborting after \var{timeout}
seconds. When \var{eom} has the value \constant{cmFlagsEOM}, an
end-of-message indicator will be written after the data (if this
concept has a meaning for this communication tool). The method returns
the number of bytes written.
\end{methoddesc}

\begin{methoddesc}[connection]{Status}{}
Return connection status as the 2-tuple \code{(\var{sizes},
\var{flags})}. \var{sizes} is a 6-tuple giving the actual buffer sizes used
(see \function{CMNew()}), \var{flags} is a set of bits describing the state
of the connection.
\end{methoddesc}

\begin{methoddesc}[connection]{GetConfig}{}
Return the configuration string of the communication tool. These
configuration strings are tool-dependent, but usually easily parsed
and modified.
\end{methoddesc}

\begin{methoddesc}[connection]{SetConfig}{str}
Set the configuration string for the tool. The strings are parsed
left-to-right, with later values taking precedence. This means
individual configuration parameters can be modified by simply appending
something like \code{'baud 4800'} to the end of the string returned by
\method{GetConfig()} and passing that to this method. The method returns
the number of characters actually parsed by the tool before it
encountered an error (or completed successfully).
\end{methoddesc}

\begin{methoddesc}[connection]{Choose}{}
Present the user with a dialog to choose a communication tool and
configure it. If there is an outstanding connection some choices (like
selecting a different tool) may cause the connection to be
aborted. The return value (one of the \constant{choose*} constants) will
indicate this.
\end{methoddesc}

\begin{methoddesc}[connection]{Idle}{}
Give the tool a chance to use the processor. You should call this
method regularly.
\end{methoddesc}

\begin{methoddesc}[connection]{Abort}{}
Abort an outstanding asynchronous \method{Open()} or \method{Listen()}.
\end{methoddesc}

\begin{methoddesc}[connection]{Reset}{}
Reset a connection. Exact meaning depends on the tool.
\end{methoddesc}

\begin{methoddesc}[connection]{Break}{length}
Send a break. Whether this means anything, what it means and
interpretation of the \var{length} parameter depends on the tool in
use.
\end{methoddesc}

\section{\module{macconsole} ---
         Think C's console package}

\declaremodule{builtin}{macconsole}
  \platform{Mac}
\modulesynopsis{Think C's console package.}


This module is available on the Macintosh, provided Python has been
built using the Think C compiler. It provides an interface to the
Think console package, with which basic text windows can be created.

\begin{datadesc}{options}
An object allowing you to set various options when creating windows,
see below.
\end{datadesc}

\begin{datadesc}{C_ECHO}
\dataline{C_NOECHO}
\dataline{C_CBREAK}
\dataline{C_RAW}
Options for the \code{setmode} method. \constant{C_ECHO} and
\constant{C_CBREAK} enable character echo, the other two disable it,
\constant{C_ECHO} and \constant{C_NOECHO} enable line-oriented input
(erase/kill processing, etc).
\end{datadesc}

\begin{funcdesc}{copen}{}
Open a new console window. Return a console window object.
\end{funcdesc}

\begin{funcdesc}{fopen}{fp}
Return the console window object corresponding with the given file
object. \var{fp} should be one of \code{sys.stdin}, \code{sys.stdout} or
\code{sys.stderr}.
\end{funcdesc}

\subsection{macconsole options object}
These options are examined when a window is created:

\setindexsubitem{(macconsole option)}
\begin{datadesc}{top}
\dataline{left}
The origin of the window.
\end{datadesc}

\begin{datadesc}{nrows}
\dataline{ncols}
The size of the window.
\end{datadesc}

\begin{datadesc}{txFont}
\dataline{txSize}
\dataline{txStyle}
The font, fontsize and fontstyle to be used in the window.
\end{datadesc}

\begin{datadesc}{title}
The title of the window.
\end{datadesc}

\begin{datadesc}{pause_atexit}
If set non-zero, the window will wait for user action before closing.
\end{datadesc}

\subsection{console window object}

\setindexsubitem{(console window attribute)}

\begin{datadesc}{file}
The file object corresponding to this console window. If the file is
buffered, you should call \code{\var{file}.flush()} between
\code{write()} and \code{read()} calls.
\end{datadesc}

\setindexsubitem{(console window method)}

\begin{funcdesc}{setmode}{mode}
Set the input mode of the console to \constant{C_ECHO}, etc.
\end{funcdesc}

\begin{funcdesc}{settabs}{n}
Set the tabsize to \var{n} spaces.
\end{funcdesc}

\begin{funcdesc}{cleos}{}
Clear to end-of-screen.
\end{funcdesc}

\begin{funcdesc}{cleol}{}
Clear to end-of-line.
\end{funcdesc}

\begin{funcdesc}{inverse}{onoff}
Enable inverse-video mode:\ characters with the high bit set are
displayed in inverse video (this disables the upper half of a
non-\ASCII{} character set).
\end{funcdesc}

\begin{funcdesc}{gotoxy}{x, y}
Set the cursor to position \code{(\var{x}, \var{y})}.
\end{funcdesc}

\begin{funcdesc}{hide}{}
Hide the window, remembering the contents.
\end{funcdesc}

\begin{funcdesc}{show}{}
Show the window again.
\end{funcdesc}

\begin{funcdesc}{echo2printer}{}
Copy everything written to the window to the printer as well.
\end{funcdesc}

\section{\module{macdnr} ---
         Interface to the Macintosh Domain Name Resolver}

\declaremodule{builtin}{macdnr}
  \platform{Mac}
\modulesynopsis{Interfaces to the Macintosh Domain Name Resolver.}


This module provides an interface to the Macintosh Domain Name
Resolver.  It is usually used in conjunction with the \refmodule{mactcp}
module, to map hostnames to IP addresses.  It may not be available in
all Mac Python versions.
\index{Macintosh Domain Name Resolver}
\index{Domain Name Resolver, Macintosh}

The \module{macdnr} module defines the following functions:


\begin{funcdesc}{Open}{\optional{filename}}
Open the domain name resolver extension.  If \var{filename} is given it
should be the pathname of the extension, otherwise a default is
used.  Normally, this call is not needed since the other calls will
open the extension automatically.
\end{funcdesc}

\begin{funcdesc}{Close}{}
Close the resolver extension.  Again, not needed for normal use.
\end{funcdesc}

\begin{funcdesc}{StrToAddr}{hostname}
Look up the IP address for \var{hostname}.  This call returns a dnr
result object of the ``address'' variation.
\end{funcdesc}

\begin{funcdesc}{AddrToName}{addr}
Do a reverse lookup on the 32-bit integer IP-address
\var{addr}.  Returns a dnr result object of the ``address'' variation.
\end{funcdesc}

\begin{funcdesc}{AddrToStr}{addr}
Convert the 32-bit integer IP-address \var{addr} to a dotted-decimal
string.  Returns the string.
\end{funcdesc}

\begin{funcdesc}{HInfo}{hostname}
Query the nameservers for a \code{HInfo} record for host
\var{hostname}.  These records contain hardware and software
information about the machine in question (if they are available in
the first place).  Returns a dnr result object of the ``hinfo''
variety.
\end{funcdesc}

\begin{funcdesc}{MXInfo}{domain}
Query the nameservers for a mail exchanger for \var{domain}.  This is
the hostname of a host willing to accept SMTP\index{SMTP} mail for the
given domain.  Returns a dnr result object of the ``mx'' variety.
\end{funcdesc}


\subsection{DNR Result Objects \label{dnr-result-object}}

Since the DNR calls all execute asynchronously you do not get the
results back immediately.  Instead, you get a dnr result object.  You
can check this object to see whether the query is complete, and access
its attributes to obtain the information when it is.

Alternatively, you can also reference the result attributes directly,
this will result in an implicit wait for the query to complete.

The \member{rtnCode} and \member{cname} attributes are always
available, the others depend on the type of query (address, hinfo or
mx).


% Add args, as in {arg1, arg2 \optional{, arg3}}
\begin{methoddesc}[dnr result]{wait}{}
Wait for the query to complete.
\end{methoddesc}

% Add args, as in {arg1, arg2 \optional{, arg3}}
\begin{methoddesc}[dnr result]{isdone}{}
Return \code{1} if the query is complete.
\end{methoddesc}


\begin{memberdesc}[dnr result]{rtnCode}
The error code returned by the query.
\end{memberdesc}

\begin{memberdesc}[dnr result]{cname}
The canonical name of the host that was queried.
\end{memberdesc}

\begin{memberdesc}[dnr result]{ip0}
\memberline{ip1}
\memberline{ip2}
\memberline{ip3}
At most four integer IP addresses for this host.  Unused entries are
zero.  Valid only for address queries.
\end{memberdesc}

\begin{memberdesc}[dnr result]{cpuType}
\memberline{osType}
Textual strings giving the machine type an OS name.  Valid for ``hinfo''
queries.
\end{memberdesc}

\begin{memberdesc}[dnr result]{exchange}
The name of a mail-exchanger host.  Valid for ``mx'' queries.
\end{memberdesc}

\begin{memberdesc}[dnr result]{preference}
The preference of this mx record.  Not too useful, since the Macintosh
will only return a single mx record.  Valid for ``mx'' queries only.
\end{memberdesc}

The simplest way to use the module to convert names to dotted-decimal
strings, without worrying about idle time, etc:

\begin{verbatim}
>>> def gethostname(name):
...     import macdnr
...     dnrr = macdnr.StrToAddr(name)
...     return macdnr.AddrToStr(dnrr.ip0)
\end{verbatim}

\section{\module{macfs} ---
         Various file system services}

\declaremodule{builtin}{macfs}
  \platform{Mac}
\modulesynopsis{Support for FSSpec, the Alias Manager,
                \program{finder} aliases, and the Standard File package.}


This module provides access to Macintosh FSSpec handling, the Alias
Manager, \program{finder} aliases and the Standard File package.
\index{Macintosh Alias Manager}
\index{Alias Manager, Macintosh}
\index{Standard File}

Whenever a function or method expects a \var{file} argument, this
argument can be one of three things:\ (1) a full or partial Macintosh
pathname, (2) an \pytype{FSSpec} object or (3) a 3-tuple \code{(\var{wdRefNum},
\var{parID}, \var{name})} as described in \citetitle{Inside
Macintosh:\ Files}. A description of aliases and the Standard File
package can also be found there.

\begin{funcdesc}{FSSpec}{file}
Create an \pytype{FSSpec} object for the specified file.
\end{funcdesc}

\begin{funcdesc}{RawFSSpec}{data}
Create an \pytype{FSSpec} object given the raw data for the \C{}
structure for the \pytype{FSSpec} as a string.  This is mainly useful
if you have obtained an \pytype{FSSpec} structure over a network.
\end{funcdesc}

\begin{funcdesc}{RawAlias}{data}
Create an \pytype{Alias} object given the raw data for the \C{}
structure for the alias as a string.  This is mainly useful if you
have obtained an \pytype{FSSpec} structure over a network.
\end{funcdesc}

\begin{funcdesc}{FInfo}{}
Create a zero-filled \pytype{FInfo} object.
\end{funcdesc}

\begin{funcdesc}{ResolveAliasFile}{file}
Resolve an alias file. Returns a 3-tuple \code{(\var{fsspec},
\var{isfolder}, \var{aliased})} where \var{fsspec} is the resulting
\pytype{FSSpec} object, \var{isfolder} is true if \var{fsspec} points
to a folder and \var{aliased} is true if the file was an alias in the
first place (otherwise the \pytype{FSSpec} object for the file itself
is returned).
\end{funcdesc}

\begin{funcdesc}{StandardGetFile}{\optional{type, ...}}
Present the user with a standard ``open input file''
dialog. Optionally, you can pass up to four 4-character file types to limit
the files the user can choose from. The function returns an \pytype{FSSpec}
object and a flag indicating that the user completed the dialog
without cancelling.
\end{funcdesc}

\begin{funcdesc}{PromptGetFile}{prompt\optional{, type, ...}}
Similar to \function{StandardGetFile()} but allows you to specify a
prompt.
\end{funcdesc}

\begin{funcdesc}{StandardPutFile}{prompt, \optional{default}}
Present the user with a standard ``open output file''
dialog. \var{prompt} is the prompt string, and the optional
\var{default} argument initializes the output file name. The function
returns an \pytype{FSSpec} object and a flag indicating that the user
completed the dialog without cancelling.
\end{funcdesc}

\begin{funcdesc}{GetDirectory}{\optional{prompt}}
Present the user with a non-standard ``select a directory''
dialog. \var{prompt} is the prompt string, and the optional.
Return an \pytype{FSSpec} object and a success-indicator.
\end{funcdesc}

\begin{funcdesc}{SetFolder}{\optional{fsspec}}
Set the folder that is initially presented to the user when one of
the file selection dialogs is presented. \var{fsspec} should point to
a file in the folder, not the folder itself (the file need not exist,
though). If no argument is passed the folder will be set to the
current directory, i.e. what \function{os.getcwd()} returns.

Note that starting with system 7.5 the user can change Standard File
behaviour with the ``general controls'' controlpanel, thereby making
this call inoperative.
\end{funcdesc}

\begin{funcdesc}{FindFolder}{where, which, create}
Locates one of the ``special'' folders that MacOS knows about, such as
the trash or the Preferences folder. \var{where} is the disk to
search, \var{which} is the 4-character string specifying which folder to
locate. Setting \var{create} causes the folder to be created if it
does not exist. Returns a \code{(\var{vrefnum}, \var{dirid})} tuple.
\end{funcdesc}

\begin{funcdesc}{NewAliasMinimalFromFullPath}{pathname}
Return a minimal \pytype{alias} object that points to the given file, which
must be specified as a full pathname. This is the only way to create an
\pytype{Alias} pointing to a non-existing file.

The constants for \var{where} and \var{which} can be obtained from the
standard module \var{MACFS}.
\end{funcdesc}

\begin{funcdesc}{FindApplication}{creator}
Locate the application with 4-char creator code \var{creator}. The
function returns an \pytype{FSSpec} object pointing to the application.
\end{funcdesc}


\subsection{FSSpec objects \label{fsspec-objects}}

\begin{memberdesc}[FSSpec]{data}
The raw data from the FSSpec object, suitable for passing
to other applications, for instance.
\end{memberdesc}

\begin{methoddesc}[FSSpec]{as_pathname}{}
Return the full pathname of the file described by the \pytype{FSSpec}
object.
\end{methoddesc}

\begin{methoddesc}[FSSpec]{as_tuple}{}
Return the \code{(\var{wdRefNum}, \var{parID}, \var{name})} tuple of
the file described by the \pytype{FSSpec} object.
\end{methoddesc}

\begin{methoddesc}[FSSpec]{NewAlias}{\optional{file}}
Create an Alias object pointing to the file described by this
FSSpec. If the optional \var{file} parameter is present the alias
will be relative to that file, otherwise it will be absolute.
\end{methoddesc}

\begin{methoddesc}[FSSpec]{NewAliasMinimal}{}
Create a minimal alias pointing to this file.
\end{methoddesc}

\begin{methoddesc}[FSSpec]{GetCreatorType}{}
Return the 4-character creator and type of the file.
\end{methoddesc}

\begin{methoddesc}[FSSpec]{SetCreatorType}{creator, type}
Set the 4-character creator and type of the file.
\end{methoddesc}

\begin{methoddesc}[FSSpec]{GetFInfo}{}
Return a \pytype{FInfo} object describing the finder info for the file.
\end{methoddesc}

\begin{methoddesc}[FSSpec]{SetFInfo}{finfo}
Set the finder info for the file to the values given as \var{finfo}
(an \pytype{FInfo} object).
\end{methoddesc}

\begin{methoddesc}[FSSpec]{GetDates}{}
Return a tuple with three floating point values representing the
creation date, modification date and backup date of the file.
\end{methoddesc}

\begin{methoddesc}[FSSpec]{SetDates}{crdate, moddate, backupdate}
Set the creation, modification and backup date of the file. The values
are in the standard floating point format used for times throughout
Python.
\end{methoddesc}


\subsection{Alias Objects \label{alias-objects}}

\begin{memberdesc}[Alias]{data}
The raw data for the Alias record, suitable for storing in a resource
or transmitting to other programs.
\end{memberdesc}

\begin{methoddesc}[Alias]{Resolve}{\optional{file}}
Resolve the alias. If the alias was created as a relative alias you
should pass the file relative to which it is. Return the FSSpec for
the file pointed to and a flag indicating whether the \pytype{Alias} object
itself was modified during the search process. If the file does
not exist but the path leading up to it does exist a valid fsspec
is returned.
\end{methoddesc}

\begin{methoddesc}[Alias]{GetInfo}{num}
An interface to the \C{} routine \cfunction{GetAliasInfo()}.
\end{methoddesc}

\begin{methoddesc}[Alias]{Update}{file, \optional{file2}}
Update the alias to point to the \var{file} given. If \var{file2} is
present a relative alias will be created.
\end{methoddesc}

Note that it is currently not possible to directly manipulate a
resource as an \pytype{Alias} object. Hence, after calling
\method{Update()} or after \method{Resolve()} indicates that the alias
has changed the Python program is responsible for getting the
\member{data} value from the \pytype{Alias} object and modifying the
resource.


\subsection{FInfo Objects \label{finfo-objects}}

See \citetitle{Inside Macintosh: Files} for a complete description of what
the various fields mean.

\begin{memberdesc}[FInfo]{Creator}
The 4-character creator code of the file.
\end{memberdesc}

\begin{memberdesc}[FInfo]{Type}
The 4-character type code of the file.
\end{memberdesc}

\begin{memberdesc}[FInfo]{Flags}
The finder flags for the file as 16-bit integer. The bit values in
\var{Flags} are defined in standard module \module{MACFS}.
\end{memberdesc}

\begin{memberdesc}[FInfo]{Location}
A Point giving the position of the file's icon in its folder.
\end{memberdesc}

\begin{memberdesc}[FInfo]{Fldr}
The folder the file is in (as an integer).
\end{memberdesc}

\section{\module{ic} ---
         Access to Internet Config}

\declaremodule{builtin}{ic}
  \platform{Mac}
\modulesynopsis{Access to Internet Config.}


This module provides access to Macintosh Internet
Config\index{Internet Config} package,
which stores preferences for Internet programs such as mail address,
default homepage, etc. Also, Internet Config contains an elaborate set
of mappings from Macintosh creator/type codes to foreign filename
extensions plus information on how to transfer files (binary, ascii,
etc).

There is a low-level companion module
\module{icglue}\refbimodindex{icglue} which provides the basic
Internet Config access functionality.  This low-level module is not
documented, but the docstrings of the routines document the parameters
and the routine names are the same as for the Pascal or \C{} API to
Internet Config, so the standard IC programmers' documentation can be
used if this module is needed.

The \module{ic} module defines the \exception{error} exception and
symbolic names for all error codes Internet Config can produce; see
the source for details.

\begin{excdesc}{error}
Exception raised on errors in the \module{ic} module.
\end{excdesc}


The \module{ic} module defines the following class and function:

\begin{classdesc}{IC}{\optional{signature\optional{, ic}}}
Create an internet config object. The signature is a 4-character creator
code of the current application (default \code{'Pyth'}) which may
influence some of ICs settings. The optional \var{ic} argument is a
low-level \code{icglue.icinstance} created beforehand, this may be
useful if you want to get preferences from a different config file,
etc.
\end{classdesc}

\begin{funcdesc}{launchurl}{url\optional{, hint}}
\funcline{parseurl}{data\optional{, start\optional{, end\optional{, hint}}}}
\funcline{mapfile}{file}
\funcline{maptypecreator}{type, creator\optional{, filename}}
\funcline{settypecreator}{file}
These functions are ``shortcuts'' to the methods of the same name,
described below.
\end{funcdesc}


\subsection{IC Objects}

\class{IC} objects have a mapping interface, hence to obtain the mail
address you simply get \code{\var{ic}['MailAddress']}. Assignment also
works, and changes the option in the configuration file.

The module knows about various datatypes, and converts the internal IC
representation to a ``logical'' Python data structure. Running the
\module{ic} module standalone will run a test program that lists all
keys and values in your IC database, this will have to server as
documentation.

If the module does not know how to represent the data it returns an
instance of the \code{ICOpaqueData} type, with the raw data in its
\member{data} attribute. Objects of this type are also acceptable values
for assignment.

Besides the dictionary interface, \class{IC} objects have the
following methods:


\begin{methoddesc}{launchurl}{url\optional{, hint}}
Parse the given URL, lauch the correct application and pass it the
URL. The optional \var{hint} can be a scheme name such as
\code{'mailto:'}, in which case incomplete URLs are completed with this
scheme.  If \var{hint} is not provided, incomplete URLs are invalid.
\end{methoddesc}

\begin{methoddesc}{parseurl}{data\optional{, start\optional{, end\optional{, hint}}}}
Find an URL somewhere in \var{data} and return start position, end
position and the URL. The optional \var{start} and \var{end} can be
used to limit the search, so for instance if a user clicks in a long
textfield you can pass the whole textfield and the click-position in
\var{start} and this routine will return the whole URL in which the
user clicked.  As above, \var{hint} is an optional scheme used to
complete incomplete URLs.
\end{methoddesc}

\begin{methoddesc}{mapfile}{file}
Return the mapping entry for the given \var{file}, which can be passed
as either a filename or an \function{macfs.FSSpec()} result, and which
need not exist.

The mapping entry is returned as a tuple \code{(\var{version},
\var{type}, \var{creator}, \var{postcreator}, \var{flags},
\var{extension}, \var{appname}, \var{postappname}, \var{mimetype},
\var{entryname})}, where \var{version} is the entry version
number, \var{type} is the 4-character filetype, \var{creator} is the
4-character creator type, \var{postcreator} is the 4-character creator
code of an
optional application to post-process the file after downloading,
\var{flags} are various bits specifying whether to transfer in binary
or ascii and such, \var{extension} is the filename extension for this
file type, \var{appname} is the printable name of the application to
which this file belongs, \var{postappname} is the name of the
postprocessing application, \var{mimetype} is the MIME type of this
file and \var{entryname} is the name of this entry.
\end{methoddesc}

\begin{methoddesc}{maptypecreator}{type, creator\optional{, filename}}
Return the mapping entry for files with given 4-character \var{type} and
\var{creator} codes. The optional \var{filename} may be specified to
further help finding the correct entry (if the creator code is
\code{'????'}, for instance).

The mapping entry is returned in the same format as for \var{mapfile}.
\end{methoddesc}

\begin{methoddesc}{settypecreator}{file}
Given an existing \var{file}, specified either as a filename or as an
\function{macfs.FSSpec()} result, set its creator and type correctly based
on its extension.  The finder is told about the change, so the finder
icon will be updated quickly.
\end{methoddesc}

\section{\module{MacOS} ---
         Access to MacOS interpreter features}

\declaremodule{builtin}{MacOS}
  \platform{Mac}
\modulesynopsis{Access to MacOS specific interpreter features.}


This module provides access to MacOS specific functionality in the
Python interpreter, such as how the interpreter eventloop functions
and the like. Use with care.

Note the capitalisation of the module name, this is a historical
artifact.

\begin{excdesc}{Error}
This exception is raised on MacOS generated errors, either from
functions in this module or from other mac-specific modules like the
toolbox interfaces. The arguments are the integer error code (the
\cdata{OSErr} value) and a textual description of the error code.
Symbolic names for all known error codes are defined in the standard
module \module{macerrors}\refstmodindex{macerrors}.
\end{excdesc}

\begin{funcdesc}{SetEventHandler}{handler}
In the inner interpreter loop Python will occasionally check for events,
unless disabled with \function{ScheduleParams()}. With this function you
can pass a Python event-handler function that will be called if an event
is available. The event is passed as parameter and the function should return
non-zero if the event has been fully processed, otherwise event processing
continues (by passing the event to the console window package, for instance).

Call \function{SetEventHandler()} without a parameter to clear the
event handler. Setting an event handler while one is already set is an
error.
\end{funcdesc}

\begin{funcdesc}{SchedParams}{\optional{doint\optional{, evtmask\optional{,
                              besocial\optional{, interval\optional{,
                              bgyield}}}}}}
Influence the interpreter inner loop event handling. \var{Interval}
specifies how often (in seconds, floating point) the interpreter
should enter the event processing code. When true, \var{doint} causes
interrupt (command-dot) checking to be done. \var{evtmask} tells the
interpreter to do event processing for events in the mask (redraws,
mouseclicks to switch to other applications, etc). The \var{besocial}
flag gives other processes a chance to run. They are granted minimal
runtime when Python is in the foreground and \var{bgyield} seconds per
\var{interval} when Python runs in the background.

All parameters are optional, and default to the current value. The return
value of this function is a tuple with the old values of these options.
Initial defaults are that all processing is enabled, checking is done every
quarter second and the CPU is given up for a quarter second when in the
background.
\end{funcdesc}

\begin{funcdesc}{HandleEvent}{ev}
Pass the event record \var{ev} back to the Python event loop, or
possibly to the handler for the \code{sys.stdout} window (based on the
compiler used to build Python). This allows Python programs that do
their own event handling to still have some command-period and
window-switching capability.

If you attempt to call this function from an event handler set through
\function{SetEventHandler()} you will get an exception.
\end{funcdesc}

\begin{funcdesc}{GetErrorString}{errno}
Return the textual description of MacOS error code \var{errno}.
\end{funcdesc}

\begin{funcdesc}{splash}{resid}
This function will put a splash window
on-screen, with the contents of the DLOG resource specified by
\var{resid}. Calling with a zero argument will remove the splash
screen. This function is useful if you want an applet to post a splash screen
early in initialization without first having to load numerous
extension modules.
\end{funcdesc}

\begin{funcdesc}{DebugStr}{message \optional{, object}}
Drop to the low-level debugger with message \var{message}. The
optional \var{object} argument is not used, but can easily be
inspected from the debugger.

Note that you should use this function with extreme care: if no
low-level debugger like MacsBug is installed this call will crash your
system. It is intended mainly for developers of Python extension
modules.
\end{funcdesc}

\begin{funcdesc}{openrf}{name \optional{, mode}}
Open the resource fork of a file. Arguments are the same as for the
built-in function \function{open()}. The object returned has file-like
semantics, but it is not a Python file object, so there may be subtle
differences.
\end{funcdesc}

\section{\module{macostools} ---
         Convenience routines for file manipulation}

\declaremodule{standard}{macostools}
  \platform{Mac}
\modulesynopsis{Convenience routines for file manipulation.}


This module contains some convenience routines for file-manipulation
on the Macintosh.

The \module{macostools} module defines the following functions:


\begin{funcdesc}{copy}{src, dst\optional{, createpath\optional{, copytimes}}}
Copy file \var{src} to \var{dst}. The files can be specified as
pathnames or \pytype{FSSpec} objects. If \var{createpath} is non-zero
\var{dst} must be a pathname and the folders leading to the
destination are created if necessary.  The method copies data and
resource fork and some finder information (creator, type, flags) and
optionally the creation, modification and backup times (default is to
copy them). Custom icons, comments and icon position are not copied.

If the source is an alias the original to which the alias points is
copied, not the aliasfile.
\end{funcdesc}

\begin{funcdesc}{copytree}{src, dst}
Recursively copy a file tree from \var{src} to \var{dst}, creating
folders as needed. \var{src} and \var{dst} should be specified as
pathnames.
\end{funcdesc}

\begin{funcdesc}{mkalias}{src, dst}
Create a finder alias \var{dst} pointing to \var{src}. Both may be
specified as pathnames or \pytype{FSSpec} objects.
\end{funcdesc}

\begin{funcdesc}{touched}{dst}
Tell the finder that some bits of finder-information such as creator
or type for file \var{dst} has changed. The file can be specified by
pathname or fsspec. This call should prod the finder into redrawing the
files icon.
\end{funcdesc}

\begin{datadesc}{BUFSIZ}
The buffer size for \code{copy}, default 1 megabyte.
\end{datadesc}

Note that the process of creating finder aliases is not specified in
the Apple documentation. Hence, aliases created with \function{mkalias()}
could conceivably have incompatible behaviour in some cases.


\section{\module{findertools} ---
         The \program{finder}'s Apple Events interface}

\declaremodule{standard}{findertools}
  \platform{Mac}
\modulesynopsis{Wrappers around the \program{finder}'s Apple Events interface.}


This module contains routines that give Python programs access to some
functionality provided by the finder. They are implemented as wrappers
around the AppleEvent\index{AppleEvents} interface to the finder.

All file and folder parameters can be specified either as full
pathnames or as \pytype{FSSpec} objects.

The \module{findertools} module defines the following functions:


\begin{funcdesc}{launch}{file}
Tell the finder to launch \var{file}. What launching means depends on the file:
applications are started, folders are opened and documents are opened
in the correct application.
\end{funcdesc}

\begin{funcdesc}{Print}{file}
Tell the finder to print a file (again specified by full pathname or
\pytype{FSSpec}). The behaviour is identical to selecting the file and using
the print command in the finder.
\end{funcdesc}

\begin{funcdesc}{copy}{file, destdir}
Tell the finder to copy a file or folder \var{file} to folder
\var{destdir}. The function returns an \pytype{Alias} object pointing to
the new file.
\end{funcdesc}

\begin{funcdesc}{move}{file, destdir}
Tell the finder to move a file or folder \var{file} to folder
\var{destdir}. The function returns an \pytype{Alias} object pointing to
the new file.
\end{funcdesc}

\begin{funcdesc}{sleep}{}
Tell the finder to put the Macintosh to sleep, if your machine
supports it.
\end{funcdesc}

\begin{funcdesc}{restart}{}
Tell the finder to perform an orderly restart of the machine.
\end{funcdesc}

\begin{funcdesc}{shutdown}{}
Tell the finder to perform an orderly shutdown of the machine.
\end{funcdesc}

\section{\module{mactcp} ---
         The MacTCP interfaces}

\declaremodule{builtin}{mactcp}
  \platform{Mac}
\modulesynopsis{The MacTCP interfaces.}


This module provides an interface to the Macintosh TCP/IP driver%
\index{MacTCP} MacTCP. There is an accompanying module,
\refmodule{macdnr}\refbimodindex{macdnr}, which provides an interface
to the name-server (allowing you to translate hostnames to IP
addresses), a module \module{MACTCPconst}\refstmodindex{MACTCPconst}
which has symbolic names for constants constants used by MacTCP. Since
the built-in module \module{socket}\refbimodindex{socket} is also
available on the Macintosh it is usually easier to use sockets instead
of the Macintosh-specific MacTCP API.

A complete description of the MacTCP interface can be found in the
Apple MacTCP API documentation.

\begin{funcdesc}{MTU}{}
Return the Maximum Transmit Unit (the packet size) of the network
interface.\index{Maximum Transmit Unit}
\end{funcdesc}

\begin{funcdesc}{IPAddr}{}
Return the 32-bit integer IP address of the network interface.
\end{funcdesc}

\begin{funcdesc}{NetMask}{}
Return the 32-bit integer network mask of the interface.
\end{funcdesc}

\begin{funcdesc}{TCPCreate}{size}
Create a TCP Stream object. \var{size} is the size of the receive
buffer, \code{4096} is suggested by various sources.
\end{funcdesc}

\begin{funcdesc}{UDPCreate}{size, port}
Create a UDP Stream object. \var{size} is the size of the receive
buffer (and, hence, the size of the biggest datagram you can receive
on this port). \var{port} is the UDP port number you want to receive
datagrams on, a value of zero will make MacTCP select a free port.
\end{funcdesc}


\subsection{TCP Stream Objects}

\begin{memberdesc}[TCP Stream]{asr}
\index{asynchronous service routine}
\index{service routine, asynchronous}
When set to a value different than \code{None} this should refer to a
function with two integer parameters:\ an event code and a detail. This
function will be called upon network-generated events such as urgent
data arrival.  Macintosh documentation calls this the
\dfn{asynchronous service routine}.  In addition, it is called with
eventcode \code{MACTCP.PassiveOpenDone} when a \method{PassiveOpen()}
completes. This is a Python addition to the MacTCP semantics.
It is safe to do further calls from \var{asr}.
\end{memberdesc}


\begin{methoddesc}[TCP Stream]{PassiveOpen}{port}
Wait for an incoming connection on TCP port \var{port} (zero makes the
system pick a free port). The call returns immediately, and you should
use \method{wait()} to wait for completion. You should not issue any method
calls other than \method{wait()}, \method{isdone()} or
\method{GetSockName()} before the call completes.
\end{methoddesc}

\begin{methoddesc}[TCP Stream]{wait}{}
Wait for \method{PassiveOpen()} to complete.
\end{methoddesc}

\begin{methoddesc}[TCP Stream]{isdone}{}
Return \code{1} if a \method{PassiveOpen()} has completed.
\end{methoddesc}

\begin{methoddesc}[TCP Stream]{GetSockName}{}
Return the TCP address of this side of a connection as a 2-tuple
\code{(\var{host}, \var{port})}, both integers.
\end{methoddesc}

\begin{methoddesc}[TCP Stream]{ActiveOpen}{lport, host, rport}
Open an outgoing connection to TCP address \code{(\var{host},
\var{rport})}. Use
local port \var{lport} (zero makes the system pick a free port). This
call blocks until the connection has been established.
\end{methoddesc}

\begin{methoddesc}[TCP Stream]{Send}{buf, push, urgent}
Send data \var{buf} over the connection. \var{push} and \var{urgent}
are flags as specified by the TCP standard.
\end{methoddesc}

\begin{methoddesc}[TCP Stream]{Rcv}{timeout}
Receive data. The call returns when \var{timeout} seconds have passed
or when (according to the MacTCP documentation) ``a reasonable amount
of data has been received''. The return value is a 3-tuple
\code{(\var{data}, \var{urgent}, \var{mark})}. If urgent data is
outstanding \code{Rcv} will always return that before looking at any
normal data. The first call returning urgent data will have the
\var{urgent} flag set, the last will have the \var{mark} flag set.
\end{methoddesc}

\begin{methoddesc}[TCP Stream]{Close}{}
Tell MacTCP that no more data will be transmitted on this
connection. The call returns when all data has been acknowledged by
the receiving side.
\end{methoddesc}

\begin{methoddesc}[TCP Stream]{Abort}{}
Forcibly close both sides of a connection, ignoring outstanding data.
\end{methoddesc}

\begin{methoddesc}[TCP Stream]{Status}{}
Return a TCP status object for this stream giving the current status
(see below).
\end{methoddesc}


\subsection{TCP Status Objects}

This object has no methods, only some members holding information on
the connection. A complete description of all fields in this objects
can be found in the Apple documentation. The most interesting ones are:

\begin{memberdesc}[TCP Status]{localHost}
\memberline{localPort}
\memberline{remoteHost}
\memberline{remotePort}
The integer IP-addresses and port numbers of both endpoints of the
connection. 
\end{memberdesc}

\begin{memberdesc}[TCP Status]{sendWindow}
The current window size.
\end{memberdesc}

\begin{memberdesc}[TCP Status]{amtUnackedData}
The number of bytes sent but not yet acknowledged. \code{sendWindow -
amtUnackedData} is what you can pass to \method{Send()} without
blocking.
\end{memberdesc}

\begin{memberdesc}[TCP Status]{amtUnreadData}
The number of bytes received but not yet read (what you can
\method{Recv()} without blocking).
\end{memberdesc}



\subsection{UDP Stream Objects}

Note that, unlike the name suggests, there is nothing stream-like
about UDP.


\begin{memberdesc}[UDP Stream]{asr}
\index{asynchronous service routine}
\index{service routine, asynchronous}
The asynchronous service routine to be called on events such as
datagram arrival without outstanding \code{Read} call. The \var{asr}
has a single argument, the event code.
\end{memberdesc}

\begin{memberdesc}[UDP Stream]{port}
A read-only member giving the port number of this UDP Stream.
\end{memberdesc}


\begin{methoddesc}[UDP Stream]{Read}{timeout}
Read a datagram, waiting at most \var{timeout} seconds (-1 is
infinite).  Return the data.
\end{methoddesc}

\begin{methoddesc}[UDP Stream]{Write}{host, port, buf}
Send \var{buf} as a datagram to IP-address \var{host}, port
\var{port}.
\end{methoddesc}

\section{\module{macspeech} ---
         Interface to the Macintosh Speech Manager}

\declaremodule{builtin}{macspeech}
  \platform{Mac}
\modulesynopsis{Interface to the Macintosh Speech Manager.}


This module provides an interface to the Macintosh Speech Manager,
\index{Macintosh Speech Manager}
\index{Speech Manager, Macintosh}
allowing you to let the Macintosh utter phrases. You need a version of
the Speech Manager extension (version 1 and 2 have been tested) in
your \file{Extensions} folder for this to work. The module does not
provide full access to all features of the Speech Manager yet.  It may
not be available in all Mac Python versions.

\begin{funcdesc}{Available}{}
Test availability of the Speech Manager extension (and, on the
PowerPC, the Speech Manager shared library). Return \code{0} or
\code{1}.
\end{funcdesc}

\begin{funcdesc}{Version}{}
Return the (integer) version number of the Speech Manager.
\end{funcdesc}

\begin{funcdesc}{SpeakString}{str}
Utter the string \var{str} using the default voice,
asynchronously. This aborts any speech that may still be active from
prior \function{SpeakString()} invocations.
\end{funcdesc}

\begin{funcdesc}{Busy}{}
Return the number of speech channels busy, system-wide.
\end{funcdesc}

\begin{funcdesc}{CountVoices}{}
Return the number of different voices available.
\end{funcdesc}

\begin{funcdesc}{GetIndVoice}{num}
Return a \pytype{Voice} object for voice number \var{num}.
\end{funcdesc}

\subsection{Voice Objects}
\label{voice-objects}

Voice objects contain the description of a voice. It is currently not
yet possible to access the parameters of a voice.

\setindexsubitem{(voice object method)}

\begin{methoddesc}[Voice]{GetGender}{}
Return the gender of the voice: \code{0} for male, \code{1} for female
and \code{-1} for neuter.
\end{methoddesc}

\begin{methoddesc}[Voice]{NewChannel}{}
Return a new Speech Channel object using this voice.
\end{methoddesc}

\subsection{Speech Channel Objects}
\label{speech-channel-objects}

A Speech Channel object allows you to speak strings with slightly more
control than \function{SpeakString()}, and allows you to use multiple
speakers at the same time. Please note that channel pitch and rate are
interrelated in some way, so that to make your Macintosh sing you will
have to adjust both.

\begin{methoddesc}[Speech Channel]{SpeakText}{str}
Start uttering the given string.
\end{methoddesc}

\begin{methoddesc}[Speech Channel]{Stop}{}
Stop babbling.
\end{methoddesc}

\begin{methoddesc}[Speech Channel]{GetPitch}{}
Return the current pitch of the channel, as a floating-point number.
\end{methoddesc}

\begin{methoddesc}[Speech Channel]{SetPitch}{pitch}
Set the pitch of the channel.
\end{methoddesc}

\begin{methoddesc}[Speech Channel]{GetRate}{}
Get the speech rate (utterances per minute) of the channel as a
floating point number.
\end{methoddesc}

\begin{methoddesc}[Speech Channel]{SetRate}{rate}
Set the speech rate of the channel.
\end{methoddesc}


\section{\module{EasyDialogs} ---
         Basic Macintosh dialogs}

\declaremodule{standard}{EasyDialogs}
  \platform{Mac}
\modulesynopsis{Basic Macintosh dialogs.}


The \module{EasyDialogs} module contains some simple dialogs for
the Macintosh, modelled after the
\module{stdwin}\refbimodindex{stdwin} dialogs with similar names. All
routines have an optional parameter \var{id} with which you can
override the DLOG resource used for the dialog, as long as the item
numbers correspond. See the source for details.
 
The \module{EasyDialogs} module defines the following functions:


\begin{funcdesc}{Message}{str}
A modal dialog with the message text \var{str}, which should be at
most 255 characters long, is displayed. Control is returned when the
user clicks ``OK''.
\end{funcdesc}

\begin{funcdesc}{AskString}{prompt\optional{, default}}
Ask the user to input a string value, in a modal dialog. \var{prompt}
is the promt message, the optional \var{default} arg is the initial
value for the string. All strings can be at most 255 bytes
long. \function{AskString()} returns the string entered or \code{None}
in case the user cancelled.
\end{funcdesc}

\begin{funcdesc}{AskYesNoCancel}{question\optional{, default}}
Present a dialog with text \var{question} and three buttons labelled
``yes'', ``no'' and ``cancel''. Return \code{1} for yes, \code{0} for
no and \code{-1} for cancel. The default return value chosen by
hitting return is \code{0}. This can be changed with the optional
\var{default} argument.
\end{funcdesc}

\begin{funcdesc}{ProgressBar}{\optional{label\optional{, maxval}}}
Display a modeless progress dialog with a thermometer bar. \var{label}
is the text string displayed (default ``Working...''), \var{maxval} is
the value at which progress is complete (default \code{100}). The
returned object has one method, \code{set(\var{value})}, which sets
the value of the progress bar. The bar remains visible until the
object returned is discarded.

The progress bar has a ``cancel'' button, but it is currently
non-functional.
\end{funcdesc}

Note that \module{EasyDialogs} does not currently use the notification
manager. This means that displaying dialogs while the program is in
the background will lead to unexpected results and possibly
crashes. Also, all dialogs are modeless and hence expect to be at the
top of the stacking order. This is true when the dialogs are created,
but windows that pop-up later (like a console window) may also result
in crashes.

\section{\module{FrameWork} ---
         Interactive application framework}

\declaremodule{standard}{FrameWork}
  \platform{Mac}
\modulesynopsis{Interactive application framework.}


The \module{FrameWork} module contains classes that together provide a
framework for an interactive Macintosh application. The programmer
builds an application by creating subclasses that override various
methods of the bases classes, thereby implementing the functionality
wanted. Overriding functionality can often be done on various
different levels, i.e. to handle clicks in a single dialog window in a
non-standard way it is not necessary to override the complete event
handling.

The \module{FrameWork} is still very much work-in-progress, and the
documentation describes only the most important functionality, and not
in the most logical manner at that. Examine the source or the examples
for more details.

The \module{FrameWork} module defines the following functions:


\begin{funcdesc}{Application}{}
An object representing the complete application. See below for a
description of the methods. The default \method{__init__()} routine
creates an empty window dictionary and a menu bar with an apple menu.
\end{funcdesc}

\begin{funcdesc}{MenuBar}{}
An object representing the menubar. This object is usually not created
by the user.
\end{funcdesc}

\begin{funcdesc}{Menu}{bar, title\optional{, after}}
An object representing a menu. Upon creation you pass the
\code{MenuBar} the menu appears in, the \var{title} string and a
position (1-based) \var{after} where the menu should appear (default:
at the end).
\end{funcdesc}

\begin{funcdesc}{MenuItem}{menu, title\optional{, shortcut, callback}}
Create a menu item object. The arguments are the menu to crate the
item it, the item title string and optionally the keyboard shortcut
and a callback routine. The callback is called with the arguments
menu-id, item number within menu (1-based), current front window and
the event record.

In stead of a callable object the callback can also be a string. In
this case menu selection causes the lookup of a method in the topmost
window and the application. The method name is the callback string
with \code{'domenu_'} prepended.

Calling the \code{MenuBar} \method{fixmenudimstate()} method sets the
correct dimming for all menu items based on the current front window.
\end{funcdesc}

\begin{funcdesc}{Separator}{menu}
Add a separator to the end of a menu.
\end{funcdesc}

\begin{funcdesc}{SubMenu}{menu, label}
Create a submenu named \var{label} under menu \var{menu}. The menu
object is returned.
\end{funcdesc}

\begin{funcdesc}{Window}{parent}
Creates a (modeless) window. \var{Parent} is the application object to
which the window belongs. The window is not displayed until later.
\end{funcdesc}

\begin{funcdesc}{DialogWindow}{parent}
Creates a modeless dialog window.
\end{funcdesc}

\begin{funcdesc}{windowbounds}{width, height}
Return a \code{(\var{left}, \var{top}, \var{right}, \var{bottom})}
tuple suitable for creation of a window of given width and height. The
window will be staggered with respect to previous windows, and an
attempt is made to keep the whole window on-screen. The window will
however always be exact the size given, so parts may be offscreen.
\end{funcdesc}

\begin{funcdesc}{setwatchcursor}{}
Set the mouse cursor to a watch.
\end{funcdesc}

\begin{funcdesc}{setarrowcursor}{}
Set the mouse cursor to an arrow.
\end{funcdesc}


\subsection{Application Objects \label{application-objects}}

Application objects have the following methods, among others:


\begin{methoddesc}[Application]{makeusermenus}{}
Override this method if you need menus in your application. Append the
menus to the attribute \member{menubar}.
\end{methoddesc}

\begin{methoddesc}[Application]{getabouttext}{}
Override this method to return a text string describing your
application.  Alternatively, override the \method{do_about()} method
for more elaborate ``about'' messages.
\end{methoddesc}

\begin{methoddesc}[Application]{mainloop}{\optional{mask\optional{, wait}}}
This routine is the main event loop, call it to set your application
rolling. \var{Mask} is the mask of events you want to handle,
\var{wait} is the number of ticks you want to leave to other
concurrent application (default 0, which is probably not a good
idea). While raising \var{self} to exit the mainloop is still
supported it is not recommended: call \code{self._quit()} instead.

The event loop is split into many small parts, each of which can be
overridden. The default methods take care of dispatching events to
windows and dialogs, handling drags and resizes, Apple Events, events
for non-FrameWork windows, etc.

In general, all event handlers should return \code{1} if the event is fully
handled and \code{0} otherwise (because the front window was not a FrameWork
window, for instance). This is needed so that update events and such
can be passed on to other windows like the Sioux console window.
Calling \function{MacOS.HandleEvent()} is not allowed within
\var{our_dispatch} or its callees, since this may result in an
infinite loop if the code is called through the Python inner-loop
event handler.
\end{methoddesc}

\begin{methoddesc}[Application]{asyncevents}{onoff}
Call this method with a nonzero parameter to enable
asynchronous event handling. This will tell the inner interpreter loop
to call the application event handler \var{async_dispatch} whenever events
are available. This will cause FrameWork window updates and the user
interface to remain working during long computations, but will slow the
interpreter down and may cause surprising results in non-reentrant code
(such as FrameWork itself). By default \var{async_dispatch} will immedeately
call \var{our_dispatch} but you may override this to handle only certain
events asynchronously. Events you do not handle will be passed to Sioux
and such.

The old on/off value is returned.
\end{methoddesc}

\begin{methoddesc}[Application]{_quit}{}
Terminate the running \method{mainloop()} call at the next convenient
moment.
\end{methoddesc}

\begin{methoddesc}[Application]{do_char}{c, event}
The user typed character \var{c}. The complete details of the event
can be found in the \var{event} structure. This method can also be
provided in a \code{Window} object, which overrides the
application-wide handler if the window is frontmost.
\end{methoddesc}

\begin{methoddesc}[Application]{do_dialogevent}{event}
Called early in the event loop to handle modeless dialog events. The
default method simply dispatches the event to the relevant dialog (not
through the the \code{DialogWindow} object involved). Override if you
need special handling of dialog events (keyboard shortcuts, etc).
\end{methoddesc}

\begin{methoddesc}[Application]{idle}{event}
Called by the main event loop when no events are available. The
null-event is passed (so you can look at mouse position, etc).
\end{methoddesc}


\subsection{Window Objects \label{window-objects}}

Window objects have the following methods, among others:

\setindexsubitem{(Window method)}

\begin{methoddesc}[Window]{open}{}
Override this method to open a window. Store the MacOS window-id in
\member{self.wid} and call the \method{do_postopen()} method to
register the window with the parent application.
\end{methoddesc}

\begin{methoddesc}[Window]{close}{}
Override this method to do any special processing on window
close. Call the \method{do_postclose()} method to cleanup the parent
state.
\end{methoddesc}

\begin{methoddesc}[Window]{do_postresize}{width, height, macoswindowid}
Called after the window is resized. Override if more needs to be done
than calling \code{InvalRect}.
\end{methoddesc}

\begin{methoddesc}[Window]{do_contentclick}{local, modifiers, event}
The user clicked in the content part of a window. The arguments are
the coordinates (window-relative), the key modifiers and the raw
event.
\end{methoddesc}

\begin{methoddesc}[Window]{do_update}{macoswindowid, event}
An update event for the window was received. Redraw the window.
\end{methoddesc}

\begin{methoddesc}{do_activate}{activate, event}
The window was activated (\code{\var{activate} == 1}) or deactivated
(\code{\var{activate} == 0}). Handle things like focus highlighting,
etc.
\end{methoddesc}


\subsection{ControlsWindow Object \label{controlswindow-object}}

ControlsWindow objects have the following methods besides those of
\code{Window} objects:


\begin{methoddesc}[ControlsWindow]{do_controlhit}{window, control,
                                                  pcode, event}
Part \var{pcode} of control \var{control} was hit by the
user. Tracking and such has already been taken care of.
\end{methoddesc}


\subsection{ScrolledWindow Object \label{scrolledwindow-object}}

ScrolledWindow objects are ControlsWindow objects with the following
extra methods:


\begin{methoddesc}[ScrolledWindow]{scrollbars}{\optional{wantx\optional{,
                                               wanty}}}
Create (or destroy) horizontal and vertical scrollbars. The arguments
specify which you want (default: both). The scrollbars always have
minimum \code{0} and maximum \code{32767}.
\end{methoddesc}

\begin{methoddesc}[ScrolledWindow]{getscrollbarvalues}{}
You must supply this method. It should return a tuple \code{(\var{x},
\var{y})} giving the current position of the scrollbars (between
\code{0} and \code{32767}). You can return \code{None} for either to
indicate the whole document is visible in that direction.
\end{methoddesc}

\begin{methoddesc}[ScrolledWindow]{updatescrollbars}{}
Call this method when the document has changed. It will call
\method{getscrollbarvalues()} and update the scrollbars.
\end{methoddesc}

\begin{methoddesc}[ScrolledWindow]{scrollbar_callback}{which, what, value}
Supplied by you and called after user interaction. \var{which} will
be \code{'x'} or \code{'y'}, \var{what} will be \code{'-'},
\code{'--'}, \code{'set'}, \code{'++'} or \code{'+'}. For
\code{'set'}, \var{value} will contain the new scrollbar position.
\end{methoddesc}

\begin{methoddesc}[ScrolledWindow]{scalebarvalues}{absmin, absmax,
                                                   curmin, curmax}
Auxiliary method to help you calculate values to return from
\method{getscrollbarvalues()}. You pass document minimum and maximum value
and topmost (leftmost) and bottommost (rightmost) visible values and
it returns the correct number or \code{None}.
\end{methoddesc}

\begin{methoddesc}[ScrolledWindow]{do_activate}{onoff, event}
Takes care of dimming/highlighting scrollbars when a window becomes
frontmost vv. If you override this method call this one at the end of
your method.
\end{methoddesc}

\begin{methoddesc}[ScrolledWindow]{do_postresize}{width, height, window}
Moves scrollbars to the correct position. Call this method initially
if you override it.
\end{methoddesc}

\begin{methoddesc}[ScrolledWindow]{do_controlhit}{window, control,
                                                  pcode, event}
Handles scrollbar interaction. If you override it call this method
first, a nonzero return value indicates the hit was in the scrollbars
and has been handled.
\end{methoddesc}


\subsection{DialogWindow Objects \label{dialogwindow-objects}}

DialogWindow objects have the following methods besides those of
\code{Window} objects:


\begin{methoddesc}[DialogWindow]{open}{resid}
Create the dialog window, from the DLOG resource with id
\var{resid}. The dialog object is stored in \member{self.wid}.
\end{methoddesc}

\begin{methoddesc}[DialogWindow]{do_itemhit}{item, event}
Item number \var{item} was hit. You are responsible for redrawing
toggle buttons, etc.
\end{methoddesc}

\section{\module{MiniAEFrame} ---
         Open Scripting Architecture server support}

\declaremodule{standard}{MiniAEFrame}
  \platform{Mac}
\modulesynopsis{Support to act as an Open Scripting Architecture (OSA) server
(``Apple Events'').}


The module \module{MiniAEFrame} provides a framework for an application
that can function as an Open Scripting Architecture
\index{Open Scripting Architecture}
(OSA) server, i.e. receive and process
AppleEvents\index{AppleEvents}. It can be used in conjunction with
\refmodule{FrameWork}\refstmodindex{FrameWork} or standalone.

This module is temporary, it will eventually be replaced by a module
that handles argument names better and possibly automates making your
application scriptable.

The \module{MiniAEFrame} module defines the following classes:


\begin{classdesc}{AEServer}{}
A class that handles AppleEvent dispatch. Your application should
subclass this class together with either
\class{MiniApplication} or
\class{FrameWork.Application}. Your \method{__init__()} method should
call the \method{__init__()} method for both classes.
\end{classdesc}

\begin{classdesc}{MiniApplication}{}
A class that is more or less compatible with
\class{FrameWork.Application} but with less functionality. Its
event loop supports the apple menu, command-dot and AppleEvents; other
events are passed on to the Python interpreter and/or Sioux.
Useful if your application wants to use \class{AEServer} but does not
provide its own windows, etc.
\end{classdesc}


\subsection{AEServer Objects \label{aeserver-objects}}

\begin{methoddesc}[AEServer]{installaehandler}{classe, type, callback}
Installs an AppleEvent handler. \var{classe} and \var{type} are the
four-character OSA Class and Type designators, \code{'****'} wildcards
are allowed. When a matching AppleEvent is received the parameters are
decoded and your callback is invoked.
\end{methoddesc}

\begin{methoddesc}[AEServer]{callback}{_object, **kwargs}
Your callback is called with the OSA Direct Object as first positional
parameter. The other parameters are passed as keyword arguments, with
the 4-character designator as name. Three extra keyword parameters are
passed: \code{_class} and \code{_type} are the Class and Type
designators and \code{_attributes} is a dictionary with the AppleEvent
attributes.

The return value of your method is packed with
\function{aetools.packevent()} and sent as reply.
\end{methoddesc}

Note that there are some serious problems with the current
design. AppleEvents which have non-identifier 4-character designators
for arguments are not implementable, and it is not possible to return
an error to the originator. This will be addressed in a future
release.


%
%  The ugly "%begin{latexonly}" pseudo-environments are really just to
%  keep LaTeX2HTML quiet during the \renewcommand{} macros; they're
%  not really valuable.
%

%begin{latexonly}
\renewcommand{\indexname}{Module Index}
%end{latexonly}
\input{modmac.ind}		% Module Index

%begin{latexonly}
\renewcommand{\indexname}{Index}
%end{latexonly}
\documentclass{howto}

\title{Macintosh Library Modules}

\input{boilerplate}

\makeindex			% tell \index to actually write the
				% .idx file
\makemodindex			% ... and the module index as well.
%\ignorePlatformAnnotation{Mac}


\begin{document}

\maketitle

\ifhtml
\chapter*{Front Matter\label{front}}
\fi

\input{copyright}

\begin{abstract}

\noindent
This library reference manual documents Python's extensions for the
Macintosh.  It should be used in conjunction with the
\citetitle[../lib/lib.html]{Python Library Reference}, which documents
the standard library and built-in types.

This manual assumes basic knowledge about the Python language.  For an
informal introduction to Python, see the
\citetitle[../tut/tut.html]{Python Tutorial}; the
\citetitle[../ref/ref.html]{Python Reference Manual} remains the
highest authority on syntactic and semantic questions.  Finally, the
manual entitled \citetitle[../ext/ext.html]{Extending and Embedding
the Python Interpreter} describes how to add new extensions to Python
and how to embed it in other applications.

\end{abstract}

\tableofcontents

\input{libmac}			% MACINTOSH ONLY
\input{libctb}
\input{libmacconsole}
\input{libmacdnr}
\input{libmacfs}
\input{libmacic}
\input{libmacos}
\input{libmacostools}
\input{libmactcp}
\input{libmacspeech}
\input{libmacui}
\input{libframework}
\input{libminiae}

%
%  The ugly "%begin{latexonly}" pseudo-environments are really just to
%  keep LaTeX2HTML quiet during the \renewcommand{} macros; they're
%  not really valuable.
%

%begin{latexonly}
\renewcommand{\indexname}{Module Index}
%end{latexonly}
\input{modmac.ind}		% Module Index

%begin{latexonly}
\renewcommand{\indexname}{Index}
%end{latexonly}
\input{mac.ind}			% Index

\end{document}
			% Index

\end{document}
			% Index

\end{document}
			% Index

\end{document}
