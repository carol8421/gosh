\section{\module{tokenize} ---
         Tokenizer for Python source}

\declaremodule{standard}{tokenize}
\modulesynopsis{Lexical scanner for Python source code.}
\moduleauthor{Ka Ping Yee}{}
\sectionauthor{Fred L. Drake, Jr.}{fdrake@acm.org}


The \module{tokenize} module provides a lexical scanner for Python
source code, implemented in Python.  The scanner in this module
returns comments as tokens as well, making it useful for implementing
``pretty-printers,'' including colorizers for on-screen displays.

The scanner is exposed by a single function:


\begin{funcdesc}{tokenize}{readline\optional{, tokeneater}}
  The \function{tokenize()} function accepts two parameters: one
  representing the input stream, and one providing an output mechanism 
  for \function{tokenize()}.

  The first parameter, \var{readline}, must be a callable object which
  provides the same interface as the \method{readline()} method of
  built-in file objects (see section~\ref{bltin-file-objects}).  Each
  call to the function should return one line of input as a string.

  The second parameter, \var{tokeneater}, must also be a callable
  object.  It is called with five parameters: the token type, the
  token string, a tuple \code{(\var{srow}, \var{scol})} specifying the 
  row and column where the token begins in the source, a tuple
  \code{(\var{erow}, \var{ecol})} giving the ending position of the
  token, and the line on which the token was found.  The line passed
  is the \emph{logical} line; continuation lines are included.
\end{funcdesc}


All constants from the \refmodule{token} module are also exported from 
\module{tokenize}, as is one additional token type value that might be 
passed to the \var{tokeneater} function by \function{tokenize()}:

\begin{datadesc}{COMMENT}
  Token value used to indicate a comment.
\end{datadesc}
