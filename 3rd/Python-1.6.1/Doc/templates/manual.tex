\documentclass{manual}

\title{Big Python Manual}

\author{Your Name Here}

% Please at least include a long-lived email address;
% the rest is at your discretion.
\authoraddress{
	Organization name, if applicable \\
	Street address, if you want to use it \\
	E-mail: \email{your-email@your.domain}
}

\date{April 30, 1999}		% update before release!
				% Use an explicit date so that reformatting
				% doesn't cause a new date to be used.  Setting
				% the date to \today can be used during draft
				% stages to make it easier to handle versions.

\release{x.y}			% release version; this is used to define the
				% \version macro

\makeindex			% tell \index to actually write the .idx file
\makemodindex			% If this contains a lot of module sections.


\begin{document}

\maketitle

% This makes the contents more accessible from the front page of the HTML.
\ifhtml
\chapter*{Front Matter\label{front}}
\fi

%Copyright \copyright{} 1995-2000 Corporation for National Research
Initiatives.  All rights reserved.

Copyright \copyright{} 1991-1995 Stichting Mathematisch Centrum.  All
rights reserved.

\centerline{\strong{CNRI OPEN SOURCE LICENSE AGREEMENT}}


IMPORTANT: PLEASE READ THE FOLLOWING AGREEMENT CAREFULLY.

BY CLICKING ON ``ACCEPT'' WHERE INDICATED BELOW, OR BY COPYING,
INSTALLING OR OTHERWISE USING PYTHON 1.6 SOFTWARE, YOU ARE DEEMED TO
HAVE AGREED TO THE TERMS AND CONDITIONS OF THIS LICENSE AGREEMENT.

\begin{enumerate}
\item
This LICENSE AGREEMENT is between the Corporation for National
Research Initiatives, having an office at 1895 Preston White Drive,
Reston, VA 20191 (``CNRI''), and the Individual or Organization
(``Licensee'') accessing and otherwise using Python 1.6 software in
source or binary form and its associated documentation, as released at
the www.python.org Internet site on September 5, 2000 (``Python 1.6'').

\item
Subject to the terms and conditions of this License Agreement, CNRI
hereby grants Licensee a nonexclusive, royalty-free, world-wide
license to reproduce, analyze, test, perform and/or display publicly,
prepare derivative works, distribute, and otherwise use Python 1.6
alone or in any derivative version, provided, however, that CNRI's
License Agreement and CNRI's notice of copyright, i.e., ``Copyright (c)
1995-2000 Corporation for National Research Initiatives; All Rights
Reserved'' are retained in Python 1.6 alone or in any derivative
version prepared by Licensee.

Alternately, in lieu of CNRI's License Agreement, Licensee may
substitute the following text (omitting the quotes): ``Python 1.6 is
made available subject to the terms and conditions in CNRI's License
Agreement.  This Agreement together with Python 1.6 may be located on
the Internet using the following unique, persistent identifier (known
as a handle): 1895.22/1012.  This Agreement may also be obtained from a
proxy server on the Internet using the following URL:
http://hdl.handle.net/1895.22/1012''.

\item
In the event Licensee prepares a derivative work that is based on
or incorporates Python 1.6 or any part thereof, and wants to make the
derivative work available to others as provided herein, then Licensee
hereby agrees to include in any such work a brief summary of the
changes made to Python 1.6.

\item
CNRI is making Python 1.6 available to Licensee on an ``AS IS''
basis.  CNRI MAKES NO REPRESENTATIONS OR WARRANTIES, EXPRESS OR
IMPLIED.  BY WAY OF EXAMPLE, BUT NOT LIMITATION, CNRI MAKES NO AND
DISCLAIMS ANY REPRESENTATION OR WARRANTY OF MERCHANTABILITY OR FITNESS
FOR ANY PARTICULAR PURPOSE OR THAT THE USE OF PYTHON 1.6 WILL NOT
INFRINGE ANY THIRD PARTY RIGHTS.

\item
CNRI SHALL NOT BE LIABLE TO LICENSEE OR ANY OTHER USERS OF PYTHON
1.6 FOR ANY INCIDENTAL, SPECIAL, OR CONSEQUENTIAL DAMAGES OR LOSS AS A
RESULT OF MODIFYING, DISTRIBUTING, OR OTHERWISE USING PYTHON 1.6, OR
ANY DERIVATIVE THEREOF, EVEN IF ADVISED OF THE POSSIBILITY THEREOF.

\item
This License Agreement will automatically terminate upon a material
breach of its terms and conditions.

\item
This License Agreement shall be governed by and interpreted in all
respects by the law of the State of Virginia, excluding conflict of
law provisions.  Nothing in this License Agreement shall be deemed to
create any relationship of agency, partnership, or joint venture
between CNRI and Licensee.  This License Agreement does not grant
permission to use CNRI trademarks or trade name in a trademark sense
to endorse or promote products or services of Licensee, or any third
party.

\item
By clicking on the ``ACCEPT'' button where indicated, or by copying,
installing or otherwise using Python 1.6, Licensee agrees to be bound
by the terms and conditions of this License Agreement.

\centerline{ACCEPT}
\end{enumerate}


\begin{abstract}

\noindent
Big Python is a special version of Python for users who require larger 
keys on their keyboards.  It accomodates their special needs by ...

\end{abstract}

\tableofcontents


\chapter{...}

My chapter.


\appendix
\chapter{...}

My appendix.

The \code{\e appendix} markup need not be repeated for additional
appendices.


%
%  The ugly "%begin{latexonly}" pseudo-environments are really just to
%  keep LaTeX2HTML quiet during the \renewcommand{} macros; they're
%  not really valuable.
%
%  If you don't want the Module Index, you can remove all of this up
%  until the second \input line.
%
%begin{latexonly}
\renewcommand{\indexname}{Module Index}
%end{latexonly}
\input{mod\jobname.ind}		% Module Index

%begin{latexonly}
\renewcommand{\indexname}{Index}
%end{latexonly}
\input{\jobname.ind}			% Index

\end{document}
